\chapter{Alcune strutture algebriche}
%\setcounter{page}{1}
\label{algebra}

\def\ceq#1#2#3{\parbox{30ex}{$\displaystyle #1$}\parbox{6ex}{\hfil$\displaystyle #2$}$\displaystyle  #3$}

\def\ceqb#1#2#3{\parbox{30ex}{\hfill\emph{$#1$}}\parbox{6ex}{\hfil\emph{\ $#2$}}\emph{$#3$}}


Il risultato pi\`u significativo di questo capitolo \`e il corollario~\ref{corol_acfElQuan}, l'eliminazione dei quantificatori per la teoria dei campi algebricamente chiusi. Un risultato che in geometria algebrica si enuncia dicendo che la proiezione di un insieme costruibile \`e a sua volta costruibile. Poi come esercizio deriveremo da questo il Nullstellensatz.

Qui non prenderemo la via pi\`u breve per arrivarci. Infatti quel che pi\`u ci preme \`e fornire una traduzione delle nozioni elementari di algebra in un linguaggio che facilita il confronto con quanto fatto nei capitoli precedenti. 

Nei primi tre paragrafi, per riscaldamento, discutiamo l'eliminazione dei quantificatori nei moduli divisibili (su un dominio di integrit\`a). Queste strutture generalizzano leggermente i campi vettoriali che, pur essendo un esempio pi\`u canonico, per la loro semplicit\`a sono didatticamente meno utili. 

A costo di qualche ripetizione, i restanti capitoli sono indipendenti dai primi tre. Chi volesse omettere la lettura dei primi capitoli, dovr\`a a suo tempo svolgere l'esercizio~\ref{EQ_stazi_vettoriali}.

Assumiamo familiarit\`a con le nozioni di base di algebra ma le principali definizioni verranno riprese. 

%%%%%%%%%%%%%%%%%%%%%%%%%%%%%%%%%%
%%%%%%%%%%%%%%%%%%%%%%%%%%%%%%%%%%
%%%%%%%%%%%%%%%%%%%%%%%%%%%%%%%%%%
%%%%%%%%%%%%%%%%%%%%%%%%%%%%%%%%%%
%%%%%%%%%%%%%%%%%%%%%%%%%%%%%%%%%%
\section{I moduli (un ripasso)}
\label{gruppi}
La teoria dei gruppi abeliani (o commutativi) \`e generalmente espressa nel linguaggio dei gruppi additivi e contiene i seguenti assiomi (come sempre, scriviamo delle formule intendendo la loro chiusura universale):
\begin{itemize}
\item[a1] $(x+y) +z\ =\ y+(x+z)$;
\item[a2] $x+(-x)\ =\ (-x)+x\ =\ 0$;
\item[a3] $x+0\ = \ 0+x\ = \ x$;
\item[a4] $x+y\ =\ y+x$.
\end{itemize}

Sia $R$ un anello (unitario). In algebra commutativa, gli anelli giocano il ruolo di linguaggio. Il \emph{linguaggio degli $R\jj$moduli}, qui denotato con \emph{$L_R$}, estende il linguaggio dei gruppi additivi con tutti gli elementi di $R$ quali simboli di funzione unaria. Un $R\jj$modulo \`e una struttura di segnatura $L_R$ che rende veri gli assiomi dei gruppi additivi \ssf{a1}-\ssf{a4} e i seguenti assiomi per $t,r,s\in R$

\begin{itemize}
\item[m1] $r\,(x+y)\ =\ r\,x+r\,y$
\item[m2] $t\,x\ =\ r\,x+s\,x$\hfill se \ \parbox{50ex}{$t\;=\;r+_Rs$}
\item[m3] $t x\ =\ -\,r\,x$\hfill se \    \parbox{50ex}{$t\;=\;-_R\,r$}
\item[m4] $t\,x\ =\ r\,(s\,x)$\hfill se \ \parbox{50ex}{$t\;=\;r\cdot_Rs$}
\item[m5] $0_R\,x\ =\ 0$
\item[m6] $1_R\,x\ =\ x$
\end{itemize}

Lo zero di $R$ non va confuso con lo zero di $M$: il primo \`e una funzione unaria il secondo una costante. Quindi abbiamo lo abbiamo denotato con $0_R$. Analogamente, abbiamo scritto $1_R$ per l'unit\`a dell'anello: $0_R$ ed $1_R$ denotano elementi di $L_R$, come anche $r+_Rs$, $-_R\,r$, e $r\cdot_Rs$ (ovvero i simboli $+_R$, $-_R$ e $\cdot_R$, vivono nel metalinguaggio). Avendo chiarito questo, nel seguito il pedice verr\`a omesso. Indicheremo on \emph{$T_\Rmod$\/} la teoria degli $R\jj$moduli.

I \emph{moduli destri\/} sono definiti analogamente sostituendo l'assioma \ssf{m4} con il seguente:

\begin{itemize}
\item[m4d] $(r\cdot_Rs)x\ =\ s(rx)$
\end{itemize}

Chiaramente se $R$ \`e un anello commutativo \ssf{m4} \`e equivalente ad \ssf{m4d}. L'aggettivo \textit{destro\/} deriva dall'osservazione che scrivendo lo scalare alla destra dell'elemento del modulo, l'assioma \ssf{m4d} diventa pi\`u naturale: $x\,(r\cdot_Rs)=(x\,r)\,s$.

Ogni gruppo abeliano \`e uno $\ZZ\jj$modulo definendo per $n$ intero positivo

\ceq{\hfill nx}{\deq}{\stackrel{n\ \rm volte}{\overbrace{x+\dots+x}},}

e, se $n$ \`e negativo, si definisce $nx\ \deq\ -(-n)x$. Infine $0x\ \deq\ 0$.

Diremo che $M$ \`e un \emph{modulo fedele\/} se per ogni $r\in R\sm\{0\}$ vale:
\begin{itemize}
\item[mf] $\E x \ rx\ \neq\ 0$
\end{itemize}
Per esempio un gruppo abeliano con esponente finito, se considerato come $\ZZ\jj$modulo, non \`e fedele. Quando $R$ \`e un campo, gli $R\jj$moduli si chiamano \emph{spazi vettoriali}. Si noti che se $r^{-1}$ esiste, allora $r\,x=0\ \iff\ r^{-1}r\,x=0\ \iff\ x=0$; quindi tutti i campi vettoriali sono in particolare moduli fedeli.
% 
% \begin{notation}[(non standard)]\label{notationConseguenzaConsistenzaModuli}
% Dato un tipo $p(x)\subseteq L$ ed una formula $\phi(x)\in L$ scriveremo $p(x)\provesR\phi(x)$ per affermare che $p(x)\imp\phi(x)$ vale in tutti gli $R\jj$moduli, ovvero $p(x)\ \cup\ T_\Rmod\proves\phi(x)$.
% 
% Molto spesso per\`o servir\`a usare la stessa notazione anche per tipi e formule con parametri. In questo caso ci limiteremo ad usarla con tipi e formule senza quantificatori. Dato un modulo $M$, un insieme $A\subseteq M$, ed un tipo $p(x)\subseteq L(A)$ ed una formula $\phi(x)\in L(A)$, scriveremo 
% 
% \hspace*{15ex}\llap{\emph{$p(x)\ \provesR\ \phi(x)$}}%
% \parbox{6ex}{\hfil$\dIFF$}
% $p(x)\ \cup\ T_\Rmod\ \cup\ \ \Diag(M/A)\ \proves\ \phi(x)$. 
% 
% Se tutte le formule in gioco sono senza quantificatori, queste nozioni non dipendono da $A$ ed $M$ ma solo dalla sottostruttura generata dai parametri che effettivamente occorrono in $p(x)$ e $\phi(x)$. 
% 
% La nozione di consistenza \`e adattata in modo analogo: diremo che $p(x)$ \`e consistente se \`e consistente con $T_\Rmod\cup\Diag(M/A)$ ovvero se \`e realizzato in un $R\jj$modulo che contiene $\<A\>_M$. 
% \end{notation}

Dato un ordinale $\alpha$, con $R^{\oplus\!\alpha}$ denoteremo l'insieme delle sequenze di elementi di $R$ di lunghezza $\alpha$ che sono quasi ovunque nulle. Considereremo $R^{\oplus\!\alpha}$ come $R\jj$modulo nel modo naturale. 

Se $x$ \`e una tupla di variabili di lunghezza $\alpha$, scriveremo $R[x]$ per l'insieme dei termini puri $t(x)$ quozientato per la relazione di equivalenza

\ceqb{\hfill t(x)\sim_R s(x)}%
{\deq}%
{T_\Rmod\proves t(x)=s(x).}

Anche $R[x]$ pu\`o essere considerato come un $R\jj$modulo nel modo naturale. Dalla seguente proposizione segue immediatamente che $R[x]$ \`e isomorfo a $R^{\oplus\!\alpha}$.

% \begin{notation}[(non standard)] Data una tupla $r\in R^{\oplus\!\alpha}$ ed una tupla di variabili $x$ di lunghezza $\alpha$, definiamo
% 
% \hspace*{15ex}\llap{\emph{$r\mdot x$}}%
% \parbox{6ex}{\hfil$\deq$}%
% $\displaystyle\sum_{i<\alpha}r_ix_i$.
% 
% La notazione \emph{$r\mdot a$}, dove $a$ \`e una tupla di parametri, \`e definita analogamente.\QED
% \end{notation}

\begin{proposition}
\label{prop_formacanonicaterminimoduli}
Fissiamo $x$, una tupla di variabili di lunghezza $\alpha$. Per ogni termine puro $t(x)$, dove $|x|=\alpha$, esiste una tupla $r\in R^{\oplus\alpha}$, tale che 

\ceq{\hfill T_\Rmod}{\proves}{\sum_{i<\alpha}r_ix_i\ =\ t(x)}.

dove $x=\<x_i:i<\alpha\>$ ed $r=\<r_i:i<\alpha\>$.\QED
\end{proposition}

% \begin{proof}
% L'esistenza si dimostra facilmente per induzione sulla sintassi di $t(x)$. Per dimostrare l'unicit\`a, supponiamo che \ssf{1} valga per entrambe le sequenze $r=\<r_i:i<\alpha\>$ ed $s=\<s_i:i<\alpha\>$. Segue immediatamente che per ogni $R\jj$modulo $M$
% 
% 
% \ceq{\ssf{2.}\hfill M}{\models}{\sum_{i<\alpha}(r_i-s_i)x_i\ =\ 0}
% 
% in particolare \ssf{2} vale in $R^{\oplus\alpha}$. Scriviamo \emph{$e_i$\/} per l'elemento di $R^{\oplus\!\alpha}$ che \`e $1$ nella coordinata $i\jj$esima e $0$ in tutte le altre. Se sostituiamo ad $x$ l'elemento $e_i$ otteniamo $r_i=s_i$.
% \end{proof}


\begin{corollary}
\label{corol_formacanonicaterminimoduli}
Con la stessa notazione della proposizione~\ref{prop_formacanonicaterminimoduli} Sia $M$ un $R\jj$modulo, e fissiamo $A\subseteq M$. Per ogni $\phi(x)\in\aL(A)$ esiste una tupla $r\in R^{\oplus\alpha}$ ed un termine chiuso $t$ a parametri in $A$, tale che 

\ceq{\hfill T_\Rmod\ \cup\ \Diag(M/A)}{\proves}{\phi(x)\ \iff\ \sum_{i<\alpha}r_ix_i\ =\ t}.\QED

\end{corollary}

Per la definizione $\Diag(M/A)$ e di $\atp_M(c/A)$ vedi paragrafo~\ref{frammenti}.

\begin{definition} Sia $M$ un $R\jj$modulo, e fissiamo $A\subseteq M$. Un elemento $c\in M$ si dice \emph{indipendente da $A$\/} se il tipo $p(x)=\atp_M(c/A)$ \`e banale, ovvero

\ceq{\hfill T_\Rmod\ \cup\ \Diag(M/A)}{\proves}{p(x).}  

Equivalentemente, se $\<A\>_M\cap\<c\>_M=\{0\}$.\QED
\end{definition}  

Questa definizione trasporta ai moduli la nozione di indipendenza lineare nel caso $M$ sia uno spazio vettoriale. In questo caso $c$ \`e indipendente da $A$ se e solo se \`e linearmente indipendente da $A$.

\begin{remark}\label{oss_liberi_qf}
Sia $k:M\to N$ \`e un immersione parziale, $a$ una tupla di elementi di $\dom k$, e fissiamo due elementi $c\in M$ e $d\in N$ liberi rispettivamente su $a$ e $ka$. Allora $M,c,a\,\equiv_{\rm at}N,d,ka$. Infatti, se $c$ \`e libero su $a$ e $\phi(x,z)\in\aL$ \`e tale che $M\models\phi(c,a)$, allora 

\ceq{\hfill T_\Rmod\ \cup\ \Diag(M/a)}{\proves}{\phi(x,a).}

Poich\'e le strutture $\<a\>_M$ e  $\<ka\>_N$ sono tra loro isomorfe 

\ceq{\hfill T_\Rmod\ \cup\ \Diag(N/ka)}{\proves}{\phi(x,ka).}

e quindi $N\models\phi(d,ka)$. Lo stesso argomento si applica anche l'implicazione opposta e quindi $M,c,a\,\equiv_{\rm at}N,d,ka$.\QED 
\end{remark}

%%%%%%%%%%%%%%%%%%%%%%%%%%%%%%%%%%%
%%%%%%%%%%%%%%%%%%%%%%%%%%%%%%%%%%%
%%%%%%%%%%%%%%%%%%%%%%%%%%%%%%%%%%%
\section{Moduli su un dominio ad ideali principali}

Sia $M$ \`e un $R\jj$modulo, diremo che $M$ \`e \emph{senza torsione\/} se per ogni  $r\in R\sm \{0\}$ vale
\begin{itemize}
\item[st] $rx= 0\imp x=0$.
\end{itemize}

La seguente proposizione \`e essenziale per dimostrare l'equivalente dei lemmi~\ref{lem_ordinericco} e~\ref{lem_graforicco}. Nel capitolo~\ref{relazionali} non serviva esplicitare un simile risultato, l'assenza di funzioni lo rende quasi banale, cfr.\@  esercizio~\ref{ex_ol+grf_tipi_princ_comp}. Si osservi che in questi esercizi si ottiene che tutti i tipi senza quantificatori sono principali, nella seguente proposizione invece il tipo degli elementi liberi rimane escluso.

\begin{proposition}\label{prop_mst_tipi princ_comp}
Sia $R$ un dominio di integrit\`a e sia $T$ la teoria degli $R\jj$moduli senza torsione. Fissiamo $A\subseteq M\models T$ ed un elemento $c\in M$. Sia $p(x)=\pmatp_M(c/A)$ allora vale una delle seguenti due possibilit\`a:   
\begin{itemize}
\item[1.] $c$ \`e indipendente da $A$;
\item[2.] $M\models\phi(c)$ per una qualche $\phi(x)\in\aL(A)$ tale che 
$T\cup\Diag(M/A)\proves\phi(x)\imp p(x)$.
\end{itemize}\end{proposition}

L'insieme $A$ pu\`o essere preso infinito, questo permetter\`a di dimostrare il lemma~\ref{lemma_modulo_divisibile_ricco_nonnumer} senza restrizioni sulla cardinalit\`a e quindi di ottenere la categoricit\`a non numerabile, vedi corollario~\ref{thm_ModDivUltraOmog}. Non \`e cos\`i nell'esercizio~\ref{ex_ol+grf_tipi_princ_comp}. 

\begin{proof}
Se \ssf{1} non vale $p(x)$ contiene una formula atomica che non segue da $T_\Rmod\cup\Diag(M/A)$. Questa ha la forma $nx=t$ dove $t$ \`e un termine chiuso a parametri in $A$. Chiaramente $n\neq 0$. Vogliamo mostrare che $T\cup\Diag(M/A)\proves nx=t\imp p(x)$.

Dobbiamo verificare che per ogni intero $m$, ed ogni termine $s$ a parametri in $A$, uno dei seguenti casi vale per ogni $N\models T$ che contiene $\<A\>_M$
\begin{itemize}
\item[a.] $N\models\A x\,[nx=t\imp mx=s]$
\item[b.] $N\models\A x\,[nx=t\imp mx\neq s]$
\end{itemize} 
Supponiamo per assurdo che ci\`o non valga per certi $m$, $s$. Esistono quindi $N_i\models T$ sovrastrutture di $\<A\>_M$ tali che per qualche $a_i\in N_i$
\begin{itemize}
\item[a'.] $N_1\models na_1=t$ ed $N_1\models ma_1\neq s$
\item[b'.] $N_2\models na_2=t$ ed $N_1\models ma_2 = s$
\end{itemize} 
Da \ssf{b'} segue $N_2\models mt=ns$ e, poich\'e $N_1$ non ha torsione, da \ssf{a'} segue $N_1\models mt\neq ns$. Ma $mt=ns$ \`e una formula a parametri in $A$ quindi la sua verit\`a dipende solo da $\<A\>_M$. Questo contraddice $\<A\>_M\subseteq N_i$.
\end{proof}


\begin{exercise}\label{ex_ol+grf_tipi_princ_comp}
Scriviamo $T$ per una qualsiasi delle teorie $T_{\rm ol}$ e $T_{\rm gfr}$. Sia $A\subseteq M\models T$ finito, e sia $c\in M$. Dimostrare che $p(x)=\pmatp_M(c/A)$ \`e un tipo principale. Ovvero, esiste $\phi(x)$ congiunzione di formule in $p(x)$ tale che

\ceq{\hfill T\cup\Diag(M/A)}{\proves}{\phi(x)\imp p(x)}.

Si osservi che questo non vale se $A$ \`e infinito.\QED
\end{exercise}




%%%%%%%%%%%%%%%%%%%%%%%%%%%%%%%%%%%
%%%%%%%%%%%%%%%%%%%%%%%%%%%%%%%%%%%
%%%%%%%%%%%%%%%%%%%%%%%%%%%%%%%%%%%
\section{Moduli divisibili}

In questo paragrafo $R$ \`e sempre un dominio di integrit\`a. Diremo che $M$ \`e \emph{divisibile\/} se \`e senza torsione e per ogni  $r\in R\sm \{0\}$ vale
\begin{itemize}
\item[div] $\E x\ rx=y$.
\end{itemize} 

La seguente proposizione traduce l'assioma di divisibilit\`a in una in propriet\`a delle formule atomiche.

\begin{proposition}\label{prop_md_cons_sodd}
Sia $R$ un dominio di integrit\`a, $T$ la teoria degli $R\jj$moduli senza torsione. e $N\models T$. Allora le seguenti affermazioni sono equivalenti:
\begin{itemize}
\item[1.] $N$ \`e divisibile;
\item[2.] $N\models\E x\,\phi(x)$ per ogni formula consistente $\phi(x)\in\aL(N)$.
\end{itemize} 
Sopra, $x$ denota una singola variabile e la consistenza va intesa modulo $T\cup\Diag(N/A)$, dove $A$ \`e l'insieme dei parametri che occorrono in $\phi(x)$. Quindi $\phi(x)$ \`e consistente se soddisfatta in un qualsiasi $R\jj$modulo senza torsione che contiene $\<A\>_N$.
\end{proposition}

Dal lemma~\ref{lemma_modulo_divisibile_ricco_nonnumer} che dimostreremo qui sotto segue che \ssf{2}$\IMP$\ssf{1} vale per ogni $\phi(x)\in L_{\rm qf}(A)$ anche quando $x$ \`e una tupla di variabili (vedi esercizio~\ref{lkhfq}), ma per arrivarci dobbiamo prima verificarla per le formule atomiche. 

\begin{proof}
Dimostriamo \ssf{2}$\IMP$\ssf{1}. Mostriamo che per ogni elemento $a\in N$ e per ogni $r\in R\sm \{0\}$ la formula $rx=a$ \`e consistente. Se consideriamo $R$ come $R\jj$modulo otteniamo un modulo senza torsione: l'assenza di torsione si riduce all'assenza di divisori dello zero.  Posto $k=\{\<s,sa\> : s\in R\}$, \`e immediato verificare che $k:R\imp\<a\>_N$ \`e un isomorfismo tra $R\jj$moduli (l'assenza di torsione serve per iniettivit\`a). L'equazione $rx=1$ ha soluzione in un estensione di $R$, precisamente, nel campo delle frazioni di $R$. Quindi, identificando strutture isomorfe, $rx=a$ ha soluzione in un estensione di $\<a\>_N$. Per \ssf{2} ha soluzione in $N$.

Dimostriamo \ssf{1}$\IMP$\ssf{2}. Per il corollario~\ref{corol_formacanonicaterminimoduli} possiamo assumere che la formula $\phi(x)$ abbia la forma $r x= a$ per un qualche $a\in N$. Se $r=0$, poich\'e $\phi(x)$ \`e consistente, dev'essere $a=0$ e la consistenza \`e ovvia. Altrimenti una soluzione in $N$ esiste per l'assioma di divisibilit\`a.
%Dimostriamo \ssf{1}$\IMP$\ssf{2}. Per la proposizione~\ref{formacanonicaterminimoduli} possiamo assumere che la formula $\phi(x)$ abbia la forma $r\cdot x= a$ per un qualche elemento $a\in M$. Se tutte le componenti della tupla $r$ sono nulle allora, poich\`e $\phi(x)$ \`e consistente, $a=0$. In questo caso la consistenza in $M$ \`e ovvia. Per facilit\`a supponiamo $r_0\neq 0$. Se $b$ \`e una soluzione dell'equazione $r_0x_0=a$, allora la tupla con prima componente $b$ e tutte le altre coordinate nulle, \`e una soluzione di $\phi(x)$. Per la divisibilit\`a, un tale $b$ esiste in $M$.
\end{proof}

Chiameremo \emph{rango\/} di $M$ \`e la minima cardinalit\`a di un insieme $A\subseteq M$ tale che $\<A\>_M=M$. (Quindi, se $A$ ha cardinalit\`a minore del rango, $M$ contiene qualche elemento indipendente da $A$.)

Il seguente \`e teorema corrisponde ai lemmi~\ref{lem_ordinericco} e~\ref{lem_graforicco} con due differenze molto importanti. La prima \`e una limitazione: serve assumere che $N$ ha rango sufficientemente grande, un'ipotesi che non ha corrispondente per le teorie $T_{\rm oldse}$ e $T_{rg}$ e che ha profonde conseguenze perch\'e non \`e traducibile in un assioma del prim'ordine. (La situazione \`e simile a quella della proposizione~\ref{prop_ex_R_omega}.) Quindi dal teorema non segue nessun risultato di $\omega\jj$categoricit\`a. La seconda differenza \`e che qui il teorema \`e esteso ad ogni cardinalit\`a $\lambda$. Poich\'e l'ipotesi sul rango svanisce (viene automaticamente soddisfatta) per $\lambda$ sufficientemente grande, otterremo un risultato di $\lambda\jj$categoricit\`a.


\begin{lemma}\label{lemma_modulo_divisibile_ricco_nonnumer}
Sia $R$ un dominio di integrit\`a e sia $T$ la teoria degli $R\jj$moduli senza torsione.
Allora ogni modulo divisibile e di rango $\lambda$ \`e un modello $\lambda\jj$ricco di $T$.
\end{lemma}

\begin{proof}
Sia $\<a_i:i<\lambda\>$ un'enumerazione di $M$. Seguiamo la stessa costruzione della dimostrazione del lemma~\ref{lem_ordinericco}, con la stessa notazione. L'unica differenza \`e che nel caso  in cui $|M|$ non \`e numerabile dovremo considerare passi limite: se $i<\lambda$ \`e limite, allora definiamo

\hfil$\displaystyle h_i\ \ :=\ \ \bigcup_{j<i} h_j$.

Ora definiremo $h_{i+1}$ in modo che $a_i\in\dom h_{i+1}$. Sia $a$ una tupla che enumera $\dom h_i$ e sia $p(x,z)=\pmatp_M(a_i,a)$. (Si osservi che $a$ potrebbe essere una tupla infinita.) Se $a_i$ non \`e indipendente da $a$, dalla proposizione~\ref{prop_mst_tipi princ_comp} otteniamo che $M\models\phi(a_i,a)$ per una formula atomica tale che 

\ceq{\ssf{i.}\hfill T\ \cup\ \Diag(M/a)}{\proves}{\phi(x,a)\imp p(x,a).} 

Inoltre ovviamente:

\ceq{\ssf{ii.}\hfill T\ \cup\ \Diag(M/a)}{\cup}{\big\{\phi(x,a)\big\}}\hspace{3ex}\`e consistente.

Per l'ipotesi induttiva $\<a\>_M$ \`e isomorfo a $\<h_ia\>_N$ e quindi \ssf{i} e \ssf{ii} valgono anche sostituendo $M,a$ con $N,h_ia$. Per la proposizione~\ref{prop_md_cons_sodd} la formula $\phi(x,h_ia)$ \`e soddisfatta in $N$. Un qualsiasi $c$ che soddisfa $\phi(x,h_ia)$ realizza $p(x,h_ia)$. Quindi posto $h_{i+1}=h_i\cup\{\<a_i,c\>\}$ otteniamo l'estensione richiesta. Veniamo al caso in cui $a_i$ \`e indipendente da $a$. Poich\'e $|a|$ \`e minore del rango di $N$, esiste un $c\in N$ e indipendente da $h_ia$. Come visto nell'osservazione~\ref{oss_liberi_qf} segue che $M,a,a_i\equiv_{\rm at} N,h_ia,c$. Quindi l'estensione richiesta si ottiene definendo $h_{i+1}=h_i\cup\{\<a_i,c\>\}$.
\end{proof}

\begin{corollary}\label{thm_ModDivUltraOmog}
Sotto le stesse ipotesi del lemma~\ref{lemma_modulo_divisibile_ricco_nonnumer}. Fissiamo $M,N\models T$ entrambi di cardinalit\`a $\lambda$ e di rango $\lambda$.  Allora ogni $k:M\imp N$, immersione parziale di cardinalit\`a $<\lambda$, si estende ad un isomorfismo.
\end{corollary}

\begin{proof}
Segue dal lemma~\ref{lemma_modulo_divisibile_ricco_nonnumer} e dal teorema~\ref{riccozigzag}.
\end{proof}

Si osservi che la cardinalit\`a di $\<A\>_N$ \`e al pi\`u $|A|+|R|+\omega$, quindi un $R\jj$modulo $N$ di cardinalit\`a $>|R|+\omega$ ha necessariamente rango $|N|$. Quindi nel lemma~\ref{lemma_modulo_divisibile_ricco_nonnumer} possiamo omettere l'ipotesi sul rango se richiediamo che la cardinalit\`a di $N$ sia sufficientemente grande. Otteniamo quindi il seguente:

\begin{corollary}\label{thm_ModDivUltraOmog}
Sia $R$ un dominio di integrit\`a e sia $T$ la teoria degli $R\jj$moduli senza torsione. Allora $T$ \`e $\lambda\jj$categorica per ogni $\lambda>|R|+\omega$.\QED
\end{corollary}



Per i teoremi~\ref{imparzialemappeelementari} e~\ref{tutteleimmersionisonoelementari=eliminazioneqantificatori}, categoricit\`a e ultraomogeneit\`a hanno la seguente importante conseguenza:

\begin{corollary}\label{corol_ModDivElQuan}
Sia $R$ un dominio di integrit\`a. La teoria degli $R\jj$moduli divisibili ha eliminazione dei quantificatori.\QED
\end{corollary}

\begin{exercise}\label{lkhfq}
Sia $R$ un dominio di integrit\`a e sia $N$ un $R\jj$modulo senza torsione. Allora le seguenti affermazioni sono equivalenti:
\begin{itemize}
\item[1.] $N$ divisibile;
\item[2.] ogni formula $\phi(x)\in L_{\rm qf}(A)$ consistente \`e soddisfatta in $N$.
\end{itemize} 
La consistenza \`e da intendersi modulo la teoria $T\cup\Diag(N/A)$ dove $A\subseteq N$ contiene i parametri della formula. (Suggerimento: l'affermazione sarebbe immediata se potessimo eliminare dal lemma~\ref{lemma_modulo_divisibile_ricco_nonnumer} la richiesta che $N$ ha rango infinito. Per ovviare, basta usare L\"owenheim-Skolem all'ins\`u.)
\end{exercise}

\begin{exercise}
Sia $R$ un dominio di integrit\`a. Si dimostri che tutti gli $R\jj$moduli divisibili sono omogenei. N.B.\@ indipendentemente dalla cardinalit\`a e dal rango. Suggerimento: si proceda per induzione sulla cardinalit\`a dell'immersione parziale da estendere.\QED
\end{exercise}

%%%%%%%%%%%%%%%%%%%%%%%%%%%%%%%%%%
%%%%%%%%%%%%%%%%%%%%%%%%%%%%%%%%%%
%%%%%%%%%%%%%%%%%%%%%%%%%%%%%%%%%%
%%%%%%%%%%%%%%%%%%%%%%%%%%%%%%%%%%
%%%%%%%%%%%%%%%%%%%%%%%%%%%%%%%%%%
\section{Anelli e domini d'integrit\`a (un ripasso)}
\label{anelli}
Riportiamo qui sotto gli assiomi degli anelli. Tutti gli anelli che considereremo saranno unitari. Il linguaggio \`e quello degli anelli unitari. 

\begin{itemize}
\item[a1-a4] come nei gruppi abeliani
\item[a5] $(x\mdot y)\mdot z\  =\ y\mdot(x\mdot z)$,
\item[a6] $1\mdot x\ =\ x\mdot 1\ =\ x$,
\item[a7] $(x+y)\mdot z\ =\ x\mdot z + y\mdot z$,
\item[a8] $z\mdot (x+y)\ =\ z\mdot x + z\mdot y$.
\end{itemize}

Un anello \`e detto \emph{commutativo\/} se vale

\begin{itemize}
\item[ac] $x\mdot y\ =\ y\mdot x$.
\end{itemize}
 
Indicheremo con \emph{$T_{\rm auc}$} la teoria degli anelli (unitari) commutativi.

Se $M$ \`e un anello e $a\in M$, diremo che $a$ \`e nilpotente se $a^n=0$ per qualche intero positivo $a$, dove 

\ceq{\hfill x^n}{\deq}{\smash{\stackrel{n\ \rm volte}{\overbrace{x\cdot x\cdot\dots\cdot x}}}}

La teoria degli \emph{anelli senza elementi nilpotenti}, che qui denoteremo con \emph{$T_{\rm asn}$} contiene, oltre agli assiomi di $T_{\rm auc}$, i seguenti

\begin{itemize}
\item[sn] $x^n=0\imp x=0$.
\end{itemize}

per ogni intero positivo $n$.

La \emph{caratteristica\/} di un anello $M$ \`e il minimo $n$, se esiste, per cui $n1=0$. Se questo $n$ non esiste, allora diremo che l'anello ha caratteristica $0$. Quindi \`e chiaro che due anelli $M$ ed $N$  hanno la stessa caratteristica se e solo se $M\equiv_{\rm at}N$ ovvero se $\0:M\imp N$ \`e un'immersione parziale.

La seguente proposizione mostra che dato un anello $M$ ed un insieme $A\subseteq M$ possiamo identificare i termini a parametri in $A$ con polinomi a coefficienti in $\<A\>_M$.

\begin{proposition}
\label{formacanonicaterminiau}
Sia $x=\<x_i: i<\alpha\>$. Dato $n\in\omega^{\oplus\alpha}$, diciamo $n=\<n_i: i<\alpha\>$, useremo la seguente abbreviazione:

\ceq{\hfill x^n}{=}{\prod_{i<\alpha} x^{n_i}_i.}

Fissati $A\subseteq M$ come sopra, per ogni $t(x)$, termine a parametri in $A$, esiste una funzione $c:\omega^{\oplus\alpha}\to\<A\>_M$, quasi ovunque nulla, tale che

\ceq{\ssf{p}\hfill T_{\rm auc}+\Diag(M/A)}{\proves}{\displaystyle t(x)\ \ =\ \ \sum_{n\in \omega^{\oplus\alpha}}c(n)\,x^n}

Inoltre, tale funzione $c$ \`e unica.% La proposizione vale anche nel caso degenere dei termini chiusi: per ogni termine chiuso $t$, esiste un $n$ tale che in tutti gli anelli commutativi vale $t=n$.   
\end{proposition}

\begin{proof}
L'esistenza si dimostra per induzione sulla sintassi di $t(x)$. L'unicit\`a \`e immediata. %Per dimostrare l'unicit\`a, supponiamo che $c$ e $d$ rendano valida l'identit\`a \ssf{p} e quindi che il termine
% 
% \hfil$\displaystyle \sum_{ n\in \omega^{\oplus\alpha}}\big(c(n)-d(n)\big)\,x^n$
% 
% \`e identicamente nullo in tutti gli anelli commutativi che contengono $\<A\>_M$.
\end{proof}

% \begin{proof}
% L'esistenza si dimostra facilmente per induzione sulla sintassi dei termini. Per dimostrare l'unicit\`a, supponiamo che $c$ e $d$ rendano valida l'identit\`a \ssf{p}. Segue immediatamente che il termine
% 
% \hfil$\displaystyle \sum_{ n\in \omega^k}\big(c(n)-d(n)\big)\,x^n$
% 
% \`e identicamente nullo in tutti gli anelli commutativi, quindi anche in $\ZZ$, e quindi $c(n)=d(n)$ per ogni $n$.
% \end{proof}
Modulo $T_{\rm auc}+\Diag(M/A)$, ogni formula atomica a parametri in $A$ \`e equivalente ad una formula della forma $t(x)=0$ dove $t(x)$ \`e un termine. Se $\phi(x)$ \`e una formula di questa forma chiameremo \emph{grado\/} di $\phi(x)$, il grado del polinomio associato a $t(x)$, ovvero

\hfil$\displaystyle\max\Bigg\{\smash{\sum_{i<\alpha}n_i}\ :\ n\in\omega^{\oplus\alpha},\ \ c(n)\neq0\Bigg\}$.

Dove $c:\omega^{\oplus\alpha}\to\<A\>_M$ \`e la funzione data dalla proposizione~\ref{formacanonicaterminiau}. 

Infine, se $A\subseteq M\models T_{\rm auc}$ e $c\in M$, diremo che $c$ \`e \emph{trascendente\/} su $A$ se $p(x)=\atp(c/A)$ \`e un tipo banale, ovvero se $T_{\rm auc}\cup\Diag(M/A)\proves p(x)$. Altrimenti diremo che $c$ \`e \emph{algebrico\/} su $A$. 



\begin{remark}\label{oss_trascendenti_qf}
L'osservazione~\ref{oss_liberi_qf} pu\`o essere ripetuta letteralmente qui, sostituendo $T_{\rm di}$ a $T_\Rmod$ e trascendente a libero.\QED 
\end{remark}
% 
% \begin{proposition}
% \label{formacanonicaterminiau}
% Sia $x$ una variabile. Per ogni termine puro $t(x)$ esiste una funzione $c:\omega\to\ZZ$, quasi ovunque nulla, tale che
% 
% \ssf{p}\hfil$T_{\rm auc}\ \proves\ \displaystyle t(x)\ \ =\ \ \sum_{n\in \omega^k}c(n)\,x^n$
% 
% Inoltre, tale funzione $c$ \`e unica. 
% \end{proposition}
% 
% La proposizione pu\`o essere estesa facilmente al caso in cui $x$ \`e una tupla (nel seguito useremo solo il caso particolare enunciato). 
% 
% \begin{proof}
% L'esistenza si dimostra facilmente per induzione sulla sintassi dei termini. Per dimostrare l'unicit\`a, supponiamo che $c$ e $d$ rendano valida l'identit\`a \ssf{p}. Segue immediatamente che il termine
% 
% \hfil$\displaystyle \sum_{ n\in \omega}\big(c(n)-d(n)\big)\,x^n$
% 
% \`e identicamente nullo in tutti gli anelli commutativi, quindi anche in $\ZZ$, e quindi $c(n)=d(n)$ per ogni $n$.
% \end{proof}

\section{Domini di integrit\`a}

Due elementi diversi da $0$ tali che $a\,b=0$ si chiamano \emph{divisori dello zero}. Un anello commutativo \`e detto un \emph{dominio di integrit\`a} se  
\begin{itemize}
\item[nb] $0\neq 1$
\item[di] $x\mdot y =0\ \imp\ x=0\ \vee\ y=0$.
\end{itemize}
La teoria dei domini di integrit\`a qui verr\`a denotata con \emph{$T_{\rm di}$}. 

% 
% La seguente proposizione \`e immediata.
% 
% \begin{proposition}[(identit\`a di B\'ezout)]
% Sia $R$ un dominio ad ideali principali, consideriamo $R$ come $R\jj$modulo. Allora per ogni $A\subseteq R$ esiste un termine $t$ a parametri in $A$ tale che $\<A\>_R=\<t\>_R$.\QED
% \end{proposition}
% 
% Vediamo in che senso la proposizione generalizza l'\emph{identit\`a di B\'ezout}. Poich\'e $\<a\>_R\subseteq\<c\>_R$ se e solo se $a$ divide $b$, il generatore di $\<a,b\>_R$ diventa $\gcd(a,b)$. Quindi la proposizione afferma che $\gcd(a,b)=t(a,b)$ per un qualche termine $t(x,y)$ ovvero, per il lemma~\ref{formacanonicaterminimoduli}, una combinazione lineare di $a$ e $b$.

\begin{proposition}\label{prop_di_tipi princ_comp}
Fissiamo $A\subseteq M\models T_{\rm di}$ e  $c\in M$. Sia $p(x)=\pmatp_M(c/A)$ allora vale una delle seguenti due possibilit\`a:   
\begin{itemize}
\item[1.] $c$ \`e trascendente su $A$;
\item[2.] $M\models\phi(c)$ per una qualche $\phi(x)\in\aL(A)$ tale che 
$T\cup\Diag(M/A)\proves\phi(x)\imp p(x)$.
\end{itemize}
\end{proposition}

L'insieme $A$ pu\`o essere preso infinito. Questo permetter\`a di dimostrare il teorema~\ref{thm_acf_ricco_nonnumer} senza restrizioni sulla cardinalit\`a e quindi di ottenere la categoricit\`a non numerabile, vedi corollario~\ref{thm_acfUltraOmog}.

\begin{proof}
Se \ssf{1} non vale esiste $\psi(x)\in\aL(A)$ non banale tale che $\psi(c)$. Scegliamo la formula di grado minimo. Sia $\xi(x)\in\aL(A)$ arbitraria. \`E sufficiente mostrare che vale una delle seguenti possibilit\`a:
\begin{itemize}
\item[1.] $\proves\ \psi(x)\imp\phantom{\neg}\xi(x)$;
\item[2.] $\proves\ \psi(x)\imp\neg\xi(x)$,
\end{itemize} 
dove abbiamo scritto $\proves$ per conseguenza logica modulo $T_{\rm di}\cup\Diag(M/A)$. Supponiamo che la formula $\psi(x)$ abbia la forma $a(x)=0$ per un polinomio $a(x)$ di grado $n$. Scriviamo $a'(x)=0$ per la formula $\xi(x)$.  Assumiamo temporaneamente che $\<A\>_M$ sia un un campo; allora esiste un polinomio $d(x)$ tale che per opportuni polinomi $b(x)$ e $b'(x)$
\begin{itemize}
\item[a.] $d(x)b(x)=a(x)$;
\item[a'.] $d(x)b'(x)=a'(x)$.
\end{itemize}   
Inoltre assumiamo che $d(x)$ abbia grado massimo tra i polinomi con questa propriet\`a. Se invece $\<A\>_M$ non \`e un campo, quanto detto resta comunque valido nel campo delle frazioni di $\<A\>_M$, e quindi le equazioni \ssf{a} e \ssf{a'} valgono a meno di un fattore in $\<A\>_M$ che possiamo, senza perdita di generalit\`a, assorbire in $a(x)$ e $a'(x)$. 

Da \ssf{a} otteniamo che $d(c)=0$ o $b(c)=0$. Nel primo caso, poich\'e $d(x)$ non pu\`o avere grado inferiore ad $a(x)$, otteniamo che $b(x)$ \`e un polinomio costante non nullo.  Questo implica immediatamente \ssf{1}. Quindi supponiamo $b(c)=0$, e questo implica che $d(x)=d$, una costante non nulla. Ancora una volta, assumiamo temporaneamente che $\<A\>_M$ sia un un campo in modo da poter applicare l'identit\`a di B\'ezout e ottenere opportuni polinomi $c(x)$ e $c'(x)$ tali che $d=a(x)c(x)+a'(x)c'(x)$. Ora \ssf{2} \`e immediata conseguenza di questa identit\`a. 

Se invece $\<A\>_M$ non \`e un campo, ragioniamo come sopra: dall'identit\`a di B\'ezout nel campo delle frazioni di $\<A\>_M$ otteniamo $d'd=a(x)c(x)+a'(x)c'(x)$ dove $d'\in\<A\>_M$ \`e non nullo. Quindi arriviamo  alla stessa conclusione.
\end{proof}



%%%%%%%%%%%%%%%%%%%%%%%%%%%%%%%%%%%%%%
%%%%%%%%%%%%%%%%%%%%%%%%%%%%%%%%%%%%%%
%%%%%%%%%%%%%%%%%%%%%%%%%%%%%%%%%%%%%%
%%%%%%%%%%%%%%%%%%%%%%%%%%%%%%%%%%%%%%
%%%%%%%%%%%%%%%%%%%%%%%%%%%%%%%%%%%%%%
%%%%%%%%%%%%%%%%%%%%%%%%%%%%%%%%%%%%%%
\section{Campi algebricamente chiusi}

Sia $a$ un elemento di un anello commutativo. Un elemento $b$ tale che $a\mdot b=b\mdot a=1$ \`e detto \emph{inverso\/} di $a$.  Un anello commutativo \`e detto un \emph{campo\/} se ogni elemento non nullo ha un inverso. In altre parole se vale il seguente assioma:
%
\begin{itemize}
\item[c]$\E  y\; \big[x\neq0\ \imp\ x\mdot y =1\big]$.
\end{itemize}
%
Ricordiamo che, per convenzione, i campi sono strutture con la segnatura degli anelli unitari, il linguaggio non contiene nessun simbolo di funzione per $x^{-1}$ e quindi le sottostrutture di un campo sono semplicemente degli anelli, o meglio, dei domini di integrit\`a.


% \begin{exercise}
% Si osservi che in un campo la formula $x\neq 0$ \`e equivalente ad una formula positiva e si deduca che tutti gli omomorfismi (totali) da un campo ad un anello commutativo sono immersioni. (Nei testi di algebra il risultato si dimostra verificando che i campi hanno solo ideali banali.)\QED
% \end{exercise}

Un \emph{campo algebricamente chiuso\/}  (in inglese, \emph{algebraically closed field}) \`e un campo che soddisfa il seguente schema di assiomi, per ogni intero $n$ positivo:

\begin{itemize}
\item[ac] $\E x\ \big(x^{n} + z_{n-1}x^{n-1} + \dots + z_1 x + z_0\ \ =\ \ 0\big)$
\end{itemize}

dove $x$ e $z_0,\dots,z_n$ denotano singole variabili. La teoria dei campi algebricamente chiusi verr\`a denotata con \emph{$T_{\rm acf}$}.


\begin{proposition}\label{prop_acf_cons_sodd}
Fissiamo $N\models T_{\rm di}$ ed una variabile $x$. Allora le seguenti affermazioni sono equivalenti:
\begin{itemize}
\item[1.] $N$ \`e un campo algebricamente chiuso;
\item[2.] $N\models\E x\,\phi(x)$ per ogni formula consistente $\phi(x)\in\aL(N)$.
\end{itemize} 
Sopra, $x$ denota una singola variabile e la consistenza va intesa modulo $T_{\rm di}\cup\Diag(N/A)$, dove $A$ \`e l'insieme dei parametri che occorrono in $\phi(x)$. Quindi $\phi(x)$ \`e consistente se soddisfatta in un qualsiasi dominio di integrit\`a che contiene $\<A\>_N$.
\end{proposition}

\begin{proof}
Dimostriamo \ssf{1}$\IMP$\ssf{2}. A meno di equivalenza la formula $\phi(x)$ ha la forma $a_nx^n + \dots + a_1 x + a_0=0$ per $a_i\in\<A\>_N$. Sia $n$ il minimo per cui questo \`e possibile. Se $n=0$ allora, la consistenza di $\phi(x)$ impone $a_0=0$ e l'affermazione \`e ovvia. Altrimenti possiamo assumere $a_n\neq0$ e l'affermazione deriva immediatamente da l'assioma \ssf{ac}. 

Per l'implicazione \ssf{2}$\IMP$\ssf{1} rimandiamo ad un qualsiasi testo di base di algebra per la dimostrazione che ogni equazione della forma $x^n + a_{n-1}x^{n-1} \dots + a_1 x + a_0=0$ per $a_i\in N$ e $n>0$ ha soluzioni in un campo che estende $\<a_0,\dots,a_n\>_N$.
\end{proof}

Il \emph{grado di trascendenza\/} di un dominio di integrit\`a $M$ \`e la minima cardinalit\`a di un insieme $A\subseteq M$ tale che tutti gli elementi di $M$ sono algebrici su $A$. Poich\'e ogni formula algebrica ha un numero finito di soluzioni, se $\omega<|M|$ il grado di trascendenza di $M$ \`e $|M|$. 


\begin{theorem}\label{thm_acf_ricco_nonnumer}
Ogni campo algebricamente chiuso di grado di trascendenza  $\lambda$ \`e un modello $\lambda\jj$ricco di $T_{\rm di}$. Di conseguenza tutti i campi  algebricamente chiusi non numerabili sono ricchi.
\end{theorem}


\begin{proof}
Letteralmente la stessa dimostrazione del teorema~\ref{lemma_modulo_divisibile_ricco_nonnumer} sostituendo $T$ con $T_{\rm di}$ e \textit{libero\/} con \textit{trascendente\/}. I riferimenti alle proposizioni~\ref{prop_mst_tipi princ_comp} e~\ref{prop_md_cons_sodd} vanno sostituiti con riferimenti alle corrispondenti proposizioni ~\ref{prop_di_tipi princ_comp} e~\ref{prop_acf_cons_sodd}
\end{proof}

Scriviamo $T^p_{\rm acf}$ per la teoria dei campi algebricamente chiusi di caratteristica $p$. 


\begin{corollary}\label{thm_acfUltraOmog}
Siano $M,N\models T_{\rm acf}$ modelli della stessa di cardinalit\`a $\lambda>\omega$. Allora ogni $k:M\imp N$ immersione parziale di cardinalit\`a $<\lambda$ si estende ad un isomorfismo. Di conseguenza $T^p_{\rm acf}$ \`e $\lambda\jj$categorica per ogni $\lambda>\omega$ e per ogni caratteristica $p$.
\end{corollary}

Pi\`u avanti estenderemo \ssf{2} anche ai modelli numerabili.

\begin{proof}
Un campo algebricamente chiuso non numerabile ha necessariamente grado di trascendenza infinito. Si osservi solo che per ottenere la $\lambda\jj$categoricit\`a \`e necessario fissare la caratteristica (la mappa vuota tra campi di diversa caratteristica non \`e un'immersione parziale).\end{proof}

Quindi dai teoremi~\ref{imparzialemappeelementari} e~\ref{tutteleimmersionisonoelementari=eliminazioneqantificatori} otteniamo:

\begin{corollary}\label{corol_acfElQuan}
Per ogni caratteristica $p$, la teoria $T^p_{\rm acf}$ \`e completa ed ha eliminazione dei quantificatori.\QED
\end{corollary}


\begin{exercise}
Si dimostri che tutti campi algebricamente chiusi sono omogenei. N.B.\@ indipendentemente dalla cardinalit\`a e dal rango. Suggerimento: si proceda per induzione sulla cardinalit\`a dell'immersione parziale da estendere.\QED
\end{exercise}

\section{Il Nullstellensatz}

\def\ceq#1#2#3{\parbox{20ex}{$\displaystyle #1$}\parbox{6ex}{\hfil$\displaystyle #2$}$\displaystyle  #3$}

Nel seguito $x$ \`e una tupla di variabili fissata ed $A$ un sottoinsieme di un anello $N$. Indicheremo con $\Delta$ l'insieme delle formule della forma $t(x)=0$ dove $t(x)$ \`e un termine a parametri in $A$. Converr\`a identificare le formule della forma $t(x)=0$ con il polinomio a coefficienti in $\<A\>_M$ che corrisponde al termine $t(x)$ dato dalla proposizione~\ref{formacanonicaterminiau}. Quindi all'occasione tratteremo $\Delta$ come un anello di polinomi. Quindi $t(x)\in\Delta$ e $t(x)=0\in\Delta$ saranno espressioni sinonime. 

A queste due interpretazioni di $\Delta$ corrispondono due operazioni di chiusura sugli insiemi $p\subseteq\Delta$. Da un lato esiste la  chiusura di $p$ per conseguenze logiche dall'altro l'ideale dell'anello generato da $p$. Queste non sono esattamente la stessa operazione e la seguente proposizione le confronta. 

\begin{proposition}\label{prop_chiusura-radicale}
Fissiamo un dominio di integrit\`a $N$, un sottoinsieme $A\subseteq N$ e sia $\Delta$ come sopra. Per ogni $p\subseteq\Delta$ le seguenti affermazioni sono equivalenti:
\begin{itemize}
\item[1.] $p$ \`e un ideale radicale, ovvero, per ogni $n$ intero positivo $t^n(x)\in p\IMP t(x)\in p$;
\item[2.] $p$ \`e un $\Delta\jj$tipo chiuso per conseguenze modulo $T_{\rm di}\cup\Diag(N/A)$.
%\item[3.] $p$ \`e chiuso per conseguenze modulo $T_{\rm asn}\cup\Diag(N/A)$.
\end{itemize}
\end{proposition}
Si osservi che avremmo potuto usare $T_{\rm asn}$ al posto di $T_{\rm di}$ con la stessa dimostrazione.
\begin{proof}
Nel simbolo $\proves$ sottointendiamo $T_{\rm di}\cup\Diag(N/A)$.  Per dimostrare \ssf{1}$\IMP$\ssf{2} fissiamo un polinomio $t(x)\notin p$ e mostriamo che $p\not\proves t(x)=0$. Da \ssf{1} sappiamo che $p$ \`e intersezione di tutti gli ideali primi che lo contengono. Quindi $t(x)\notin q$ per qualche ideale primo $q$ che contiene $p$.  Per la primalit\`a dell'ideale $q$, l'anello $\Delta/q$ \`e un dominio di integrit\`a.  Denotiamo con $x+q$ la classe in $\Delta/q$ di cui $x$ \`e rappresentante. Come \`e noto $x+q$ \`e una soluzione di tutte e sole le equazioni in $q$. Questo mostra che $q\not\proves t(x)=0$. A fortiori $p\not\proves t(x)=0$. Per l'implicazione \ssf{2}$\IMP$\ssf{1} osserviamo che $p$ \`e sicuramente un ideale e che da $p\proves t^n(x)=0$ segue ovviamente $p\proves t(x)=0$. 
\end{proof}


Dato $p\subseteq\Delta$ scriviamo \emph{${\rm\bf rad}\,p$} per l'ideale radicale generato da $p$, ovvero l'intersezione di tutti gli ideali radicali che contengono $p$. 

\begin{corollary}\label{corol_caratterizzazione_rad}
Sotto le stesse ipotesi della proposizione~\ref{prop_chiusura-radicale}, se $p\subseteq\Delta$ allora

\ceq{\hfill\rad p}{=}{\Big\{t(x)\ :\ p\proves t(x)=0\Big\}}

dove $\proves$ sta per la conseguenza logica modulo $T_{\rm di}\cup\Diag(N/A)$.\QED
\end{corollary}


\begin{proposition}\label{prop_tipi_e_ideali_primi}
Sotto le stesse ipotesi della proposizione~\ref{prop_chiusura-radicale}, se $p\subseteq\Delta$ \`e un ideale radicale allora le seguenti affermazioni sono equivalenti:
\begin{itemize}
\item[1.] $p$ \`e un $\Delta\jj$tipo primo;
\item[2.] $p$ \`e un ideale primo.
\end{itemize}
\end{proposition}

\begin{proof}
Nel simbolo $\proves$ sottointendiamo $T_{\rm di}\cup\Diag(N/A)$.  Dimostriamo \ssf{1}$\IMP$\ssf{2}. Assumiamo \ssf{1} e supponiamo che $t(x)\cdot s(x)$ appartenga $p$. Quindi $p\proves t(x)=0\vee s(x)=0$. Per primalit\`a di $p$ come $\Delta\jj$tipo,  $p\proves t(x)=0$ oppure $p\proves s(x)=0$. Per la proposizione~\ref{prop_chiusura-radicale}, $p$ \`e chiuso per conseguenze, quindi $t(x)\in p$ o $s(x)\in p$.

Dimostriamo \ssf{2}$\IMP$\ssf{1}. Assumiamo \ssf{2} e supponiamo che

\ceq{\hfill p}{\proves}{\bigvee^n_{i=1} t_i(x)=0.}

Quindi 

\ceq{\hfill p}{\proves}{\prod^n_{i=1} t_i(x)\ =\ 0}

Per la chiusura di $p$ e la primalit\`a di $p$ come ideale, otteniamo che $p\proves t_i(x)=0$ per qualche $i$. Per il corollario~\ref{coroll_test_primalita} questo \`e sufficiente per stabilire che $p$ \`e un tipo primo.
\end{proof}



\begin{proposition}
Sotto le stesse ipotesi della proposizione~\ref{prop_chiusura-radicale} assumiamo che $N$ sia un campo algebricamente chiuso, $|A|<|N|$ e $|x|\le |N|$. Allora per ogni $p\subseteq\Delta$:

\ceq{\hfill\rad p}{=}{\Big\{t(x)\ :\ N\models\A x\,[ p(x)\imp t(x)=0]\Big\}}


\end{proposition}

\begin{proof}
L'inclusione $\subseteq$ segue immediatamente dalla definizione di $\rad p$. Per dimostrare l'inclusione $\supseteq$ fissiamo $t(x)\notin\rad p$ e mostriamo che $N\models\E x\,[ p(x)\wedge t(x)\neq0]$. Poich\'e  $p\notproves t(x)=0$, in un qualche dominio di integrit\`a $M$ che contiene $\<A\>_N$ esiste una soluzione di $p(x)\wedge t(x)\neq0$. Senza perdere generalit\`a possiamo assumere che $|M|\le|N|$. Quindi esiste un'immersione $h:M\to N$ che estende $\id_A$. Questa immersione mappa detta soluzione in una soluzione in $N$.
\end{proof}

Quando la tupla $x$ \`e finita possiamo omettere dalle ipotesi $|A|<|N|$ utilizzando il teorema della base di Hilbert (per il quale rimandiamo alla letteratura) che dice che $p$ \`e un $\Delta\jj$tipo principale. Otteniamo quindi il seguente corollario che (a parte la notazione che qui abbiamo adattato al contesto) \`e una delle forme in cui solitamente viene enunciato il \emph{Nullstellensatz di Hilbert}.

\begin{corollary}
Sotto le stesse ipotesi della proposizione~\ref{prop_chiusura-radicale} assumiamo che $N$ sia un campo algebricamente chiuso ed $x$ una tupla finita. Allora per ogni $p\subseteq\Delta$:

\ceq{\hfill\rad p}{=}{\Big\{t(x)\ :\ N\models\A x\,[ p(x)\imp t(x)=0]\Big\}}\QED

\end{corollary}

\begin{exercise}
Sotto le stesse ipotesi della proposizione~\ref{prop_chiusura-radicale} sia $p\subseteq\Delta$ si dimostri che le seguenti affermazioni sono equivalenti.
\begin{itemize}
\item[1.] $p$ \`e un $\Delta\jj$tipo principale;
\item[2.] $p$ \`e un ideale finitamente generato.
\end{itemize}
\end{exercise}

\section{I campi reali chiusi}

%\def\ceq#1#2#3{\parbox{30ex}{$\displaystyle #1$}\parbox{6ex}{\hfil$\displaystyle #2$}$\displaystyle  #3$}

\lavori

In tutto questo paragrafo il linguaggio \`e quello degli anelli ordinati. La teoria dei \emph{domini di integrit\`a ordinati}, che qui indicheremo con \emph{$T_{\rm dio}$}, contiene oltre agli assiomi di $T_{\rm di}\cup T_{\rm ol}$ i seguenti
\begin{itemize}
\item[t] $x<y\ \imp\ x+z<y+z$\hfill (l'ordine \`e invariante per translazione);

\item[p] $0<x\wedge 0<y\ \imp\ 0<xy$\hfill(i positivi sono sottogruppo moltiplicativo).

\end{itemize}
\`E facile verificare che $0<1$. Altrimenti, $T_{\rm ol}$ implicherebbe $1<0$, quindi per traslazione $0<-1$, e infine da \ssf{p} otterremmo $0<1$. Da questo e da \ssf{t} segue che i domini di integrit\`a ordinati hanno caratteristica $0$. 

Per assiomatizzare i domini di integrit\`a ordinati avremmo potuto usare al posto della relazione d'ordine un predicato unario di positivit\`a e richiedere (per comodit\`a scriviamo $0<x$ per il predicato di positivit\`a):
\begin{itemize}
\item[p1.] $0\not<0$;

\item[p2.] $x=0\ \vee\ 0<x\ \vee\ 0<-x$;

\item[p3.] $0<x\ \wedge\  0<y\ \ \imp\ \ 0<xy\ \wedge\  0<x+y$.
\end{itemize}
\`E immediato verificare che definendo $x<y$ come $0<y-x$ otteniamo una relazione d'ordine lineare che soddisfa \ssf{t} e \ssf{p}. 

Diremo che gli elementi positivi hanno una radice quadrata se vale

\begin{itemize}
\item[rq]   $0<y\ \imp\ \E x\ x^2=y$.
\end{itemize}


\begin{proposition}\label{prop_radicidipositivi}
Sia $M\models T_{\rm dio}$. Allora esiste $M\subseteq N\models T_{\rm dio}$ in cui tutti gli elementi positivi hanno una radice quadrata
\end{proposition}

\begin{proof}
\`E sufficiente mostrare che se $c$ un elemento positivo di $M$ allora esiste $M\subseteq N\models T_{\rm dio}$ in cui vale $\E x\ x^2=c$. Il modello $N$ richiesto dalla proposizione si ottiene iterando questa estensione. Come dominio di $N$ prendiamo $M^2$ e identifichiamo $M$ con $M\times\{0\}\subseteq N$. Le operazioni in $N$ sono definite immaginando la coppia $\<a,b\>$ come la somma formale $a+b\sqrt{c}$. Analogamente si definisce l'ordine in $N$. Precisamente, $\<a,b\>\in N$ \`e positivo in uno dei seguenti tre casi
\begin{itemize}
\item[1.] $0<a\ \wedge\ 0<b$;
\item[2.] $a\le0\ \wedge\ 0<b\ \wedge\  a^2\le b^2 c$;
\item[3.] $0<a\ \wedge\ b\le0\ \wedge\  b^2c\le a^2$.
\end{itemize}

Si verifica facilmente che valgono gli assiomi \ssf{p1}-\ssf{p3}.
\end{proof}


Inoltre, sono \emph{formalmente reali\/} ovvero vale 

\begin{itemize}
\item[fr]  $x^2\neq -1$
\end{itemize}

Pi\`u avanti dimostreremo come esercizio che in tutti i domini di integrit\`a formalmente reali \`e possibile definire un ordine lineare. Si osservi che 

\ceq{\hfill T_{\rm dio}}{\proves}{x<y\ \iff\ \E z\ x+z^2=y}.

\begin{proposition}
Sia $\Delta$ l'insieme delle formule senza quantificatori del linguaggio degli anelli (ovvero non contengono $<$).  Allora per ogni $M,N$ modelli di $T_{\rm dio}$ le seguenti affermazioni sono equivalenti
\begin{itemize}
 \item[1.] $k:M\to N$ \`e un $\Delta\jj$morfismo;
 \item[2.] $k:M\to N$ \`e un'immersione parziale.
\end{itemize}
\end{proposition}
\begin{proof}
L'implicazione \ssf{2}$\IMP$\ssf{1} \`e banale. Dimostriamo \ssf{1}$\IMP$\ssf{2}. Per la proposizione~\ref{prop_radicidipositivi} possiamo assumere che \ssf{rq} valga sia in $M$ che in $N$. 

Sia $c\in\dom k$ e siano $a\in M$ e $b\in N$ tali che $M\models a^2=c$ ed  $N\models b^2=kc$. \`E facile verificare che $k\cup\{\<a,b\>\}\ :\ M\to N$ \`e ancora un $\Delta\jj$morfismo. Iterando a sufficienza questa estensione, possiamo anche assumere che \ssf{rq} valga sia in $\<\dom k\>_M$ che in $\<\range k\>_N$


Fatte queste premesse, fissiamo due termini $t(x)$ e $s(x)$ ed una tupla $c\in(\dom k)^{|x|}$ dobbiamo mostrare che 

\ceq{\hfill M\models t(c)<s(c)}{\IMP}{N\models t(kc)<s(kc)}.

Esiste un $a\in\dom k$ tale che $M\models t(c) + a^2= s(c)$ e quindi $N\models t(kc) + (ka)^2= s(kc)$ che implica $N\models t(kc)<s(kc)$ come richiesto.
\end{proof}


\begin{comment}


%%%%%%%%%%%%%%%%%%%%%%%%%%%%%%%%%%
%%%%%%%%%%%%%%%%%%%%%%%%%%%%%%%%%%
%%%%%%%%%%%%%%%%%%%%%%%%%%%%%%%%%%
%%%%%%%%%%%%%%%%%%%%%%%%%%%%%%%%%%
\section{Sistemi di equazioni lineari}


Il prodotto diretto di due $R\jj$moduli \`e un $R\jj$modulo, verr\`a chiamato anche \emph{somma diretta\/} e denotato con $M\oplus N$. Dato un omomorfismo parziale $h:M\to N$ definito in un sottomodulo di $M$ definiamo su $M\oplus N$ la seguente relazione di equivalenza: 

\hspace*{35ex}\llap{$(a_1,b_1)\sim_h (a_2,b_2)$}%
\parbox{6ex}{\hfil$\dIFF$}%
$a_1-a_2\in\dom h$\ \ e\ \ $h(a_1-a_2) = b_2-b_1$.

Si osservi che $(a,0)\sim_h(0,ha)$ per ogni $a\in\dom h$. Ovvero, identificando $M$ ed $N$ con le sottostrutture $M\oplus 0$ e $0\oplus N$, questa \`e la relazione di equivalenza che identifica $a$ con $ha$. Denoteremo con $M\oplus_h N$ il modulo che si ottiene quozientando $M\oplus N$ con $\sim_h$. La classe di equivalenza dell'elemento $(a,b)$ verr\`a denotata con $[a,b]_h$. Se $I\subseteq M$ \`e un sottomodulo ed $h:M\to 0$ \`e l'omomorfismo parziale definito su $I$, allora $M\oplus_h 0$ viene denotato pi\`u semplicemente con $M/I$ e la classe di equivalenza dell'elemento $a$ con $[a]_I$.



Fissiamo una tupla di variabili $x$ di lunghezza $\alpha$. Le formule atomiche $\phi(x)$ a parametri in $A\subseteq M$ hanno, a meno di equivalenza, la forma $r\mdot x=t$, dove $r$ \`e una tupla in $R^{\oplus\alpha}$ ed $t$ un termine chiuso a parametri in $A$. Queste verranno chiamate \emph{equazioni (in $x$ su $A$)\/}. Se $t=0$ diremo che l'equazione \`e \emph{omegenea}. Tipicamente $x$ ed $A$ sono fissati e dovranno essere dedotti dal contesto. Diremo \emph{sistema di equazioni\/} per tipo atomico (cfr.\@ paragrafo~\ref{frammenti}). Quindi i tipi puri corrispondono quindi a sistemi di equazioni omogenee. %Fino al lemma~\ref{eliminazionequantificatoritipi} considereremo solo sistemi di equazioni in una variabile. La generalizzzione \`e semplice, ma distrarrebbe dai contenuti. 

Sia $h\;\subseteq\; R^{\oplus\alpha}\oplus N$ e sia $p(x)$ un tipo a parametri in $N$, definiamo:

\parbox{13ex}{\hfill\emph{$p_h(x)$}}%
\parbox{6ex}{\hfil$\deq$}%
\parbox{19ex}{$\displaystyle\Big\{r\mdot x=a\ :\ (r,a)\in h\Big\}$}%
\parbox{8ex}{\hfil ed}%
\parbox{3ex}{\hfill\emph{$h_p$}}
\parbox{6ex}{\hfil$\deq$}%
\parbox{20ex}{$\Big\{(r,a)\ :\ r\mdot x=a\ \in\ p\Big\}$.}%

\`E chiaro che questi operatori sono uno l'inverso dell'altro. Il lettore pu\`o facilmente verificare che se $p$ \`e consistente allora $h_p:R^{\oplus\alpha}\to N$ \`e un omomorfismo parziale. Viceversa, se $h:R^{\oplus\alpha}\to N$ \`e un omomorfismo parziale allora $p_h$ \`e consistente. Questa seconda affermazione \`e meno immediata, ma segue dal lemma~\ref{lem_fondhom} che ora dimostreremo.

Se $h\;\subseteq\;R^{\oplus\alpha}\oplus N$ scriveremo $\<h\>$, omettendo il pedice, per la sottostruttura di $R^{\oplus\alpha}\oplus N$ generata da $h$. Se $h:R^{\oplus\alpha} \to N$ \`e un omomorfismo parziale $\<h\>:R^{\oplus\alpha}\to N$ \`e la sua (unica) estensione ad un omomorfismo definito in $\<\dom h\>_{R^{\oplus\alpha}}$.

\begin{lemma}\label{lem_fondhom}
Sia $h: R^{\oplus\alpha}\to N$ un omomorfismo parziale tra moduli, allora:
\begin{itemize}
\item[1.] $p_{\<h\>}(x)$ \`e la chiusura per conseguenze logiche di $p_h(x)$;
\item[2.] $p_h(x)$ \`e un tipo primo.
\end{itemize}
\end{lemma}

\begin{proof}
Per prima cosa verifichiamo che $p_h(x)\provesR p_{\<h\>}(x)$. Le equazioni in $p_{\<h\>}(x)$ sono della forma $t(r_1,\dots,r_n)\mdot x=h(t(r_1,\dots,r_n))$ per un qualche termine puro $t(z_1\dots,z_n)$ e qualche $r_1,\dots,r_n\in\dom h$. Ma questa equazione \`e una conseguenza, modulo $T_\Rmod$, delle formule $r_i\mdot x= h(r_i)$, che appartengono tutte a $p_h(x)$.

Ora mostreremo che il seguente tipo $q(x)$ \`e consistente

\parbox{13ex}{\hfill$q(x)$}%
\parbox{6ex}{\hfil$\deq$}%
$\displaystyle p_{\<h\>}(x)\ \cup\ \Big\{r\mdot x\neq a\parbox{4ex}{\hfil:} p_h(x)\notprovesR r\mdot x= a\Big\}$.

Da questo otterremo \ssf{1} e, per il lemma~\ref{lemmatipiprimiconsistenti}, anche \ssf{2}. 

Per il teorema di compattezza \`e sufficiente dimostrare la consistenza finita, quindi possiamo assumere che $\alpha$ sia finito. Come struttura in cui realizzare $q(x)$ prendiamo $R^{\oplus\alpha}\oplus_{\<h\>} N$ e come realizzazione prendiamo la tupla $c=([e_i,0]_{\<h\>}:i<\alpha)$ dove $e_i\in R^{\oplus\alpha}$ \`e $1$ nella coordinata $i\jj$esima $0$ altrimenti. Allora $r\mdot c\ =\ [r,0]_{\<h\>}\ =\ [0,hr]_{\<h\>}\  =\ [0,a]_{\<h\>}$.
\end{proof}

Il seguente corollario \`e immediato.

\begin{corollary}\label{thmhomgen}
Siano $h, k : R^{\oplus\alpha}\to N$ omomorfismi parziali tra $R\jj$moduli. Le seguenti affermazioni sono equivalenti:
\begin{itemize}
\item[1.] $\provesR\ p_h(x)\iff p_k(x)$
\item[2.] $\<h\>\ =\ \<k\>$.
\end{itemize}
\end{corollary}

La seguente semplice osservazione ha utili conseguenze (vedi per esempio, criterio di Baer~\ref{prop_CriteioBaer}).

\begin{lemma}\label{totale_realizzato}
Sia $h : R^{\oplus\alpha}\to N$ omomorfismo parziale tra $R\jj$moduli. Le seguenti affermazioni sono equivalenti:
\begin{itemize}
\item[1.] $p_h(x)$ \`e realizzato in $N$
\item[2.] $h : R^{\oplus\alpha}\to N$ si estende ad un omomorfismo totale.
\end{itemize}
\end{lemma}
\begin{proof}
Per dimostrare la direzione \ssf{1}$\IMP$\ssf{2} fissiamo una tupla $a$ che realizza $p_h(x)$. Posto $q(x)=\atp_N(a)$, \`e immediato verificare che $h_q: R^{\oplus\alpha}\to N$ \`e l'estensione richiesta.  Dimostriamo \ssf{2}$\IMP$\ssf{1}. Scriviamo \emph{$e_i$\/} per l'elemento di $R^{\oplus\!\alpha}$ che \`e $1$ nella coordinata $i\jj$esima e $0$ in tutte le altre. Se $k : R^{\oplus\alpha}\to N$ \`e un'estensione di $h$ allora $p_h(x)\subseteq p_k(x)$. Se \`e totale allora \`e definita in $e_i$, \`e immediato che la tupla $(ke_i:i<\alpha)$ realizza $p_k(x)$.
\end{proof}

\begin{lemma}\label{noetheriano_tipiprincipali}
Per ogni omomorfismo parziale $h:R^{\oplus\alpha}\to N$ le seguenti affermazioni sono equivalenti:
\begin{itemize}
\item[1.] il modulo $\<\dom h\>_{R^{\oplus\alpha}}$ \`e finitamente generato;
\item[2.] il tipo $p_h(x)$ \`e principale.
\end{itemize} 
\end{lemma}
\begin{proof}
Dimostriamo \ssf{1}$\IMP$\ssf{2}. Siano $r_1,\dots,r_m$ dei generatori di $\<\dom h\>_{R^{\oplus\alpha}}$. Il sistema di equazioni che contiene le formule $r_i\mdot x=h\,r_i$ \`e chiaramente equivalente a $p_h(x)$. Per dimostrare che \ssf{2}$\IMP$\ssf{1} supponiamo che 

\noindent\parbox{15ex}{\ssf{1.}}\llap{$\proves\ \ p_h(x)$}%
\parbox{6ex}{\hfil$\iff$}%
$\displaystyle\bigwedge^m_{i=1}r_i\mdot x=a_i$.

Per il teorema di compattezza possiamo assumere che $r_i\in\dom h$ e $a_i=hr_i$. Sia $k=h\mathord\restriction\{r_1,\dots,r_m\}$.  Quindi \ssf{1} pu\`o essere riscritta come $\proves p_h(x)\iff p_k(x)$. Dal lemma~\ref{thmhomgen} otteniamo $\<h\>\ =\ \<k\>$ quindi  che $r_1,\dots,r_m$ generano $\<\dom h\>_{R^{\oplus\alpha}}=\<\dom k\>_{R^{\oplus\alpha}}$.
\end{proof}

Un anello $R$ si dice \emph{coerente\/} (nulla a che vedere con la nozione di teoria coerente) se per ogni $h:R\imp R$ omomorfismo parziale tra $R\jj$moduli, se $\<\range h\>_R$ \`e finitamente generato anche $\<\dom h\>_R$ \`e finitamente generato.

\begin{theorem}
Per ogni anello $R$ le seguenti affermazioni sono equivalenti:
\begin{itemize}
\item[1.] $R$ \`e coerente;
\item[2.] per ogni ogni insieme finito $A\subseteq M\models T_\Rmod$, ogni tipo atomico $p(x)\subseteq L(A)$ \`e principale.
\end{itemize}
\end{theorem}
\begin{proof}
Per dimostrare \ssf{2}$\IMP$\ssf{1} fissiamo $h:R\imp R$, un omomorfismo parziale tra $R\jj$moduli ed un insieme $A\subseteq R$ finito che genera $\<\range h\>_R$. Assumiamo \ssf{2} con $R$ per $M$, quindi che il tipo $p_h(x)$ \`e equivalente ad un tipo su $A$ e quindi ad uno della forma $\{r_ix=h(r_i): i=1,\dots,n\}$.   Posto $k=h\restriction\{r_1,\dots,r_n\}$, dal lemma~\ref{thmhomgen} otteniamo $\<h\>=\<k\>$ quindi  che $r_1,\dots,r_m$ generano $\<\dom h\>_{R}$.

Per dimostrare \ssf{1}$\IMP$\ssf{2} fissiamo un insieme finito $A\subseteq M\models T_\Rmod$, ogni tipo atomico $p(x)\subseteq L(A)$. 
\end{proof}


Un modulo si dice \emph{noetheriano\/} se tutti i suoi sottomoduli sono finitamente generati. Diremo che $R$ \`e un \emph{anello noetheriano\/} se, in quanto $R\jj$modulo, \`e noetheriano (si ricordi che i sottomoduli di $R$ sono gli ideali).

\begin{lemma}\label{sommadinoetheriani}
Siano $M$ ed $N$ due $R\jj$moduli noetheriani allora anche $M\oplus N$ \`e noetheriano.
\end{lemma}
\begin{proof} 
Dobbiamo mostrare che ogni $K\subseteq M\oplus N$ \`e finitamente generato. Per ipotesi esistono due tuple finite $a$ e $b$ che generano $K\cap \big(M\oplus 0\big)$ e $K\cap \big(0\oplus N\big)$. \`E immediato verificare che $a,b$ genera $K$.
\end{proof}

Per il lemma~\ref{noetheriano_tipiprincipali} ed il seguente corollario, in un modulo su un anello noetheriano $R$ i tutti sistemi di equazioni $p(x)$, dove $x$ di lunghezza finita, sono equivalenti a sistemi finiti.

\begin{corollary}
Se $R$ \`e un anello noetheriano ed $\alpha$ \`e finito allora $R^{\oplus\alpha}$ \`e un modulo noetheriano. Se $\alpha$ \`e infinito $R^{\oplus\alpha}$ non \`e mai noetheriano.
\end{corollary}
\begin{proof} Per $\alpha$ \`e finito il corollario si dimostra per induzione: dal lemma~\ref{sommadinoetheriani} si ottiene il passo induttivo. Se $\alpha$ \`e infinito, l'ideale $R^{\oplus\alpha}$ stesso \`e un ideale non finitamente generato. 
\end{proof}
















%%%%%%%%%%%%%%%%%%
%%%%%%%%%%%%%%%%%%%%%%
%%%%%%%%%%%%%%%%%%
%%%%%%%%%%%%%%%%%%%
%%%%%%%%%%%%%%%%%%%%%%%%%%
%%%%%%%%%%%%%
\section{Moduli liberi}

Sia $R$ un anello, $M$ un $R\jj$modulo, ed $A\subseteq M$. Una tupla $c$ di elementi di $M$ si dice \emph{libera su $A$\/} se il tipo $p(x)=\atp_M(c/A)$ \`e banale, ovvero $\provesR p(x)$.  Chiaramente la nozione non dipende da $M$ ma solo da $\<A,c\>_M$. 

\begin{proposition}\label{prop_multipli_elem_liberi}
Dati $A\subseteq M$ come sopra, sia $c$ un elemento indipendente da $A$. Allora $rc$ \`e indipendente da $A$ per ogni $r\in R$ che non divide $0$. 
\end{proposition}
\begin{proof}
Se $M\models src=0$ per qualche $s\in R$ allora $\provesR srx=0$ quindi $sr=0$ e quindi $s=0$.
\end{proof}

Un $R\jj$modulo $M$ si dice \emph{libero\/} se esiste una tupla $a$ di elementi di $M$ che genera $M$ e che \`e libera (su $\0$). Diremo che $a$ \`e una \emph{tupla di generatori libera}. La lunghezza di una tupla che genera liberamente $M$ \`e detta il \emph{rango} di $M$. La definizione di rango \`e ben data per la seguente proposizione (tutte le tupla di generatori liberi hanno la stessa lunghezza).

\begin{proposition}
Sia $M$ \`e un $R\jj$modulo libero e sia $a$ una tupla di generatori liberi di lunghezza $\alpha$. Allora $M$ \`e isomorfo a $R^{\oplus\alpha}$.
\end{proposition}

\begin{proof}
Definiamo $h=\{(r,r\mdot a):r\in R^{\oplus\alpha}\}$. Verifichiamo che $h:R^{\oplus\alpha}\to M$ \`e un isomorfismo. Totalit\`a e suriettivit\`a sono immediate, per l'iniettivit\`a supponiamo che $r\mdot a=s\mdot a$ per qualche $r,s\in R^{\oplus\alpha}$. Poich\'e $\atp_M(a)$ \`e banale allora $r\mdot x=s\mdot x$ vale in tutti gli $R\jj$moduli. Questo succede solo se $r=s$.
\end{proof}

\begin{proposition}\label{prop_pid-submoluli} 
Sia $R$ un dominio ad ideali principali, $N$ un $R\jj$modulo, $A\subseteq N$ e sia $c\in N$ indipendente da $A$. Se $M\subseteq\<A,c\>_N$ allora $M=\<A,rc\>_N$ per qualche $r\in R$. 
\end{proposition}
\begin{proof}
Chiaramente $I=\{s\in R\ :\ \<A,sc\>_N\subseteq M\}$ \`e un ideale di $R$ ed \`e quindi  generato da un elemento $r$. Verifichiamo che $M=\<A,rc\>_N$. L'inclusione $\<A,sc\>_N\subseteq M$ \`e ovvia perch\`e $r\in I$. Per dimostrare $M\subseteq\<A,sc\>_N$, fissiamo un elemento arbitrario di $M$. Poich\'e $M \subseteq\<A,c\>_N$ possiamo assumere che questo elemento abbia la forma $t+sc$ per un qualche $s\in R$ ed un qualche $t$ termine chiuso a parametri in $A$. Quindi $\<A,sc\>_N\subseteq M$, ovvero $s\in I$. Allora $s$ \`e un multiplo di $r$. Segue che $t+sc\in\<A,rc\>_N$.
\end{proof}




\begin{proposition}
Sia $R$ un dominio ad ideali principali e siano $M\subseteq N$ due $R\jj$moduli. Allora se $N$ \`e libero e di rango finito anche $M$ \`e libero. In particolare, se $\{a_i : i<n\}$ genera liberamente $N$, allora esistono $r_0,\dots,r_{n-1}\in R$ tali che la tupla $\{r_ia_i: i<n\}\sm\{0\}$ genera liberamente $M$.
\end{proposition}

\begin{proof}
Mostriamo per induzione su $m\le n$ che esistono $r_0,\dots,r_{n-1}\in R$

\def\ceq#1#2#3{\parbox{30ex}{$#1$}\parbox{6ex}{\hfil$#2$}$#3$}

\ceq{\ssf{1.}\hfill\<r_0a_0,\dots,r_{m-1}a_{m-1}\>_N}{=}{M \cap \<a_0,\dots,a_{m-1}\>_N}.

Per $m=0$ l'affermazione \`e vuota. Assumiamo \ssf{1} vera e abbreviamo con $A_{m-1}$ il modulo in detta identit\`a.  \`E immediato verificare che $M \cap \<a_0,\dots,a_{m}\>_N \subseteq \<A_{m-1},a_m\>_N$, inoltre $a_m$ \`e indipendente da $A_{m-1}$. Quindi dalla proposizione~\ref{prop_pid-submoluli} segue l'esistenza di $r_m\in R$ che rende \ssf{1} vera con $m$ per $m-1$. 

Quindi $\{r_ia_i: i<n\}$ genera $M$. Che $\{r_ia_i: i<n\}\sm\{0\}$ sia un insieme libero segue dalla proposizione~\ref{prop_multipli_elem_liberi}. (Serve sottrarre $0$ nel caso qualche $r_i$ sia nullo.) 
\end{proof}


\begin{proposition}
Sia $R$ un anello che considereremo come $R\jj$modulo. Allora le seguenti affermazioni sono equivalenti:
\begin{itemize}
\item[1.] $R$ \`e un dominio ad ideali principali;
\item[2.] ogni sottomodulo non nullo di $R$ \`e isomorfo ad $R$.
\end{itemize}
\end{proposition}
\begin{proof}
Dimostriamo \ssf{1}$\IMP$\ssf{2}. Sia $I\subseteq R$ un sottomodulo (ovvero, un ideale). Sia $a$ un generatore di $I$ e definiamo $h=\{(r,ra): r\in R\}$. Ovviamente $h:R\to I$ \`e un epimorfismo, ed \`e anche iniettivo perch\'e non ci sono divisori dello $0$.  Dimostriamo \ssf{2}$\IMP$\ssf{1}. Sia $I\subseteq R$ un sottomodulo ed $h:R\to I$ un isomorfismo. Poich\'e $R$ \`e generato da $1$, allora $I$ \`e generato da $h1$. Quindi ogni ideale di $R$ \`e principale. Verifichiamo che $R$ \`e un dominio. Sia $a\in R$ arbitrario e sia $I_a=\{ar=0:r\in R\}$ e sia $h:R\to I_a$ un isomeofismo. Allora $h(a1)=ah(1)=0$ e quindi $a=0$.
\end{proof}




%%%%%%%%%%%%%%%%%%
%%%%%%%%%%%%%%%%%%%%%%
%%%%%%%%%%%%%%%%%%
%%%%%%%%%%%%%%%%%%%
%%%%%%%%%%%%%%%%%%%%%%%%%%
%%%%%%%%%%%%%%%%%%%%%%%%%%
\section{Moduli iniettivi}

Un $R\jj$modulo $N$ di dice \emph{iniettivo\/} se, per ogni altro $R\jj$modulo $M$, ogni omomorfismo parziale $k:M\to N$ ha un estensione ad un omomorfismo (totale). 

L'equivalenza \ssf{1}$\IFF$\ssf{4} della proposizione seguente \`e detta \emph{criterio di Baer}.

\begin{proposition}\label{prop_CriteioBaer}
Per ogni $R\jj$modulo $N$, le seguenti affermazioni sono equivalenti:
\begin{itemize}
\item[1.] $N$ \`e iniettivo;
\item[2.] ogni sistema di equazioni $p(x)$ a parametri in $N$ se consistente, \`e realizzato in $N$;
\item[3.] ogni equazione della forma $rx=a$ con $r\in R$ ed $a\in N$ se consistente ha soluzione in $N$ (ovvero \ssf{2} vale con $x$ una singola variabile).
\item[4.] ogni omomorfismo parziale $k:R\to N$ si estende ad uno totale.
\end{itemize}
\end{proposition}
\begin{proof}
L'implicazione \ssf{1}$\IMP$\ssf{2} \`e una conseguenza della lemma~\ref{totale_realizzato}. L'implicazione \ssf{2}$\IMP$\ssf{3}  \`e ovvia mentre \ssf{2}$\IMP$\ssf{3} segue dal   lemma~\ref{totale_realizzato}.

Per dimostre \ssf{2}$\IMP$\ssf{1} \`e sufficiente verificare che ogni omomorfismo $k:M\to N$ si estende ad un omomorfismo $k':M\to N$ definito in $\<\dom k,a\>_M$, dove $a$ \`e un arbitrario elemento di $M$.  Posto $h=\{(r,k(ra)): ra\in\dom k\}$, osserviamo che $h:R\to N$ \`e un omomorfismo parziale e quindi, per ipotesi, ha un estensione totale $h':R\to N$. Gli elementi di $\<\dom k,a\>_M$ hanno la forma $t+ra$ per qualche $t\in\<\dom k\>_M$ e qualche $r\in R$. 

Definiamo $k'(t+ra)=kt+(h'r)a$ e verifichiamo che questo definisce una mappa $k':M\to N$, ovvero che la definizione \`e ben data. Per la linearit\`a di $k$ ed $h'$ sufficiente verificare che se $t+ra=0$ allora $kt+(h'r)a=0$. Ma questo segue dalla definizione di $h'$, infatti se $t+ra=0$ allora $r\in I$ quindi $h'(r)=k(ra)$.

Ora \`e immediato verificare che $k':M\to N$ \`e l'omomorfismo desiderato.
\end{proof}

Siano $r\in R$ un divisore dello $0$ e sia $a\in M$ arbitrario. L'equazione $rx=a$ ha soluzione in $M$ tali che $rs=0$ e sia $a\neq0$ un elemento di $M$, un $R\jj$modulo. Allora o $rx=a$ o $sx=a$ non ha soluzione. In fatti se $sc=a$ allora $ra=0$. Analogamente $sa=0$.  Allora Un $R\jj$modulo $M$ si dice \emph{divisibile\/} se per ogni $a\in M$ ed ogni $r\in R$ che non divide $0$ l'equazione $rx=a$ ha soluzione in $M$. Si dice \emph{senza torsione\/} se per ogni $a$ ed $r$ come sopra, l'equazione $rx=a$ ha al pi\`u una soluzione in $M$ (equivalentemente, se $rx=0$ non ha altre soluzione oltre a $0$).

Dimostriamo che ogni modulo iniettivo \`e divisibile e che il viceversa vale se $R$ \`e un dominio ad ideali principali. 

\begin{proposition}
Se $R$ un \`e  dominio ad ideali principali allora per ogni $R\jj$modulo $N$ le seguenti affermazioni sono equivalenti:
\begin{itemize}
\item[1.] $N$ \`e iniettivo;
\item[2.] $N$ \`e divisibile.
\end{itemize}
\end{proposition}
\begin{proof}
Dimostriamo  \ssf{1}$\IMP$\ssf{2}. Fissiamo un elemento $a$ in un $R\jj$modulo iniettivo $M$ ed un $r\in R$. Mostriamo che l'equazione $rx=a$ ha soluzione in $M$. Definiamo $k=\{\<r,a\>\}$. Poich\'e $r$ non divide $0$, la mappa $k:R\to N$ \`e un omomorfismo parziale. Quindi si estende ad un omomorfismo totale $h:R\to N$. L'elemento $h1$ \`e la soluzione cercata. Per dimostrare \ssf{2}$\IMP$\ssf{1} applichiamo il criterio di Baer: fissiamo un arbitrario omomorfismo parziale $k:R\to N$ un omomorfismo parziale e mostriamo che possiamo estenderlo ad un omomorfismo totale. Per ipotesi $\<\dom k\>_R$ \`e principale, sia $r$ un suo generatore.  Sia $a$ una soluzione dell'equazione $rx=kr$. La funzione $h=\{\<r,ra\>:r\in R\}$ estende $k$, quindi l'omomorfismo richiesto \`e $h:R\to N$.
\end{proof}

\begin{lemma}\label{eliminazionequantificatoritipi}
Per ogni sistema di equazioni omogenee $p(x,z)$, dove $x$ e $z$ hanno lunghezza arbitraria, esiste un sistema di equazioni omogenee $q(z)$ tale che per ogni modulo $M$ e per ogni tupla $a$ di elementi di $M$,

\hspace*{15ex}\llap{$M\models q(a)$}%
\parbox{6ex}{\hfil$\IFF$}%
$N\models \E x\, p(x,a)$ per un qualche $N\supseteq \<a\>_M$.
\end{lemma}

\begin{proof}
Definiamo i seguenti tipi

\hspace*{15ex}\llap{$p^+(x,z)$}%
\parbox{6ex}{\hfil$=$}%
$\Big\{r\mdot x=s\mdot z\ :\ r\in R^{\lh x},\ \ s\in R^{\le s},\ \ p(x,z)\proves r\mdot x=s\mdot z\Big\}$.

\hspace*{15ex}\llap{$p^-(x,z)$}%
\parbox{6ex}{\hfil$=$}%
$\Big\{r\mdot x\neq s\mdot z\ :\ r\in R^{\lh x},\ \ s\in R^{\le s},\ \ p(x,z)\notproves r\mdot x=s\mdot z\Big\}$.

Chiaramente $p^+(x,z)\cup p^-(x,z)\proves p(x,z)$.  Ora fissiamo opportune tuple $\bar r=\<r_i:i<\lambda\>$ ed $\bar s=\<s_i:i<\lambda\>$ tali che

\hspace*{15ex}\llap{$p^+(x,z)$}%
\parbox{6ex}{\hfil$=$}%
$\Big\{r_i\mdot x= s_i\mdot z\ :\ i<\lambda\Big\}$. 

Definiamo $q(w)\ \deq\ \etp_{R^{\lh x}}(r)$, dove $w=\<w_i:i<\lambda\>$. Con leggero abuso di notazione scriveremo $q(s\mdot z)$ per il tipo ottenuto sostituendo in $q(w)$ le variabili $w_i$ con i temini $s_i\mdot z$. Dimostriamo che il tipo $q(s\mdot z)$ \`e quello richiesto dal lemma. 

Per verificare la direzione $\IMP$, osserviamo che se $M\models q(s\mdot a)$ allora la mappa $h:R^{\lh x}\to M$, dove $h=\big\{\<r_i,s_ia\>\ :\ i<\lambda\big\}$, \`e un omomorfismo parziale e che $p_h(x)=p^+(x,a)$. Quindi, per il lemma~\ref{thmfondhom}, il tipo $p^+(x,a)$ \`e primo, quindi un qualche  $N\supseteq \<a\>_M$ realizza $p^+(x,a)\cup p^-(x,a)$. Per la direzione $\PMI$ assumiamo che $\E x\,p(x,a)$ sia consistente, e fissiamo $N\models p(b,a)$. Una generica formula in $q(s\mdot z)$ ha la forma

\hspace*{15ex}%
\parbox{6ex}{}%
$\displaystyle\sum_{i\in I}u_i (s_i\mdot z)\ \ =\ \ 0$

per un qualche insieme finito $I\subseteq\lambda$ ed una tupla $\<u_i:i\in I\>$ di elementi di $R$. Poich\'e per definizione $R\models  q(r)$ otteniamo

\hspace*{15ex}\llap{$R$}%
\parbox{6ex}{\hfil$\models$}%
$\displaystyle\sum_{i\in I}u_i r_i\ \ =\ \ 0$

e quindi

\hspace*{15ex}\llap{$M$}%
\parbox{6ex}{\hfil$\models$}%
$\displaystyle\Big(\sum_{i\in I}u_i r_i\Big)\, b\ \ =\ \ \sum_{i\in I}u_i (r_i b)$ 

Poich\'e $M\models p(b,a)$, possiamo sostituire $r_i b$ con $s_i\mdot a$ e ottenere 

\hspace*{15ex}\llap{$M$}%
\parbox{6ex}{\hfil$\models$}%
$\displaystyle\sum_{i\in I}u_i (s_i\mdot a)\ \ =\ \ 0$

Dall'arbitrariet\`a di $I\subseteq\lambda$ segue $M\models q(s\mdot a)$ come richiesto.
\end{proof}




\begin{lemma}
le seguenti affermazioni sono equivalenti:
\begin{itemize}
\item[1.] $N$ \`e iniettivo;
\item[2.] ogni immersione parziale $k:M\to N$, tale che $|k|<|M|\le|N|$, si estende ad un'immersione (totale).
\end{itemize}
\end{lemma}
\begin{proof}

\end{proof}



\section{Altro}









\begin{lemma}\label{eliminazionequantificatoritipi}
Per ogni tipo formula libera $\phi(x,z)$, dove $x$ e $z$ hanno lunghezza arbitraria, esiste un sistema di equazioni omogenee $q(z)$ tale che per ogni modulo $M$ e per ogni tupla $a$ di elementi di $M$,

\hspace*{15ex}\llap{$M\models q(a)$}%
\parbox{6ex}{\hfil$\IFF$}%
$\phi(x,a)$ \`e consistente.

Si ricordi che, come definito in~\ref{notationConseguenzaConsistenzaModuli}, per consistente intendiamo realizzato in qualche modulo che contiene $\<a\>_M$.
\end{lemma}

\begin{proof}
Sia $\alpha$ la lunghezza di $x$. \`E sufficiente dimostrare il lemma per formule $\phi(x,z)$ della forma $\phi_+(x,z)\wedge\phi_-(x,z)$ dove $\phi_+(x,z)$ \`e congiunzione di equazioni e $\phi_-(x,z)$ \`e congiunzione di negazioni di equazioni. Infatti ogni formula libera \`e disgiunzione di formule $\phi_i(x,z)$ di questa forma di questa forma ed \`e immediato verificare che la disgiunzione dei corrispondenti tipi $q_i(z)$ dimostra il lemma.

Supponiamo quindi che 

\parbox{15ex}{\hfill $\phi_+(x,z)$}%
\parbox{6ex}{}%
$\displaystyle\bigwedge^n_{i=1} r_i\mdot x= t_i(z)$.

dove $r_i\in R^{\oplus\alpha}$ e $t_i(z)$ sono termini puri. Sia $q'(w_1,\dots,w_n)\ \deq\ \atp_{R^{\oplus\alpha}}(r_1,\dots,r_n)$ e infine sia $q(z)=q'(t_1(z),\dots,t_n(z))$.  Dimostriamo che $q(z)$ \`e il tipo richiesto dal lemma. 

Per verificare la direzione $\IMP$, osserviamo che se $M\models q(a)$ allora la mappa $h:R^{\oplus\alpha}\to M$, dove $h=\big\{\<r_i,t_i(a)\>\ :\ i<\lambda\big\}$, \`e un omomorfismo parziale e che $p_h(x)$ \`e equivalente a $\phi_+(x,a)$. Poich\'e per il lemma~\ref{thmfondhom} il tipo $p_h(x,a)$ \`e primo, otteniamo che un qualche $N\supseteq \<a\>_M$ che realizza $p_h(x,a)\cup p^-_h(x,a)$. Se $\phi(x,a)$ \`e consistente  $p^-_h(x,a)$


Siano $p^+(x,z)$ e $p^-(x,z)$ definiti come nel lemma~\ref{lemmatipiprimiconsistenti2} con $\Delta$ l'insieme delle formule atomiche con variabili in $x,z$. Ora sia $\big\{\ :\ i<\lambda\big\}$ l'insieme delle equazioni che occorrono in $p^+(x,z)$, dove $r_i\in R^{\oplus\alpha}$ e $t_i(z)$ sono termini puri. Sia $q'(\<w_i:i<\lambda\>)\ \deq\ \atp_{R^{\oplus\alpha}}(\<r_i:i<\lambda\>)$ e sia $q(z)$ il tipo ottenuto sostituendo $t_i(z)$ a $w_i$ in $q'(\<w_i:i<\lambda\>)$.  Dimostriamo che $q(z)$ \`e il tipo richiesto dal lemma. 

Per verificare la direzione $\IMP$, osserviamo che se $M\models q(a)$ allora la mappa $h:R^{\oplus\alpha}\to M$, dove $h=\big\{\<r_i,t_i(a)\>\ :\ i<\lambda\big\}$, \`e un omomorfismo parziale e che $p_h(x)=p^+(x,a)$. Poich\'e per il lemma~\ref{thmfondhom} il tipo $p^+(x,a)$ \`e primo, dal lemma~\ref{lemmatipiprimiconsistenti2} otteniamo che un qualche  $N\supseteq \<a\>_M$ realizza $p^+(x,a)\cup p^-(x,a)$.  Per la direzione $\PMI$ assumiamo che $\E x\,p(x,a)$ sia consistente, e fissiamo $N\models p(b,a)$. Una generica formula in $q(z)$ ha la forma

\hspace*{15ex}%
\parbox{6ex}{}%
$\displaystyle\sum_{i\in I}s_i t_i(z)\ \ =\ \ 0$

per un qualche insieme finito $I\subseteq\lambda$ ed una tupla $\<s_i:i\in I\>$ di elementi di $R$. Poich\'e per costruzione $R^{\oplus\alpha}\models  q'((\<r_i:i<\lambda\>)$ otteniamo

\hspace*{15ex}\llap{$R$}%
\parbox{6ex}{\hfil$\models$}%
$\displaystyle\sum_{i\in I}u_i r_i\ \ =\ \ 0$

e quindi

\hspace*{15ex}\llap{$M$}%
\parbox{6ex}{\hfil$\models$}%
$\displaystyle\Big(\sum_{i\in I}u_i r_i\Big)\mdot b\ \ =\ \ \sum_{i\in I}u_i (r_i\mdot b)$ 

Poich\'e $M\models p(b,a)$, possiamo sostituire $r_i b$ con $t_i(a)$ e ottenere 

\hspace*{15ex}\llap{$M$}%
\parbox{6ex}{\hfil$\models$}%
$\displaystyle\sum_{i\in I}u_i t_i(a)\ \ =\ \ 0$

Dall'arbitrariet\`a di $I\subseteq\lambda$ segue $M\models q(z)$ come richiesto.
\end{proof}




\begin{definition}
%Diremo che $R$ \`e un \emph{anello coerente\/} (nulla a che vedere con l'omonima nozione in logica) se ogni sistema di equazioni omogenee $p(x)$, dove $x$ ha ariet\`a finita, realizzato in $R$ \`e equivalente modulo $T_\Rmod$ ad un sistema finito.

Diremo che $R$ \`e un \emph{anello coerente\/} (nulla a che vedere con l'omonima nozione in logica) se per ogni $r$ tupla finita di elementi di $R^{\oplus n}$, il tipo $\atp_{R^{\oplus n}}(r)$ \`e principale.
\end{definition}




\begin{lemma}\label{eqformule}
Per ogni formula libera $\phi(x,z)$, dove $x$ e $z$ hanno lunghezza arbitraria, esiste un sistema di equazioni finito $\psi(z)$ tale che per ogni modulo $M$ e per ogni tupla $a$ di elementi di $M$,

\hspace*{15ex}\llap{$M\models \psi(a)$}%
\parbox{6ex}{\hfil$\IFF$}%
$\phi(x,a)$ \`e consistente.

\end{lemma}











\begin{lemma}\label{eliminazionequantificatoritipi}
Per ogni tipo libero $p(x,z)$, dove $x$ e $z$ hanno lunghezza arbitraria, esiste un sistema di equazioni omogenero $q(z)$ tale che per ogni modulo $M$ e per ogni tupla $a$ di elementi di $M$,

\hspace*{15ex}\llap{$M\models q(a)$}%
\parbox{6ex}{\hfil$\IFF$}%
$p(x,a)$ \`e consistente.

Si ricordi che, come definito in~\ref{notationConseguenzaConsistenzaModuli}, per consistente intendiamo realizzato in qualche modulo che contiene $\<a\>_M$.
\end{lemma}

\begin{proof}
Sia $\alpha$ la lunghezza di $x$. Definiamo i seguenti tipi

\hspace*{15ex}\llap{$p^+(x,z)$}%
\parbox{6ex}{\hfil$=$}%
$\Big\{r\mdot x=t(z)\ :\ r\in R^{\oplus\alpha},\ \ t(z),\ \ p(x,z)\proves r\mdot x= t(z)\Big\}$.

\hspace*{15ex}\llap{$p^-(x,z)$}%
\parbox{6ex}{\hfil$=$}%
$\Big\{r\mdot x\neq t(z)\ :\ r\in R^{\oplus\alpha},\ \ t(z),\ \ p(x,z)\notproves r\mdot x= t(z)\Big\}$.

Chiaramente $p^+(x,z)\cup p^-(x,z)\proves p(x,z)$.  Ora fissiamo opportune tuple $\bar r=\<r_i:i<\lambda\>$ ed $\bar s=\<s_i:i<\lambda\>$ tali che

\hspace*{15ex}\llap{$p^+(x,z)$}%
\parbox{6ex}{\hfil$=$}%
$\Big\{r_i\mdot x= s_i\mdot z\ :\ i<\lambda\Big\}$. 

Definiamo $q(w)\ \deq\ \etp_{R^{\lh x}}(r)$, dove $w=\<w_i:i<\lambda\>$. Con leggero abuso di notazione scriveremo $q(s\mdot z)$ per il tipo ottenuto sostituendo in $q(w)$ le variabili $w_i$ con i temini $s_i\mdot z$. Dimostriamo che il tipo $q(s\mdot z)$ \`e quello richiesto dal lemma. 

Per verificare la direzione $\IMP$, osserviamo che se $M\models q(s\mdot a)$ allora la mappa $h:R^{\lh x}\to M$, dove $h=\big\{\<r_i,s_ia\>\ :\ i<\lambda\big\}$, \`e un omomorfismo parziale e che $p_h(x)=p^+(x,a)$. Quindi, per il lemma~\ref{thmfondhom}, il tipo $p^+(x,a)$ \`e primo, quindi un qualche  $N\supseteq \<a\>_M$ realizza $p^+(x,a)\cup p^-(x,a)$. Per la direzione $\PMI$ assumiamo che $\E x\,p(x,a)$ sia consistente, e fissiamo $N\models p(b,a)$. Una generica formula in $q(s\mdot z)$ ha la forma

\hspace*{15ex}%
\parbox{6ex}{}%
$\displaystyle\sum_{i\in I}u_i (s_i\mdot z)\ \ =\ \ 0$

per un qualche insieme finito $I\subseteq\lambda$ ed una tupla $\<u_i:i\in I\>$ di elementi di $R$. Poich\'e per definizione $R\models  q(r)$ otteniamo

\hspace*{15ex}\llap{$R$}%
\parbox{6ex}{\hfil$\models$}%
$\displaystyle\sum_{i\in I}u_i r_i\ \ =\ \ 0$

e quindi

\hspace*{15ex}\llap{$M$}%
\parbox{6ex}{\hfil$\models$}%
$\displaystyle\Big(\sum_{i\in I}u_i r_i\Big)\, b\ \ =\ \ \sum_{i\in I}u_i (r_i b)$ 

Poich\'e $M\models p(b,a)$, possiamo sostituire $r_i b$ con $s_i\mdot a$ e ottenere 

\hspace*{15ex}\llap{$M$}%
\parbox{6ex}{\hfil$\models$}%
$\displaystyle\sum_{i\in I}u_i (s_i\mdot a)\ \ =\ \ 0$

Dall'arbitrariet\`a di $I\subseteq\lambda$ segue $M\models q(s\mdot a)$ come richiesto.
\end{proof}
















Sia $R$ un dominio di integrit\`a. Un $R\jj$modulo si dice \emph{senza torsione} se  per ogni $r\in R\sm\{0\}$ vale il seguente assioma

\begin{itemize}
\item[mst] $x\neq0\ \imp\ rx\neq 0$
\end{itemize}

Si osservi che \ssf{mst} non pu\`o mai vale se se $R$ ha divisori dello zero. La teoria degli $R\jj$moduli senza torsione verr\`a indicata con \emph{$T_{R{\rm mst}}$}. Un $R\jj$modulo si dice \emph{divisibile\/} se \`e senza torsione e inoltre soddisfa i seguenti assiomi per ogni $r\in R\sm\{0\}$

\begin{itemize}
\item[mnb] $\E x\ x\neq 0$\hfill(questo dice semplicemente che il modulo \`e non banale)
\item[md] $\E y\ ry=x$
\end{itemize}

La teoria degli $R\jj$moduli divisibili e senza torsione la indicheremo con \emph{$T_\Rmod\st$}. I gruppi $\QQ$ e $\QQ\times\QQ$, considerati come $\ZZ\jj$moduli, sono moduli divisibili mentre  $\ZZ$ e $\QQ\times\ZZ$ non lo sono. Tutti gli spazi vetoriali sono banalmente divisibili e senza torsione.

Se $M$ un $R\jj$modulo,  $A\subseteq M$, e $x$ una singola variabile. Per il lemma~\ref{formacanonicaterminiga}, le formule atomiche $\phi(x)$ a parametri in $A$ sono, a meno di equivalenza, equazioni della forma $rx=t$ dove $r\in R$ e $t$ \`e un termine chiuso a parametri in $A$. Se non \`e torsione, e una soluzione esiste, questa \`e unica. Infatti se $rx_1=t$ e $rx_2=t$ allora $r(x_1-x_2)=0$ e dall'assioma \ssf{mst} otteniamo $x_1=x_2$. Se il modulo \`e anche divisibile ogni equazione di questa forma con $r\neq0$ ha soluzione. Formalizziamo questa osservazione in un lemma per riferimenti futuri.

\begin{lemma}\label{consistenzastabileallinsu}
Siano $A\subseteq M\models T_\Rmod\st$. Sia $\phi(x)$ una formula atomica a parametri in $A$ soddisfatta in $M$. Allora $N\models\E x\,\phi(x)$ per ogni $N$ tale che $\<A\>_M\subseteq N\models T_\Rmod\st$.
\end{lemma}

\begin{proof}
Modulo $T_\Rmod\st$ la formula $\phi(x)$ \`e equivalente ad una equazione della forma $rx=t$ con $t$ un termine chiuso a parametri in $A$. Nel caso $r\neq0$ l'equazione $rx=t$ ha sempre soluzione in $N\models T_\Rmod\st$. Se $r=0$ allora l'equazione ha soluzione solo se \`e banale ovvero $t=0$.  Si osservi che 

\hfil $M\models t=0\parbox{7ex}{\hfil$\IFF$}\<A\>_M\models t(a)=0\parbox{7ex}{\hfil$\IFF$}\<A\>_N\models t=0\parbox{7ex}{\hfil$\IFF$}N\models t=0$.

Quindi in questo caso ha soluzione in $M$ se e solo se ha soluzione in $N$.
\end{proof}

Per enunciare il prossimo teorema serve precisare un po' di terminologia. Fissiamo una teoria $T$ ed un insieme di parametri $A$ in un qualche $M\models T$. Diremo che la la formula $\phi(x)$ a parametri in $A$ \`e \emph{banale\/} se \`e una tautologia modulo $T\cup\Diag(M/A)$. Equivalentemente, se $N\models\A x\,\phi(x)$ per ogni $N\models T$ che contiene $\<A\>_M$. Diremo che $\phi(x)$ \`e consistente se \`e consistente con $T\cup\Diag(M/A)$. Diremo che $\phi(x)$ \`e \emph{completa\/} se $\phi(x)$, considerata come $\Delta\jj$tipo con $\Delta$ l'insieme delle formule atomiche con parametri in $A$, \`e completa modulo $T\cup\Diag(M/A)$. (Cfr.\@ paragrafo~\ref{frammenti}.) Questo vuol dire che per ogni $N\models T$ che contiene $\<A\>_M$ e per ogni formula $\psi(x)$ a parametri in $A$ una delle seguenti due possibilit\`a occorre
\begin{itemize}
\item[+] $N\models\A x\,[\phi(x)\ \imp \phantom{\neg}\psi(x)]$
\item[--]$N\models\A x\,[\phi(x)\ \imp \neg\psi(x)]$.
\end{itemize} 
L'occorrenza dell'uno o dell'altro caso dipende da $\psi$ ma \`e indipendente da $N$.

Possiamo parafrasare queste due definizioni fissando $a$, una enumerazione di $A$, e scrivendo $\phi(a,x)$ al posto di $\phi(x)$, dove $\phi(z,x)$ \`e una formula pura. Allora, posto $q(z)=\qftp(a)$, otteniamo che $\phi(a,x)$ \`e banale se $q(z)\provesT\phi(z,x)$ ed \`e completa se $q(z)\cup\{\phi(z,x)\}$, in quanto tipo libero e puro, \`e completo. Infatti un modello che realizza $q(z)$ \`e, a meno di isomorfismo, un modello che contiene $\<A\>_M$.


\end{comment}

