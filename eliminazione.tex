\chapter{Eliminazione dei quantificatori}
\label{eliminazione}

Vedremo per prima cosa alcuni teoremi che legano la complessit\`a sintattica di una formula alla classe di morfismi che la preservano. Useremo questi per ottenere un criterio generale per dimostrare l'eliminazione dei quantificatori. 

 
\section{Teoremi di preservazione}
\label{TeoremidiPreservazione}

In questo paragrafo $T$ \`e una qualunque teoria consistente senza modelli finiti e $\Delta$ un insieme di formule pure chiuso per sostituzione di variabili con altre variabili. In prima lettura si pu\`o immaginare che $T$ sia vuota e $\Delta$ sia $L_{\rm qf}$. 

Se $\ssf{C}\subseteq\{\A,\E,\neg,\vee,\wedge\}$ \`e un insieme di connettivi, scriveremo \emph{$\ssf{C}\Delta$} per la la chiusura di $\Delta$ rispetto a tutti i connettivi in $\ssf{C}$. Scriveremo \emph{$\neg\Delta$} per l'insieme che contiene le negazioni delle formule in $\Delta$. Attenzione a non confonderlo con $\{\neg\}\Delta$.





% Diremo che $\Delta$ \`e un \emph{insieme di eliminazione\/} per $T$ se ogni formula $\phi(x)$, dove $x$ \`e un'arbitraria tupla di variabili che varia con $\phi(x)$, \`e equivalente modulo $T$ ad una formula in $\Delta$. Detto esplicitamente, se per ogni formula $\phi(x)$ esiste una formula $\psi(x)$ in $\Delta$ tale che
% 
% \hfil$\displaystyle T\proves\ \phi(x)\iff\psi(x)$.
% 
% Per esempio $\Delta$ pu\`o essere l'insieme delle formule senza quantificatori, l'insieme delle formule esistenziali, o delle combinazioni booleane di queste. Quando $\Delta$ \`e l'insieme delle formule senza quantificatori 



Ricordiamo che un \emph{$\Delta$-morfismo\/} \`e una mappa $k:M\imp N$ che preserva la verit\`a delle formule in $\Delta$. Useremo senza ulteriore commento quanto stabilito nell'osservazione~\ref{oss_Delta-morfismi}. Faremo uso anche della seguente: 

\begin{remark}\label{prop_pokj}
Fissiamo $M\models T$ ed una tupla $a$ di elementi di $M$. Sia $p(x)=\Deltatp_M(a)$. Per ogni formula pura $\phi(x)$ le seguenti affermazioni sono equivalenti:
\begin{itemize}
\item[1.] $N\models\phi(ka)$ per ogni $N\models T$ e per ogni $\Delta$-morfismo $k:M\imp N$ definito in $a$;
%\item[1'.] $N\models\phi(b)$ per ogni $N\models T$ ed ogni tupla $b$ in $N$ tale che $M,a\Rrightarrow_\Delta N,b$.
\item[2.] $T\ \proves\  p(x)\imp\phi(x)$.
\end{itemize}
Infatti, \ssf{2}$\IMP$\ssf{1} \`e immediata, e per verificare \ssf{1}$\IMP$\ssf{2} 
assumiamo $\neg$\ssf{2} e fissiamo $N\models T$ ed una tupla $b$ tale che $N\models p(b)\wedge\neg\phi(b)$. La mappa $k:M\to N$ con $k=\{\<a,b\>\}$ prova $\neg\ssf{1}$.\QED
\end{remark}

Il seguente \`e a volte chiamato Lyndon-Robinson Lemma: 

\begin{theorem}\label{qfdefinability}
Sia $\phi(x)$ una formula pura. Le seguenti affermazioni sono equivalenti:
\begin{itemize}
\item[1.] $\phi(x)$ \`e, modulo $T$, equivalente ad una formula in $\{\mathord\wedge\!\mathord\vee\}\Delta$;
\item[2.] ogni $\Delta$-morfismo tra modelli di $T$ preserva $\phi(x)$.
%\item[2'.] $M\models\phi(a)\IMP N\models\phi(b)$ per ogni $M,a\Rrightarrow_\Delta N,b$ con $M$ ed $N$ modelli di $T$.
\end{itemize}
\end{theorem}
\begin{proof} L'implicazione \ssf{1}$\IMP$\ssf{2} \`e ovvia, dimostriamo \ssf{2}$\IMP$\ssf{1}. Da \ssf{2}, per l'osservazione~\ref{prop_pokj}, abbiamo la seguente

\hfil$\displaystyle T\ \ \proves\ \ \ \phi(x)\ \iff\bigvee_{T\proves p(x)\imp\phi(x)} p(x)$

dove $p(x)$ corre su tutti i $\Delta$-tipi. Per compattezza otteniamo

\hfil$\displaystyle T\ \ \proves\ \ \ \phi(x)\ \iff\bigvee_{T\proves\psi(x)\imp\phi(x)} \psi(x)$

dove $\psi(x)$ corre sulle formule in $\{\wedge\}\Delta$. Ora osserviamo che la disgiunzione infinita pu\`o essere sostituita con una finita per compattezza.
\end{proof}
















% 
% \begin{corollary}\label{prop_po}
% Fissiamo un modello $M\models T$ ed una tupla $a$ di elementi di $M$. Sia $p(x)=\E\Deltatp_M(a)$. Per ogni formula pura $\phi(x)$ le seguenti affermazioni sono equivalenti:
% \begin{itemize}
% \item[1.] $N\models\phi(ha)$ per ogni modello $N\models T$ e per ogni $\Delta$-morfismo totale $h:M\imp N$;
% \item[2.] $T\ \cup\ p(x)\proves\phi(x)$.
% \end{itemize}
% \end{corollary}
% \begin{proof} 
% L'implicazione \ssf{2}$\IMP$\ssf{1} \`e immediata per  la proposizione~\ref{presesis}. Dimostriamo \ssf{1}$\IMP$\ssf{2}. Se  $\neg\ssf{2}$ allora esiste un $N\models T$ ed una tupla $b$ tale che  $N\models p(b)\wedge\neg\phi(b)$. Possiamo assumere $N$ sia $\lambda$ satturo per $\lambda$ sufficientemente grande. La mappa $k:M\to N$ con $k=\{\<a,b\>\}$ \`e un $\E\Delta$-morfismo e quindi si estende ad un $\E\Delta$-morfismo totale $k:M\imp N$ che prova $\neg\ssf{1}$.
% \end{proof}

\begin{lemma}\label{lem_EDelta_estensione}
Sia $N$ una struttura $\lambda$-satura e sia $k:M\imp N$ un $\Delta$-morfismo di cardinalit\`a $<\lambda$. Allora le seguenti affermazioni sono equivalenti:
\begin{itemize}
\item[1.] $k:M\imp N$ \`e un $\{\E\wedge\}\Delta$-morfismo;
\item[2.] per ogni $b\in M^{<\omega}$ esiste un $\{\E\wedge\}\Delta$-morfismo $h:M\imp N$ tale che $k\subseteq h$ e $b\in\dom h^{<\omega}$.
\item[3.] come \ssf{2}  ma con $\Delta$ al posto di $\{\E\wedge\}\Delta$.
\end{itemize}
\end{lemma}

\begin{proof}
Dimostriamo \ssf{1}$\IMP$\ssf{2}. Fissiamo una tupla $a$ che enumera $\dom k$ e definiamo $p(x,y)=\{\E\wedge\}\Deltatp_M (a,b)$. Da \ssf{1} otteniamo che $p(ka,y)$ \`e finitamente consistente in $N$. Per saturazione esiste $b'\in N$ che realizza $p(ka,y)$. Quindi la mappa $h:M\to N$ con $h=k\cup\{\<b,b'\>\}$ prova \ssf{2}.

L'implicazione \ssf{2}$\IMP$\ssf{3} \`e banale, dimostriamo \ssf{3}$\IMP$\ssf{1}. Ovviamente \ssf{3} implica che $k:M\imp N$ \`e un $\{\wedge\}\Delta$-morfismo. A meno di equivalenza logica, le formule in $\{\E\wedge\}\Delta$ hanno la forma  $\E y \phi (x,y)$ dove $y$ \`e una tupla di variabili e $\phi(x,y)$ una formula in $\{\wedge\}\Delta$. Supponiamo che $M\models\E y\,\phi(a,y)$ e fissiamo un $b$ tale che $M\models\phi(a,b)$ poi, per \ssf{3}, estendiamo $k$ ad un $\Delta$-morfismo $h:M\to N$ definito in $b$. Allora $N\models\phi(ha,hb)$ e quindi $N\models\E y\,\phi(ka,y)$.
\end{proof}


\begin{corollary}\label{corol_EDelta_estensione}
Sia $N$ una struttura $\lambda$-satura e sia $k:M\imp N$ una mappa tale che $|k|<\lambda$ e $|M|\le\lambda$. Allora le seguenti affermazioni sono equivalenti:
\begin{itemize}
\item[1.] $k:M\imp N$ \`e un $\{\E\wedge\}\Delta$-morfismo;
\item[2.] $k:M\imp N$ si estende ad una $\Delta$-immersione;
\item[3.] $k:M\imp N$ si estende ad una $\{\E\wedge\}\Delta$-immersione.
\end{itemize}
\end{corollary}
\begin{proof}
L'implicazione \ssf{1}$\IMP$\ssf{2} si ottiene iterando $|M|\le\lambda$ volte l'estensione in \ssf{2} del lemma~\ref{lem_EDelta_estensione}. L'implicazione \ssf{2}$\IMP$\ssf{3} \`e immediata perch\'e le $\Delta$-immersioni sono automaticamente anche delle $\{\E\wedge\}\Delta$-immersioni, come verificato nella proposizione~\ref{presesis}. L'implicazione \ssf{3}$\IMP$\ssf{1} \`e ovvia.
\end{proof}


\begin{theorem}
Le seguenti affermazioni sono equivalenti:
\begin{itemize}
\item[1.] $\phi(x)$ \`e equivalente ad una formula in $\{\E\wedge\}\Delta$;
\item[2.] ogni $\Delta$-immersione tra modelli di $T$ preserva $\phi(x)$.
\end{itemize}
\end{theorem}
\begin{proof}
L'implicazione \ssf{1}$\IMP$\ssf{2} \`e immediata. Per dimostrare  \ssf{2}$\IMP$\ssf{1} assumiamo $\neg$\ssf{1}. Per il teorema~\ref{qfdefinability} esiste un $\{\E\wedge\}\Delta$-morfismo $k:M\to N$ tra modelli di $T$ che non preserva $\phi(x)$. Possiamo assumere che $N$ sia $\lambda$-saturo con $\lambda$ sufficientemente grande. Quindi per il corollario~\ref{corol_EDelta_estensione} esiste una $\Delta$-immersione $h:M\to N$ che estende $k$. Questa immersione testimonia $\neg$\ssf{2}.
\end{proof}

Ora dimostriamo il duale delle ultime tre proposizioni. Si osservi che, se $\Delta$ contiene la formula $x=y$ ed \`e chiuso per negazione, allora  $k:M\imp N$ \`e un $\Delta$-morfismo se e solo se lo \`e anche $k^{-1}:N\imp M$. In questo caso, le seguenti tre propozioni sono solo una riformulazione delle precedenti.  Il caso generale richiede una dimostrazione indipendente (ma molto simile).

\begin{lemma}\label{lem_ADelta_estensione}
Sia $M$ una struttura $\lambda$-satura e sia $k:M\imp N$ un $\Delta$-morfismo di cardinalit\`a $<\lambda$. Allora le seguenti affermazioni sono equivalenti:
\begin{itemize}
\item[1.] $k:M\imp N$ \`e un $\{\A\vee\}\Delta$-morfismo;
\item[2.] per ogni $c\in N^{<\omega}$ esiste un $\{\A\vee\}\Delta$-morfismo $h:M\imp N$ tale che $k\subseteq h$ e $c\in\range h^{<\omega}$.
\item[3.] come \ssf{2}  ma con $\Delta$ al posto di $\{\A\wedge\}\Delta$.
\end{itemize}
\end{lemma}

\begin{proof}
Dimostriamo \ssf{1}$\IMP$\ssf{2}. Definiamo $p(x,y)=\neg\{\A\vee\}\Deltatp_N (ka,c)$, dove $a$ una tupla che enumera $\dom k$. Da \ssf{1} otteniamo che $p(a,y)$ \`e finitamente consistente in $M$. Infatti, in caso contrario esisterebbe una formula $\phi(z,x)$, congiunzione di formule in $\neg\{\A\vee\}\Delta$, tale che $N\models\phi(ka,c)$ e $M\models\neg\E x\,\phi(a,x)$. Ma $\neg\E x\,\phi(z,x)$ \`e equivalente ad una formula in $\{\A\vee\}\Delta$ e quindi preservata da $k:M\imp N$, contraddizione. Quindi per $\omega$-saturazione esiste $c'\in M$ che realizza $p(a,y)$.  Posto $h=k\cup\{\<c',c\>\}$, la mappa $h:M\to N$ \`e quella richiesta in \ssf{2}.

L'implicazione \ssf{2}$\IMP$\ssf{3} \`e banale, dimostriamo \ssf{3}$\IMP$\ssf{1}.  Ovviamente \ssf{3} implica che $k:M\imp N$ \`e un $\{\vee\}\Delta$-morfismo. Ogni formula in $\{\A\vee\}\Delta$ \`e logicamente equivalente ad una della forma $\A y\,\phi (x,y)$ dove $y$ \`e una tupla di variabili e $\phi(x,y)$ una formula in $\{\vee\}\Delta$. Assumiamo $M\models\A y\,\phi(a,y)$ e sia $c\in N^{|y|}$ arbitrario. Mostriamo che $N\models\phi(ka,c)$. Per \ssf{3}, estendiamo $k$ ad un $\Delta$-morfismo $h:M\to N$ tale che $c=hc'$ per un qualche $c'\in(\dom h)^{|y|}$. Allora da $M\models\phi(a,c')$ otteniamo $N\models\phi(ha,hc')$ e quindi $N\models\phi(ka,c)$.
\end{proof}


\begin{corollary}\label{corol_ADelta_estensione}
Sia $N$ una struttura $\lambda$-satura e sia $k:M\imp N$ una mappa tale che $|k|<\lambda$ e $|M|\le\lambda$. Allora le seguenti affermazioni sono equivalenti:
\begin{itemize}
\item[1.] $k:M\imp N$ \`e un $\{\A\wedge\}\Delta$-morfismo;
\item[2.] $k:M\imp N$ si estende ad un $\Delta$-epimorfismo;
\item[3.] $k:M\imp N$ si estende ad un $\{\E\wedge\}\Delta$-epimorfismo.
\end{itemize}
\end{corollary}
\begin{proof}
L'implicazione \ssf{1}$\IMP$\ssf{2} si ottiene iterando $|M|\le\lambda$ volte l'estensione in \ssf{2} del lemma~\ref{lem_ADelta_estensione}. L'implicazione \ssf{2}$\IMP$\ssf{3} \`e immediata perch\'e i $\Delta$-epimorfismi sono automaticamente anche dei $\{\A\wedge\}\Delta$-epimorfismi, come verificato nella proposizione~\ref{presuniv}. L'implicazione \ssf{3}$\IMP$\ssf{1} \`e ovvia.
\end{proof}



\begin{theorem}
Le seguenti affermazioni sono equivalenti:
\begin{itemize}
\item[1.] $\phi(x)$ \`e equivalente ad una formula in $\{\A\vee\}\Delta$;
\item[2.] ogni $\Delta$-epimorfismo tra modelli di $T$ preserva $\phi(x)$.
\end{itemize}
\end{theorem}
\begin{proof}
L'implicazione \ssf{1}$\IMP$\ssf{2} \`e immediata. Per dimostrare  \ssf{2}$\IMP$\ssf{1} assumiamo $\neg$\ssf{1}. Per il teorema~\ref{qfdefinability} esiste un $\{\A\wedge\}\Delta$-morfismo $k:M\to N$ tra modelli di $T$ che non preserva $\phi(x)$. Possiamo assumere che $N$ sia $\lambda$-saturo con $\lambda$ sufficientemente grande. Quindi per il corollario~\ref{corol_ADelta_estensione} esiste un $\Delta$-epimorfismo $h:M\to N$ che estende $k$. Questo epimorfismo testimonia $\neg$\ssf{2}.
\end{proof}



\begin{exercise}
Per ogni enunciato $\phi$ le seguenti affermazioni sono equivalenti:
\begin{itemize}
\item[1.] $\phi$ \`e equivalente modulo $T$ ad un enunciato in $\{\A\vee\}L_{\rm qf}$\,;
\item[2.] $\phi$ \`e preservato per sottostrutture, ovvero, se $M,N\models T$ e $M\subseteq N\models \phi$ allora $M\models\phi$.\QED
\end{itemize}
\end{exercise}










\section{Un criterio per l'eliminazione dei quantificatori}
\label{eliminazionequantificatoricriterio}

Diremo che \emph{$T$ ammette\/} (oppure \emph{ha\/}) \emph{$\Delta$-eliminazione (positiva) dei quantificatori\/} se per ogni formula $\phi(x)$ in $\{\A\E\mathord\wedge\!\mathord\vee\}\Delta$ esiste una formula $\psi(x)$ in $\{\mathord\wedge\!\mathord\vee\}\Delta$ tale che

\hfil$\displaystyle T\proves\ \phi(x)\iff\psi(x)$.

Diremo semplicemente che $T$ ha eliminazione dei quantificatori, senza menzionare $\Delta$, se questo \`e l'insieme di tutte le formule senza quantificatori. La relativizzazione dell'eliminazione dei quantificatori a insiemi $\Delta$ arbitrari \`e meno comune e la relativa terminologia non \`e standard.

Spesso l'eliminazione dei quantificatori viene usata come primo passo per dimostrare la completezza di una teoria. Vale quindi la pena di mettere in risalto la seguente proposizione che si ottiene dalla definizione prendendo per $x$ la tupla vuota.

\begin{remark}
Se $T$ ha eliminazione dei quantificatori allora le seguenti affermazioni sono equivalenti. 
\begin{itemize}
\item[1.] $T$ \`e completa per gli enunciati senza quantificatori;
\item[2.] $T$ \`e completa.
\end{itemize}
Quindi, una teoria con eliminazione dei quantificatori \`e completa se e solo se tutti i suoi modelli hanno la stessa caratteristica (vedi  corollario~\ref{corollariocaratteristica}).\QED
\end{remark}

% Scriveremo  ${\vee}\hskip-.33ex\Delta$ e ${\wedge}\hskip-.2ex\Delta$ per l'insieme delle formule che sono disgiunzione, rispettivamente congiunzione, di formule in $\Delta$ e scriveremo $\E\Delta$ per le formule della forma $\E y\,\phi(x,y)$ dove $y$ \`e una tupla di variabili e $\phi(x,y)$ \`e in $\Delta$. Se vogliamo limitarci a tuple $y$ di lunghezza $1$ scriveremo $\E_1\Delta$. Infine $\pmDelta$ \`e l'insieme che contiene le formule in $\Delta$ e le loro negazioni. Quindi, a meno di equivalenza logica, le formule in $\veewedge\pmDelta$ sono le combinazioni booleane di formule in $\Delta$ e le formule in $\veewedge\Delta$ le combinazioni booleane positive.

Il seguente corollario \`e una conseguenza del lemma~\ref{qfdefinability}.

\begin{corollary}\label{criterioeq1}
Le seguenti affermazioni sono equivalenti:
\begin{itemize}
\item[1.] $T$ ha $\Delta$-eliminazione dei quantificatori;
\item[2.] ogni $\Delta$-morfismo finito tra modelli di $T$ \`e sia un $\{\E\wedge\}\Delta$ che un  $\{\A\vee\}\Delta$-morfismo.
\end{itemize}
\end{corollary}
\begin{proof}
L'implicazione \ssf{1}$\IMP$\ssf{2} \`e ovvia, dimostriamo \ssf{2}$\IMP$\ssf{1}. Procedendo per induzione sulla sintassi verifichiamo che ogni $\Delta$-morfismo preserva la verit\`a delle formule in $\{\A\E\mathord\wedge\!\mathord\vee\}\Delta$. Per il lemma~\ref{qfdefinability} questo \`e sufficiente per ottenere \ssf{1}. Il passo induttivo per i connettivi $\vee$ e $\wedge$ \`e immediato.  Assumiamo come ipotesi induttiva che la verit\`a di $\phi(x,y)$ sia preservata.  Per il  lemma~\ref{qfdefinability} la formula $\phi(x,y)$ \`e equivalente ad una formula in $\Delta$ e quindi per \ssf{2} anche la verit\`a di $\E y\, \phi (x,y)$ e $\A y \,\phi (x,y)$ \`e preservata.
\end{proof}

La condizione \ssf{2} del corollario~\ref{criterioeq1} \`e difficile da verificare direttamente quindi si ricorre alle condizioni \ssf{3} dei lemmi~\ref{lem_EDelta_estensione} e~\ref{lem_ADelta_estensione}.

\begin{corollary}\label{criterioeq2}
Sia $\lambda$ un qualsiasi cardinale infinito $\ge|L|$. Le seguenti affermazioni sono equivalenti:
\begin{itemize}
\item[1.] $T$ ha $\Delta$-eliminazione dei quantificatori.
\item[2.] per ogni $\Delta$-morfismo finito $k:M\imp N$ tra modelli $\lambda$-saturi di $T$:
\begin{itemize}                                                                                               \item[a.] per ogni $b\in M^{<\omega}$ esiste un $\Delta$-morfismo $h:M\imp N$ tale che $k\subseteq h$ e $b\in\dom h^{<\omega}$.                                                                                            \item[b.]per ogni $c\,\in\,N^{<\omega}$ esiste un $\Delta$-morfismo $h:M\imp N$ tale che $k\subseteq h$ e $c\in\range h^{<\omega}$.\QED                                                                                          \end{itemize}
\end{itemize}
\end{corollary}

Si osservi che, se $\Delta$ contiene la formula $x=y$ ed \`e chiuso per negazione, allora  $k:M\imp N$ \`e un $\Delta$-morfismo se e solo se lo \`e anche $k^{-1}:N\imp M$. In questo caso \ssf{a} e \ssf{b} sono equivalenti.



\begin{exercise}\label{EQ_stazi_vettoriali}
Sia $R$ un campo e sia $T_\Rmod$ la teoria degli $R\jj$moduli (ovvero spazi vettoriali su $R$) definita nel paragrafo~\ref{gruppi}. Si dia una dimostrazione diretta dell'eliminazione dei quantificatori in $T_\Rmod$. Suggerimento: dei modelli saturi di $T_\Rmod$ \`e sufficientemente dimostrare che hanno dimensione $>|R|$.
\end{exercise}


\begin{exercise}
Il linguaggio \`e quello degli ordini stretti e la notazione \`e quella del paragrafo~\ref{ordinilinearidensi}. Sia $T$ la teoria assiomatizzata da $T_{\rm ol}$ pi\`u la seguente coppia di assiomi:
\begin{itemize}
\item[dis$\uparrow$.] $\E z\ \big[x<z\ \wedge\ \neg\E y\ x<y<z\big]$;
\item[dis$\downarrow$.] $\E z\ \big[ z<x\ \wedge\ \neg\E y\ z<y<x\big]$.
\end{itemize}
Osserviamo che $T$ \`e la stessa teoria dell'esercizio~\ref{ex_QQxZZ_saturo} espressa in un linguaggio diverso. Sia $\Delta$ un insieme di formule chiuso per negazione, contenente le formule atomiche e quelle della forma $\E^{=n}y\ (x\mathord<y\mathord<z)$. Si dimostri che $T$ ha $\Delta$-eliminazione dei quantificatori. Si dimostri che $T$ ha modelli saturi numerabili.\QED
\end{exercise}


\begin{exercise}
Sia $T$ una teoria completa in un linguaggio che consiste solo di simboli di relazioni unarie. Si dimostri che $T$ ha eliminazione dei quantificatori.\QED
\end{exercise}

\begin{comment}


\section{La model-completezza}

Per ragioni storiche introduciamo la seguente nozione. 

Diremo che $T$ \`e $\Delta$-modello completa se ogni $\Delta$-morfismo $h:M\to N$ totale tra modelli di $T$ \`e un $\{\A\E\mathord\wedge\!\mathord\vee\}\Delta$-morfismo. 

Diremo che $T$ \`e $\Delta$-modello completa se ogni $\Delta$-morfismo $h:M\to N$ suriettivo tra modelli di $T$ \`e un $\{\A\E\mathord\wedge\!\mathord\vee\}\Delta$-morfismo. 




 (la terminologia \`e bizzarra Abraham Robinson.)


%%%%%%%%%%%%%%%%%%%%%
%%%%%%%%%%%%%%%%%%%%
%%%%%%%%%%%%%%%%%%%%%%%
%%%%%%%%%%%%%%%%%%%%%%%
\section{Un esempio}

Il seguente esempio ha il solo scopo di applicare in un caso molto semplice quando visto nel paragrafo~\ref{eliminazionequantificatoricriterio}. Vale un risultato pi\`u forte e pi\`u generale di quanto affermato dal teorema~\ref{eliminazionearitmeticaadditiva}. Questo infatti \`e un semplice corollario del teorema di Bauer-Monk sull'eliminazione dei quantificatori per i moduli (che qui non dimostreremo).

Il linguaggio, che qui denoteremo con $L_{\rm ga\small+1}$, estende quello dei gruppi additivi con la costante $1$. 

Per ogni intero $n$ scriveremo $\delta_n(x)$ per $\E y\ ny=x$. La formula $\delta_0(x)$ verr\`a letta come $x=0$.  Per il tutto questo paragrafo $\Delta$ denota l'insieme che contiene tutte le formule $\delta_n(x)$, con $n$ non negativo, ed \`e chiuso per sostituzione di variabili con termini puri. Osserviamo che per $\delta_0(x)$ \`e equivalente alla formula $x=0$ quindi $\Delta$ contiene a meno di equivalenza tutte le formule atomiche. Definiamo ora la teoria $T_{\rm 1}$. Questa contiene gli assiomi dei gruppi additivi e
\begin{itemize}
\item[1.] $\neg\delta_n(m)$ per ogni coppia di interi non negativi $m,n$ tali che $n\nmid m$;
\item[2.] $\displaystyle\A x\ \bigvee^{n-1}_{r=0}\delta_{n}(x+r)$ \ \ per ogni coppia di interi $m>0$ ed $n>1$.
\end{itemize}

Tra gli assiomi in \ssf{1} ci sono le formule $\neg\delta_0(n)$ per ogni $n$ positivo, i modelli di $T_1$ sono gruppi senza torsione e contengono tutti una copia di $\ZZ$.

\begin{lemma}
Sia $\Delta$ come sopra. La teoria $T_{\rm 1}$ \`e completa per gli enunciati in $\Delta$.
\end{lemma}

\begin{proof}
Gli enunciati in $\Delta$ hanno la forma $\delta_n(m)$. Il lemma dice che per ogni $M\models T_{\rm 1}$

\hfil $M\models\delta_n(m)\ \ \IFF\ \ \Z\models\delta_n(m)$.

la direzione $\PMI$ \`e ovvia perch\'e $\Z\subseteq M$. La direzione opposta \`e garantita dallo schema di assiomi \ssf{1}.
\end{proof}

\begin{theorem}\label{eliminazionearitmeticaadditiva} 
La teoria $T_1$ ammette $\Delta$-eliminazione dei quantificatori.
\end{theorem}

\begin{proof}
Fissiamo $M$ ed $N$ due modelli $\omega$-saturi di $T_{\rm 1}$ e sia $k:M\to N$  e sia $b$ un elemento di $M$. Verifichiamo che vale la condizione \ssf{2a} del corollario~\ref{criterioeq2}. Sia $a$ una enumerazione di $\dom k$ e poniamo $p(z,x)=\Deltatp_M(a,b)$, vogliamo mostrare che $p(c,x)$ \`e realizzato in $N$.

Prima di procedere osserviamo che dagli assiomi \ssf{1} e \ssf{2} segue 

\ssf{3.}\hfil $\displaystyle\neg\delta_n(x+r)\ \iff\ \bigvee^{n-1}_{r\neq s=0}\delta_n(x+s)$

Le formule in $p(z,x)$ hanno la forma $\delta_n\big(mx+t(z)\big)$ per un intero $m$ ed un termine $t(z)$. Modulo 

$T_1\cup\Th_\Delta(M/a)\proves q(x)\iff p(a,x)$


$T_1\cup\Th_\Delta(N/ka)\proves q(x)\iff p(ka,x)$


La seguente formula vale in tutti i modelli di $T_{\rm 1}$, segue semplicemente dagli assiomi dei gruppi abeliani,

\hfil$\delta_n(t(z)-r)\ \imp\ \A x\,\Big[\delta_{n}\big(mx+t(z)\big)\ \iff\ \delta_{n}\big(mx+r\big)\Big]$.


Per \ssf{2.} sappiamo che per ogni termine $t(z)$ esiste un intero $0\le r<n$ tale che $M\models\delta_{n}(t(a)+r)$. 




Quindi nelle formule che occorrono in $p(z,x)$ possiamo sostituire i termini $t(z)$ con le opportune costanti $r$ e ottenere un tipo $q(x)$ che in $M$ \`e equivalente $p(a,x)$ ed in $N$ \`e equivalente a $p(c,x)$. 

Per verificare che $p(c,x)$ \`e finitamente consistente in $N$ \`e sufficiente verificare che  $q(x)$ \`e finitamente consistente. Mostriamo che tutte le congiunzione di formule in $q(x)$ sono consistenti in $\Z$ e quindi, poich\'e sono formule esistenziali anche in $N$. Supponiamo per assurdo che esista una congiunzione di formule in $q(x)$ che non \`e soddisfatta in $\Z$, quindi che in $\Z$ valga 

\hfil$\displaystyle\neg\E x\ \bigwedge_{i\in I}\delta_{n_i}(m_ix+r_i)$.

Semplicemente per de Morgan questa formula \`e equivalente a

\hfil$\displaystyle\A x\ \bigvee_{i\in I}\neg\delta_{n_i}(m_ix+r_i)$

Quindi, sostituendo le negazioni con disgiunzioni come in \ssf{4}, ci riduciamo ad una formula della forma

\hfil$\displaystyle\A x\ \bigvee_{j\in J}\delta_{n_j}(m_jx+r_j)$

Questa, se vera in $\Z$, \`e un'istanza dello schema di assiomi \ssf{2}, quindi \`e vera in $M$. Contraddizione.
\end{proof}


\end{comment}


%%%%%%%%%%%%%%%%%%%%%
%%%%%%%%%%%%%%%%%%%%
%%%%%%%%%%%%%%%%%%%%%%%
%%%%%%%%%%%%%%%%%%%%%%%
%\section{Esempi: L'aritmetica di Pre\ss burger}


 
%\end{comment}