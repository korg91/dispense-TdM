\chapter{La teoria degli insiemi}\label{insiemi}


\section{Gli assiomi di Zermelo-Fraenkel}

Il linguaggio $L$ contiene solo il simbolo di relazione binaria $\in$. 

\begin{itemize}
\item[a1]$\E x\ x = x\phantom{\big]}$\hfill\emph{Esistenza}

\item[a2]$\A z\ \big[z\in x \iff z\in y\big]\ \imp\ x=y$\hfill \emph{Estensionalit\`a}
\end{itemize}

(Qui e nel seguito per chiarezza scriveremo delle formule, gli assiomi sono la chiusura universale di queste.) Gli elementi di una struttura $\VV$ di segnatura $L$ che soddisfa (almeno) a questi due assiomi verranno chiamati \emph{insiemi}. C'\`e quindi una omonimia fastidiosa: abbiamo la stessa parola per sottoinsiemi di $\VV$ e gli elementi di $\VV$. Quando \`e importante distinguere cercheremo di usare la parola \emph{classe\/} per indicare un sottoinsieme di $\VV$. Diremo quindi che una classe \`e definibile se \`e della forma $\phi(\VV)$ per qualche formula $\phi(x)$. Ad ogni insieme $a$ associamo la classe definibile che chiameremo la sua \emph{estensione}: l'estensione di $a\in \VV$ \`e la classe definita dalla formula $x\in a$, dove $a$ \`e un parametro.  Spesso confonderemo gli insiemi con le rispettive estensioni: dato l'assioma di estensionalit\`a, i rischi di ambiguit\`a sono limitati. 

Poich\'e ad ogni insieme corrisponde una classe definibile, saremmo tentati di assiomatizzare il viceversa: ad ogni classe definibile corrisponde un insieme. Questo si chiama assioma di \emph{comprensione (illimitata)}:

\begin{itemize}
\item[{\blue\Large$\lightning$}]$\E x \A z\ \big[ z\in x \iff \phi(z,y)\big]$.\hfill\emph{Comprensione} 
\end{itemize}

Purtroppo, cos\`i com'\`e l'assioma \`e contradditorio! Prima di tutto dobbiamo escludere che $x$ occorra in $\phi(z,y)$ altrimenti sostituendo $z\notin x$ a $\phi(z,y)$ otteniamo un assurdo. Ma con questo i problemi non sono finiti. Se per $\phi(z,y)$ prendiamo la formula $z\notin z$ otteniamo la famosa antinomia di Russel (scoperta indipendentemente anche da Zermelo). 

Gli assiomi che elenchiamo in questo paragrafo sono tutti versioni ristrette dell'assioma di comprensione illimitata. Ristrette proprio per evitare antinomie. Il primo esempio \`e il seguente:

\begin{itemize}
\item[a3]$\E x \A z\ \big[ z\in x \iff z\in u\wedge\phi(z,y)\big]$, dove $x$ non occorre in $\phi(z,y)$.\hfill\emph{Separazione} 
\end{itemize}

Questo assioma afferma l'esistenza dell'insieme che ha per estensione la classe usualmente denotata con $\{z\in u : \phi(z,y)\}$. In \ssf{a3} la variabile $y$ sta semplicemente per un arbitrario parametro, immaginiamola fissata. La variabile $z$ corre sugli elementi che costituiscono $x$, l'insieme $u$ gioca il ruolo (fondamentale) di un limite all'interno del quale gli elementi di $x$ vanno cercati. Nell'assioma di comprensione illimitata non c'\`e limite e gli elementi di $x$ andavano cercati in tutto $\VV$. Possiamo riformulare \ssf{a3} cos\`i: se $A$ \`e una classe definibile e $b$ \`e un insieme tale che $A\subseteq b$ allora $A$ \`e (l'estensione di) un insieme.

Notiamo che \ssf{a3} \`e uno \textit{schema di assiomi\/} cio\`e si consiste di infiniti enunciati al variare di $\phi$ tra tutte le formule del linguaggio della teoria degli insiemi in cui $x$ non occorre. Comunque, per facilit\`a di linguaggio, parleremo semplicemente di \textit{assioma di separazione}. Si osservi che una precisa definizione di formula del linguaggio del prim'ordine \`e un requisito indispensabile per poter assiomatizzare rigorosamente la teoria degli insiemi.

Da \ssf{a1} e separazione otteniamo l'esistenza dell'insieme vuoto che denoteremo con $\0$. Infatti dato un qualsiasi insieme $a$, che esiste per \ssf{a1}, otteniamo per separazione $\0=\{z\in a: z\neq z\}$.


Fin qui l'insieme vuoto \`e l'unico insieme la cui esistenza \`e garantita (infatti $\VV=\{\0\}$ \`e un modello di \ssf{a1}-\ssf{a3}). Per poterci costruire almeno tutti gli insiemi finiti introduciamo l'\textit{assioma della coppia\/} che afferma l'esistenza dell'insieme $\{u,v\}$ dati due qualsiasi insiemi $u$ e $v$:

\begin{itemize}
\item[a4]$\E x \ \big[z\in x \iff  z=u\vee z=v\big]$\hfill \emph{Coppia}
\end{itemize}

Il seguente \`e detto \textit{assioma dell'unione}:
\begin{itemize}
\item[a5]$\E x \A z\ \Big[z\in x \iff \E y\ z\in y\in u\Big]$\hfill \emph{Unione}
\end{itemize}

Dato $u$, l'assioma afferma l'esistenza dell'insieme che qui denoteremo con $\bigcup u$ e che viene anche denotato con 

\hfil$\displaystyle \bigcup_{y\in u} y$.

L'assioma dell'unione, assieme a quello della coppia, garantisce l'esistenza di $a\cup b=\bigcup\{a,b\}$.

Gli assiomi elencati fin qui non garantiscono l'esistenza di un insieme infinito. Introduciamo queindi il seguente \textit{assioma dell'infinito}:

\begin{itemize}
\item[a7]$\E x\ \Big[\0\in x \wedge \A z\ \big[z\in x \imp z\cup\{z\}\in x\big]\Big]$\hfill\emph{Infinito} 
\end{itemize}

Quest'assioma richiede l'esistenza di un insieme che contiene egli elementi:

\hfil $\0,\ \  \{\0\},\ \  \big\{\0,\, \{\0\}\big\},\ \ \Big\{\0,\ \big\{\0\},\ \{\0,\, \{\0\}\big\}\Big\},\   \dots\dots$

Verifichiamo che siano elementi effettivamente diversi tra loro. Conviene prima semplificare la notazione. Denotiamo \textit{temporaneamente\/} l'insieme $\0$ con $\widehat{0}$ e con $\widehat{n+1}$ denoteremo $\widehat{n}\cup\{\widehat{n}\}$. Con questa notazione gli elementi elencati qui sopra sono $\widehat{0},\,\widehat{1},\,\widehat{2},\,\,\widehat{3},\,\dots$. %\`E facile verificare per induzione che $n<m$ se e solo se $\widehat{n}\in\widehat{m}$ se e solo se  $\widehat{n}\subset\widehat{m}$.

\begin{remark}
Le seguenti afferamzioni sono equivalenti:
\begin{itemize}
\item[1.] $n<m$;
\item[2.] $\widehat{n}\subseteq\widehat{m}$;
\item[3.] $\widehat{n}\in\widehat{m}$.
\end{itemize}
\end{remark}
\begin{proof}
L'implicazione \ssf{1}$\IMP$\ssf{2} si dimostra per induzione su $m$ e da questa segue immediatamente and l'implicazione \ssf{1}$\IMP$\ssf{3}. Mostramo ora  \ssf{3}$\IMP$\ssf{1}. Per induzione su $m$ verifichiamo che $x\in\widehat{m}$ allora $x=\widehat{n}$ per un qualche $n<m$. Quando $m=0$ l'affermazione \`e vuota, se $x\in \widehat{m+1}$ allora $x=\widehat{m}$ oppure $x\in\widehat{m}$ e l'affermazione segue per ipotesi induttiva. 

L'implicazione \ssf{2}$\IMP$\ssf{3} \`e immediata per 
\end{proof}




Per induzione su $m$ \`e immediato che  $n\le m$ implica $\widehat{n}\subseteq\widehat{m}$ e quindi che $n<m$ implica $\widehat{n}\in\widehat{m}$. Verifichiamo il viceversa:

%Quindi $n<m$ e 
% 
% Ora verificheamo che $\widehat{n}\in\widehat{m}$ implica $n<m$. .  
% 
% 
% Ora invece per induzione su $n$ verifichiamo che se $n< m$ allora  $\widehat{n}\neq\widehat{m}$. Se $n=0$ l'affermazione vale perch\'e quando $m$ \`e positivo $\widehat{m}$ non \`e mai vuoto. Assumiamo l'affermazione vera per $n$ e dimostriamola per $n+1$. 
% 
% Se $n+1<m$ per ipotesi induttiva $\widehat{n}\neq\widehat{m}$. Ma $\widehat{n}\in\widehat{n+1}$
% 
% Poich\'e $\widehat{n}\in\widehat{n+1}$ e per ipotesi induttiva nessuno degli elementi di $\widehat{n}$ \`e uguale a $\widehat{n}$. 

In teoria degli insiemi si usa confondere $\widehat{n}$ con $n$. La notazione $\widehat{n}$ non verr\`a pi\`u usata e le espressioni $n<m$, $n\subset m$, e $n\in m$ verranno considerate equivalenti.% Scriveremo  \emph{$\omega$} per $\{0,1,\dots\}$, per il momento semplicemente una classe, servir\`a un po' di lavoro mostrare che \`e un insieme.

La combinazione dell'assioma dell'infinito con il seguente \textit{assioma della potenza\/} genera un universo di insiemi sufficiente mente ricco da poterci formalizzare gran parte della matematica. Si legga l'espressione \emph{$z\subseteq u$\/} come un abbreviazione di $\A w\,\big[w\in z\imp w\in u\big]$.

\begin{itemize}
\item[a8]$\E x \A z\ \Big[z\in x \iff z\subseteq u\Big]$\hfill\emph{Insieme potenza} 
\end{itemize}

Sebbene non sia essenziale si preferisce escludere l'esistenza di insiemi appartenenti a se stessi e pi\`u in generale di catene $a_0\in\dots\in a_n$ in cui $a_0=a_n$. In generale si vorrebbero anche escludere catene infinite discendenti $a_0\ni a_1\ni\dots $. Tali catene contrastano con la nostra intuizione di un universo insiemistico costruito dal basso (cio\`e dall'insieme vuoto) verso l'alto. Purtroppo quest'ultima richiesta \`e troppo ottimista (si pensi ai modelli non standard dell'aritmetica). Comunque, col seguente \textit{assioma di regolarit\`a\/} riusciremo ad escludere l'esistenza di catene infinite discendenti \textit{definibili}. 

Scriviamo $u\cap x$ per l'insieme $\{z\in u: z\in x\}$ la cui esistenza \`e garantita dall'assioma di separazione.

\begin{itemize}
\item[a9]$\A u\neq\0\ \E x\in u\ \ u\cap x=\0$.\hfill\emph{Regolarit\`a} 
\end{itemize}

Per mostrare la relazione tra questo assioma e le catene discendenti, supponiamo esista un $a_0$ che invalidi l'assioma di regolarit\`a. Sia $a_1$ un qualsiasi elemento di $a_0$. Poich\`e $a_1\cap a_0\neq\0$ possiamo trovare $a_2\in a_1$ che \`e anche in $a_0$ e come tale $a_2\cap a_0\neq\0$. Quindi troveremo $a_3\in a_2\in a_1\in a_0$, ed iterando, una catena infinita discendente. 

Come detto, l'assioma \ssf{a9} equivale all'escludere l'esistenza di catene infinite discendenti  \textit{definibili}, ma la dimostrazione viene posticipata per avere il tempo di introdurre la nozione di catena definibile.

Fin qui gli assiomi proposti da Zermelo (a quel tempo non ancora formalizzati nella logica del prim'ordine). Con questa teoria \`e possibile formalizzare una buona parte della matematica elementare. Argomenti pi\`u sofisticati richiedono l'\textit{assioma del rimpiazzamento\/} dovuto a Fraenkel. 


\begin{itemize}
\item[a10]$\A y\in u\,\E^{= 1} z\,\phi\imp\E x \A z\,\big[z\in x\iff \E y\in u\;\phi\big]$, dove $x$ non occorre in $\phi$\hfill\emph{Rimpiazzamento} 
\end{itemize}

Il rimpiazzamente \`e un assioma (per la precisione anche questo \`e uno schema) meno semplice da leggere: l'antecedente afferma che $\phi(y,z)$ \`e il grafo di una funzione con dominio $u$. L'insieme $x$ di cui l'assioma afferma che l'immagine di $u$ secondo questa funzione \`e un insieme. %Alcuni testi enunciano l'assioma di rimpiazzamento in versione pi\`u forte del nostro \ssf{a4}. La versione pi\`u forte ha il vantaggio di congiungere in un unico schema la separazione e il rimpiazzamento. Si veda l'esercizio~\ref{ex_sep-rimpiazzamento}. 

\begin{exercise}
Si dimostri che rimpiazzamento e separazione possono essere sostituiti da quest'unico schema:
\begin{itemize}
\item[]$\A y\in u\E^{\le 1} z\,\phi(y,z)\imp\E x \A z\,\big[z\in x\iff \E y\in u\;\phi(y,z)\big]$, dove $x$ non occorre in $\phi(y,z)$
\end{itemize}
\end{exercise}

\begin{exercise}
Si dimostri che l'assioma della coppia segue dai rimanenti. Suggerimento: si usi l'assioma delle parti per costruire un insieme con due elementi, poi il rimpiazzamento.
\end{exercise}


%%%%%%%%%%%%%%%%%%%%%%%
%%%%%%%%%%%%%%%%%%%%%%%
%%%%%%%%%%%%%%%%%%%%%%%
%%%%%%%%%%%%%%%%%%%%%%%
%%%%%%%%%%%%%%%%%%%%%%%
\section{Gli ordinali}
\def\ord{{\rm ord}}
\def\Ord{{\rm Ord}}
%\def\tran{{\rm tran}}
\def\lim{{\rm lim}}
\def\rel{{\rm rel}}
\def\fun{{\rm fun}}

Sia $A$ una classe definibile (in particolare, un insieme). Diremo che $A$ \`e una classe \emph{transitiva\/} se $\A x\in A\ [x\subseteq A]$ o, equivalentemente, se $\bigcup A\subseteq A$.

Diremo che un insieme $\alpha$ \`e un \emph{ordinale\/} se 
\begin{itemize}
\item[o1.] $\alpha$ \`e un insieme transitivo;
\item[o2.] la relazione $\in$ ristretta ad $\alpha$ \`e antisimmetrica: $\A x,y\in\alpha \big[x=y\ \vee\ y\in x\ \vee\ x\in y\big]$.
\end{itemize}
Scriveremo \emph{Ord\/} per la classe di tutti gli ordinali. Mostreremo che $\Ord$ \`e una classe propria. Di solito gli ordinali vengono indicati con le lettere greche, e scriveremo $\A\alpha$ e $\E\alpha$ per $\A x\in\Ord$ ed $\E x\in\Ord$. 

Se $u$ \`e un  insieme scriveremo \emph{$u+1$\/} per l'insieme $u\cup\{u\}$. Se $A$ \`e una classe definibile, diremo che $u\in A$ \`e un \emph{elemento $\in$-minimale} se $A\cap u=\0$. L'assioma di regolarit\`a dice esattamente che ogni insieme non vuoto ha elementi minimali. %\`E immediato verificare che questo vale anche per le classi definibili: un elemento $\in$-minimale di $A$ si trova prendendo un qualsiasi $u\in A$ e, se $A\cap u\neq\0$, fissare un elemento $\in$-minimale $v\in A\cap u$. Poich\'e $v\cap A\cap u=\0$, a maggior ragione $v\cap A$

\begin{proposition}\label{prop_3}
Se $\alpha$ e $\beta$ sono ordinali allora
\begin{itemize}
\item[1.] $0\ \deq\ \0$ \`e un ordinale;
\item[2.] $\alpha+1$ \`e un ordinale (in particolare, gli insiemi $n$ definiti sopra sono ordinali);
\item[3.] se $b\in\alpha$ allora $b$ \`e un ordinale;
\item[4.] se $\gamma$ \`e un elemento $\in$-minimale di $\alpha\sm\beta$ allora $\gamma=\beta$.
%\item[5.] se $\beta\subseteq\alpha$ allora $\beta\in\alpha$ oppure $\beta=\alpha$.
\end{itemize}
\end{proposition}
\begin{proof} La prima affermazione \`e banale. Per le altre, si fissi un ordinale $\alpha$ arbitrario.\nobreak
\begin{itemize}
\item[2.] Verifichiamo la transitivit\`a. Sia $b\in\alpha+1$. Consideriamo due casi $b=\alpha$ oppure $b\in\alpha$. Nel primo caso $b\subseteq\alpha+1$ \`e ovvio nel secondo caso $b\subseteq\alpha\subseteq\alpha+1$ per la trasitivit\`a di $\alpha$. La verifica di \ssf{o2} \`e analoga: si considerino quattro casi e si usi che \ssf{o2} vale per $\alpha$. 
\item[3.] \`E sufficiente verificare la transitivit\`a, poi da questa segue immediatamente \ssf{o2}. Fissiamo $d\in c\in b$ e dimostriamo che $d\in b$. Sappiamo che $d\in \alpha$, per trasitivit\`a di quest'ultimo. Quindi se per assurdo $d\notin b$ per \ssf{o2} avremmo che $d=b$ oppure $b\in d$. Entrambe queste possibilit\`a sono escluse dall'assioma di regolarit\`a.
\item[4.] L'inclusione $\gamma\subseteq\beta$ segue immediatamente dalla transitivit\`a di $\beta$. Per l'inclusione opposta, fissiamo $\delta\in\beta$ e mostriamo che $\delta\in\gamma$. Se cos\`i non fosse, per \ssf{o2} avremmo che  $\delta=\gamma$ oppure $\gamma\in\delta$. Dalla prima otteniamo $\gamma\in\gamma$ che contraddice la regolarit\`a. Dalla seconda otteniamo per traansitivit\`a $\gamma\in\beta$ che contraddice la scelta di $\gamma$.%
%\item[5.] Supponiamo $\alpha\neq\beta$. Per l'assioma di regolarit\`a esiste $\gamma\in\alpha\sm\beta$ tale che $\gamma\cap(\alpha\sm\beta)=\0$. Quindi $\gamma$ \`e il minimo elemento di $\alpha\sm\beta$ allora $\gamma=\beta$ per il punto \ssf{4}.
\end{itemize}\vskip-\baselineskip\vskip-\parskip
\end{proof}

Notiamo il seguente corollario (segue immediatmente dal punto \ssf{3} della proposizione~\ref{prop_3}).

\begin{corollary}
La classe $\Ord$ \`e transitiva.\hfill\qed
\end{corollary}


Ricordiamo che un \emph{ordine lineare (stretto)\/} \`e  una relazione irriflessiva, transitiva, e antisimmetrica (si dice anche \emph{ordine totale\/}). Un ordine lineare si dice un \emph{buon ordinamento\/} se ogni suo sottoinsieme non vuoto ha un elemento minimale. La nozione di buon ordinamento si applica naturalmente anche a classi ma \`e bene insistere che l'esistenza dell'elemento minimale \`e richiesta solo per i sotto-\textit{insiemi\/} della classe. Nulla \`e specificato per le sotto-\textit{classi}.   

\begin{proposition}\label{prop_Ord_wo}
La relazione $\in$ ristretta alla classe $\Ord$ \`e un buon ordinamento. (Quindi anche se ristretta ad un qualsiasi ordinale.)
\end{proposition}

\begin{proof}
L'affermazione tra parentesi segue dalla prima perch\`e se $\alpha\in\Ord$ dal punto \ssf{3} della proposizione~\ref{prop_3} segue $\alpha\subseteq\Ord$. \`E immediato verificare che la relazione di appartenenza ristretta ad $\Ord$ \`e irriflessiva e transitiva. Mostriamo prima che \`e un buon ordinamento e poi che \`e antisimmetrica. Sia $u\subseteq\Ord$ non vuoto e fissiamo un arbitrario $\alpha\in u$. Se $\alpha\cap u$ allora $\alpha$ \`e proprio il minimo elemento di $u$ cercato. Altrimenti sia $\beta\in\alpha\cap u$ tale che  $\beta\cap\alpha\cap u=\0$. Un tale $\beta$ esiste per regolarit\`a. Ora da $\beta\in\alpha$ otteniamo $\beta\subseteq\alpha$ e quindi $\beta\cap u=\beta\cap\alpha\cap u=\0$. Quindi $\beta$ \`e il minimo elemento di $u$ cercato.  

Ora dimostriamo l'assimetria. Supponiamo $\alpha\neq\beta$. Per simmetria possiamo anche supporre $\alpha\not\subseteq\beta$. Sia $\gamma\in\alpha\sm\beta$ il minimo. Dal punto \ssf{4}  della proposizione~\ref{prop_3} otteniamo $\gamma=\beta$ quindi $\beta\in\alpha$. 
\end{proof}

Se $\alpha,\beta\in\Ord$ scriveremo anche \emph{$\alpha<\beta$\/} per $\alpha\in\beta$.

La seguente \`e una formulazione equivalente della proposizione~\ref{prop_Ord_wo} che \`e nota come \emph{principio di induzione transfinita}:

\begin{proposition}
Per ogni formula $\phi(x)$ valono le seguenti formule
\begin{itemize}
\item[1.]$\A\beta\; \big[\A x<\beta\ \phi(x)\,\imp\,\phi(\beta)\big]\ \imp\ \A\alpha\,\phi(\alpha)$;
\item[2.]$\E\alpha\;\phi(\alpha)\imp \E\alpha\,\big[\phi(\alpha)\,\wedge\,\A\beta<\alpha\neg\phi(\beta)\big]$.
\end{itemize}
\end{proposition}

\begin{proof}
Si noti che sostituendo $\neg\phi(x)$ per $\phi(x)$ in \ssf{2} ottteniamo una formula equivalente ad \ssf{1}. Quindi \`e sufficente dimostrare \ssf{2}. Assumiamo l'antecedente in \ssf{2}. Sia $\gamma$ tale che $\phi(\gamma)$ e sia $\alpha$ il minimo elemento di  $\{\beta\in\gamma : \phi(\beta)\}$. Questo $\alpha$ testimonia la verit\`a della coseguenza di \ssf{2}.
\end{proof}

Il seguente \`e un utile corollario. La dimostrazione \`e immediata.

\begin{corollary}\label{prop_transitive_subset_Ord}
Se $b\subseteq\Ord$ \`e un insieme transitivo allora $b$ \`e un ordinale.\QED
\end{corollary}

\section{La chiusura transitiva}

\def\ceq#1#2{\parbox{15ex}{\hfill#1}\parbox{8ex}{\hfil#2}}


Vediamo un'importante conseguenza dell'assioma dell'infinito e del rimpiazzamento.

Scriveremo \emph{lim($x)$\/} per $\A y\in x\ [y+1\in x]$. A parole, diremo che $x$ \`e \emph{limite}. La notazione si usa tipicamente per ordinali, noi la generalizziamo per convenienza.). Osserviamo che l'assioma dell'infinito dice esattamente $\E x\,\lim(x)$. Definiamo

\hfil \emph{$\omega$}$\ \ \deq\ \ \{x\in\Ord: \neg\lim(x)\wedge\A y\in x\neg\lim(y)\}$. 

A prima vista possiamo solo dire che si tratta di una classe definibile, la seguente proposizione dimostra in particolare che si tratta di un insieme.

\begin{proposition}
$\omega$ \`e un un ordinale.
\end{proposition}

\begin{proof}
\`E immediato verificare che $\omega$ \`e una classe transitiva, quindi se dimostriamo che \`e un insieme per la proposizione~\ref{prop_transitive_subset_Ord} possiamo concludere che \`e un ordinale. 

Dimostraimo che $\omega$ \`e un insieme. Per l'assioma dell'infinito esiste un insieme $a$ tale che $\lim(a)$. \`E sufficiente verificare che $\omega\subseteq a$. Supponiamo per assurdo che esista $x\in\omega\sm a$ e sia $\alpha$ il minimo di questi. Poich\'e $\alpha\in\omega$ esiste un $\beta$ tale che $\beta+1=\alpha$. Per la transitivit\`a di $\omega$ abbiamo $\beta\in\omega$. Quindi per la minimalit\`a di $\alpha$ concludiamo che $\beta\in \omega\cap\alpha$. Ma allora anche $\beta+1$ appartiene ad $a$, contraddizione.
\end{proof}

Una \emph{coppia ordinata\/} \`e un insieme della forma $\big\{a,\{a,b\}\big\}$ che abbrevieremo come di consueto con $(a,b)$. Una \emph{relazione binaria\/} \`e un insieme di coppie. Una \emph{funzione\/} \`e una relazione binaria univoca. \`E immediato verificare che esistono formule che definiscono queste classi di insiemi:

\ceq{$w=\{x,y\}$}{$\dIFF$}$\A z\in w\ \big[z=x\;\vee\; z=y\big]$

\ceq{$a=(x,y)$}{$\dIFF$} $\E w\ \big[\,w=\{x,y\}\;\wedge\; a=\{x,w\}\,\big]$

\ceq{$\rel(r)$}{$\dIFF$}$\A w\in r\ \E x,y\ w=(x,y)$\hfill $r$ \`e una relazione

\ceq{$(x,y)\in r$}{$\dIFF$} $\E w\ \big[\,w=(x,y)\;\wedge\; w\in r\,\big]$

\ceq{$x\in\dom(r)$}{$\dIFF$}$\rel(r) \wedge \E y\ (x,y)\in r$\hfill $x$ appartiene al dominio (di definizione) di $r$

\ceq{$\fun(f)$}{$\IFF$}$\rel(r) \wedge \A x\ \E^{\le 1} y\ (x,y)\in f$\hfill $f$ \`e una funzione


Finalmente arriviamo alla definizione che \`e a titolo di questo paragrafo: se $A$ \`e una classe definibile, la \emph{chiusura transitiva\/} di $A$ \`e la classe
\def\TC{{\rm TC}}

\ceq{$x\in\TC(A)$}{$\dIFF$} $\E n\in\omega\ \E f\ \Big[\dom(f)=n+1\;\wedge\; f(0)\in A\;\wedge\;f(n)=x\;\wedge\;\A i\in n\ f(i+1)\in f(i)\Big]$.


\ceq{$\phi(f,n,a)$}{$\dIFF$} $n\in\omega\;\wedge\;\dom(f)=n+1\;\wedge\; f(0)= a\;\wedge\;\A i\in n\ f(i+1)=\bigcup f(i)$.


 

