%%%%%%%%%%%%%%%%%%%%%%%%%%%%%%%%%%%%
%%%%%%%%%%%%%%%%%%%%%%%%%%%%%%%%%%%%%
%%%%%%%%%%%%%%%%%%%%%%%%%%%%%%%%%%%%
\chapter{Modelli numerabili}\label{numerabili}
% %\setcounter{page}{1}


\def\ceq#1#2#3{\parbox{15ex}{$\displaystyle #1$}\parbox{6ex}{\hfil$\displaystyle #2$}$\displaystyle  #3$}


Il primo paragrafo \`e dedicato al teorema di omissione dei tipi. Un risultato classico fondamentale per lo studio dei modelli numerabili ed in particolare delle teorie $\omega\jj$categoriche. La dimostrazione \`e breve ma un po' criptica, tanto da ispirare un celebre aforisma di Gerard Sacks: \textit{any fool can realize a type, but it takes a model-theorist to omit one}.

Di nuovo, non prenderemo la via pi\`u breve per dimostrare il teorema (il teorema di omissione dei tipi prende una pagina del libro di Tent-Ziegler). Invece presenteremo una dimostrazione alternativa che dovrebbe convincere il lettore che si tratta di un risultato pi\`u vicino alla teoria descrittiva degli insiemi che alla teoria dei modelli. Un'altra differenza rispetto alle esposizioni tradizionali \`e che, invece di lavorare con tutte le formule in $L(A)$, lavoreremo con un insieme di formule $\Delta$ non necessariamente chiuso per negazione, i teoremi rilevanti valgono solo sotto l'ipotesi $\Delta=\{\A\!\wedge\!\vee\}\Delta$. L'intento \`e di rendere pi\`u visibile la natura topologica dell'argomento (che viene un po oscurata se si assume che i chiusi di base siano anche aperti). Esistono casi in cui questa generalizzazione \`e necessaria (per esempio, nella logica continua che \`e essenzialmente una logica positiva) ma questi non verranno trattati qui. La nozione di $\Delta\jj$modello (essenziale solo se si vuole studiare logica positiva) \`e comunque di interesse culturale.

Chi ritenesse l'approccio inutilmente tedioso (o fosse infastidito dalle troppe negazioni) pu\`o \textit{omettere\/} il primo paragrafo svolgendo per\`o l'esercizio~\ref{ex_omissionetipiclassica}. 

Il secondo ed il terzo paragrafo sono una classica introduzione alle teorie $\omega\jj$categoriche. L'ultimo paragrafo \`e da considerarsi come un esercizio svolto. 

%Per tutto il capitolo tranne il primo paragrafo, lavoreremo modulo una teoria completa $T$ senza modelli finiti e con la notazione e le assunzioni implicite presentate nel paragrafo~\ref{mostro}.


\section{Teorema di omissione dei tipi}\label{paragrafoOTT}

Fissiamo un modello infinito $\U$ in cui valuteremo la verit\`a di tutte le formule (non c'\`e alcuna assunzione implicita di saturazione); eventuali variabili libere si sottintendono quantificate \textit{universalmente}. Per affermare la verit\`a della chiusura \textit{esistenziale\/} di una formula diremo che questa \`e \textit{consistente}.  Diremo che una formula \`e \emph{non banale\/} se la sua negazione e consistente. Quando diremo che l'insieme di formule $\Delta$ \`e numerabile, sottintenderemo sempre: modulo equivalenza in $\U$.

\begin{definition}\label{def_Delta_Baire} Fissiamo $\Delta=\{\wedge\!\vee\}\Delta\subseteq L(\U)$, una tupla di variabili $x$, e sia $\D\subseteq\U^{|x|}$ un insieme arbitrario. Diremo che $\D$ \`e \emph{$\Delta\jj$denso\/} se per ogni $\psi(x)\in\Delta$ non banale $\neg\psi(x)\wedge x\in\D$ \`e consistente.  Diremo che $\D$ ha \emph{interno $\Delta\jj$denso\/} in  se per ogni $\psi(x)\in\Delta$ non banale esiste $\phi(x)\in\Delta$ non banale tale che $\neg\phi(x)\imp x\in\D\wedge\neg\psi(x)$.

Un insieme $\C\subseteq\U^{|x|}$ si dice \emph{$\Delta\jj$comagro\/} se \`e intersezione di una famiglia numerabile di insiemi con interno $\Delta\jj$denso. Diremo che $\U^{|x|}$ \`e \emph{$\Delta\jj$Baire\/} se ogni suo sottoinsieme $\Delta\jj$comagro \`e $\Delta\jj$denso.\QED
\end{definition}
% 
% \begin{definition}\label{def_Delta_T2}  Fissiamo $\Delta\subseteq L(\U)$ e una tupla di variabili $x$. Diremo che $\U^{|x|}$ \`e \emph{$\Delta\jj$Hausdorff\/} se per ogni $a,b\in\U^{|x|}$ tali che $a\nequiv_{\Delta}b$ esistono $\psi(x),\phi(x)\in\Delta$ tali che $\neg\psi(a)\wedge\neg\phi(b)$ e $\neg\psi(x)\wedge\neg\phi(x)$ \`e inconsistente.\QED
% \end{definition}
% 
% 
% \begin{definition}\label{def_Delta_sep}  Fissiamo $\Delta\subseteq L(\U)$ e una tupla di variabili $x$. Diremo che $\U^{|x|}$ \`e \emph{$\Delta\jj$separato\/} se per ogni $a_1,a_2\in\U^{|x|}$, posto $p_i(x)=\Deltatp(a_i)$, allora $p_1(\U)\subseteq p_2(\U)$ implica $p_1(\U)=p_2(\U)$.\QED
% \end{definition}

\begin{definition}\label{def_Delta_ceompatto} Fissiamo $\Delta=\{\wedge\!\vee\}\Delta\subseteq L(\U)$ ed una tupla di variabili $x$. Se $\U$ realizza ogni  $\Delta\jj$tipo $p(x)$ finitamente consistente in $\U$, diremo che  $\U^{|x|}$ \`e \emph{$\Delta\jj$compatto}.\QED
\end{definition}

Dati $a,b\in\U^{|x|}$, scriveremo $a\equiv_{\Delta} b$ quando $\phi(a)\iff\phi(b)$ per ogni $\phi(x)\in\Delta$. La tupla $x$ non compare nella notazione ma sar\`a fissata dal contesto. Per la notazione $\vee\Delta$, $\neg\Delta$, ecc., si rimanda al paragrafo~\ref{TeoremidiPreservazione}. 

\begin{definition}\label{def_Delta_T2}  Fissiamo $\Delta=\{\wedge\!\vee\}\Delta\subseteq L(\U)$ ed una tupla di variabili $x$. Se per ogni $a,b\in\U^{|x|}$ tali che $a\nequiv_{\Delta} b$ esistono due formule $\phi(x),\psi(x)\in\Delta$ tali che $\neg\phi(a)\wedge\neg\psi(b)$ e $\neg\phi(x)\wedge\neg\psi(x)$ \`e inconsistente, diremo che $\U^{|x|}$ \`e \emph{$\Delta\jj$Hausdorff}.\QED
\end{definition}

Si noti che la precedente definizione \`e rilevante quando $\Delta$ non \`e chiuso per negazione, perch\'e $a\nequiv_{\Delta} b$ significa che esiste $\gamma(x) \in \Delta$ tale che vale $\gamma(a) \wedge \neg\gamma(b)$, quindi basta prendere $\phi=\gamma$ e $\psi=\neg\gamma$.

Possiamo associare a $\Delta$ una topologia su $\U^{|x|}$. Una base di chiusi della topologia \`e formata dagli insiemi della forma $\phi(\U)$ per $\phi(x)\in\Delta$. I chiusi sono quindi gli insiemi definiti da $\Delta\jj$tipi $p(x)$. Le nozioni di insieme denso o con interno denso, e di spazio compatto definite qui sopra coincidono con quelle usuali in topologia. Solo la definizione~\ref{def_Delta_T2} si allontana da quella standard in topologia (per la nozione standard avremmo dovuto usare $a\neq b$ invece di $a\nequiv_{\Delta} b$). Infatti, a parte casi banali, in $\U^{|x|}$ esistono sempre coppie di elementi distinti $a\equiv_\Delta b$ che quindi hanno gli stessi intorni aperti. Questo mostra che $\U^{|x|}$ non soddisfa l'assioma di separazione T$_0$. Ma la differenza \`e innocua,  perch\'e si verifica facilmente che la richiesta in~\ref{def_Delta_T2} implica che la topologia indotta sul quoziente $\U^{|x|}/\!\equiv_{\Delta}$ soddisfa l'usuale assioma di separazione T$_2$.
% 
% \begin{exercise}
% Si dimostri che se $\U^{|x|}/\!\equiv_{\Delta}$ soddisfa l'assioma di separazione T$_1$ allora $\neg\!\Delta\jj$compatto implica $\Delta\jj$Hausdorff. In particolare, $\Delta\jj$compatto implica $\Delta\jj$Hausdorff quando $\Delta$ \`e chiuso per negazione. Ricordiamo che $T_1$ nel nostro contesto significa che  per ogni $a,b\in\U^{|x|}$ se $a\Rrightarrow_\Delta b$ allora $a\equiv_\Delta b$.\QED
% \end{exercise}

% Notiamo che $p(x)$ \`e $\Delta\jj$isolato se e solo se $\neg p(\U)$ non \`e un insieme $\Delta\jj$denso.
%
%Inoltre \`e utile tenere a mente che se $\Delta=L(M)$ per qualche $M\preceq\U$ allora $\Delta\jj$isolato equivale a 
% 
% \begin{proposition}\label{coroll_regolare_denso}
% Fissiamo $\Delta\subseteq L(\U)$ ed una tupla di variabili $x$. Le seguenti affermazioni sono equivalenti:
% \begin{itemize}
% \item[1.] $p(x)$ non \`e $\Delta\jj$isolato;
% \item[2.] $\neg p(\U)$ \`e un insieme $\Delta\jj$denso.\QED
% \end{itemize}
% \end{proposition}
Nelle applicazioni in questo capitolo $\U$ \`e un modello saturo e $\Delta=L(A)$. Come $\D$ prenderemo un insieme della forma $\neg p(\U)$ con $p(x)\subseteq L(A)$. Vedi~\ref{omissionetipiclassica}. Avere interno $L(A)\jj$denso si riduce all'esistenza di una formula $\phi(x)\in\L(A)$ tale che $\phi(x)\imp p(x)$.

In topologia generale la seguente propriet\`a \`e detta \emph{regolarit\`a\/} e segue dalla propriet\`a di Hausdorff e dalla compattezza (ed \`e assolutamente ovvia quando $\Delta$ \`e chiuso per negazione). La dimostrazione ripete l'argomento usato in topologia ed \`e riportata solo per completezza.


\begin{proposition}\label{prop_regolarita}
Fissiamo $\Delta=\{\wedge\!\vee\}\Delta\subseteq L(\U)$ ed una tupla di variabili $x$ tali che $\U^{|x|}$ sia $\Delta\jj$Haus\-dorff e $\Delta\jj$compatto. Allora per ogni $\xi(x)\in\Delta$ non banale esistono $\phi(x),\psi(x)\in\Delta$, dove $\psi(x)$ \`e non banale, tali che $\neg\psi(x)\imp\phi(x)$ e $\phi(x)\imp\neg\xi(x)$.
\end{proposition}

\begin{proof}
Fissiamo un arbitrario $b\models\neg\xi(x)$ e sia 

\ceq{\hfill q(x)}{=}{\Big\{\phi(x)\in\Delta\ :\ \neg\psi(x)\imp\phi(x) \textrm{ per qualche }\psi(x)\in\Delta\textrm{ tale che }\neg\psi(b)\Big\}}. 

Mostriamo che se $a\models q(x)$ allora $a\equiv_\Delta b$. Se $a\nequiv_\Delta b$ dalla definizione~\ref{def_Delta_T2} otteniamo due formule $\psi(x),\phi(x)\in\Delta$ tali che $\neg\psi(x)\imp\phi(x)$ e $\neg\psi(b)\wedge\neg\phi(a)$. Quindi $a\notmodels q(x)$. Ora verifichiamo che $q(x)\wedge\xi(x)$ \`e inconsistente: se per assurdo esistesse $a\models q(x)\wedge\xi(x)$, per quanto dimostrato sopra, $a\equiv_\Delta b$ contraddicendo $\neg\xi(b)$. Quindi $q(x) \proves \neg\xi(x)$ 

Per compattezza, otteniamo $\phi_1(x),\dots,\phi_n(x)\in q(x)$ tali che

\ceq{\ }{\ }{\bigwedge^n_{i=1}\phi_i(x)\ \imp\ \neg\xi(x).}

Dalla definizione di $q(x)$ otteniamo $\psi_1(x),\dots,\psi_n(x)\in\Delta$ tali che 

\ceq{\ssf{i.}}{\ }{\bigwedge^n_{i=1}\neg\psi_i(x)\ \imp\ \bigwedge^n_{i=1}\phi_i(x).}

L'antecedente in \ssf{i} \`e soddisfatto da $b$, quindi la formula \smash{$\displaystyle\bigvee^n_{i=1}\psi_i(x)$} \`e non banale, come richiesto.
\end{proof}
% Sia $\Delta\!^{\circ}$ l'insieme delle formule $\phi(x)\in\Delta$ tali che per $\neg\psi(x)\imp\phi(x)$ per qualche $\psi(x)\in\Delta$ non banale. Osserviamo che se $\U^{|x|}$ \`e $\Delta\jj$Hausdorff allora 

%\def\ceq#1#2#3{\hspace*{25ex}\llap{#1}\parbox{6ex}{\hfil#2}\rlap{#3}}

% \ceq{\ssf{r}\hfill a\Rrightarrow_{\Delta\!^{\circ}} b}{\IFF}{a\equiv_\Delta b.}

\begin{theorem}\label{thm_cat_baire} Fissiamo $\Delta=\{\wedge\!\vee\}\Delta\subseteq L(\U)$ ed una tupla di variabili $x$. Entrambe le seguenti ipotesi implicano che $\U^{|x|}$ \`e $\Delta\jj$Baire:
\begin{itemize}
\item[1.]  $\U^{|x|}$ \`e $\Delta\jj$Hausdorff e $\Delta\jj$compatto.
\item[2.]  $\U^{|x|}$ \`e $\neg\!\Delta\jj$compatto.
\end{itemize}
\end{theorem}

\begin{proof}
Siano $\D_i\subseteq\U^{|x|}$, per $i\in\omega$, insiemi con interno $\Delta\jj$denso. Fissiamo una formula $\xi(x)\in\Delta$ non banale. Usando ripetutamente il lemma~\ref{prop_regolarita} possiamo costruire due sequenze di formule $\phi_i(x),\psi_i(x)\in\Delta$ tali che $\neg\psi_{i+1}(x)\imp\phi_i(x)$ e  $\phi_i(x)\imp\neg\psi_i(x)$. Inoltre  per la densit\`a di $\D_i$ possiamo richiedere che $\psi_0(x)=\xi(x)$ e $\neg\psi_{i+1}(x)\imp x\in\D_i$. Il tipo $\{\phi_i(x) : i\in\omega\}$ \`e finitamente consistente, quindi per compattezza realizzato in $\U$. Una qualsiasi sua realizzazione appartiene sia a $\neg\xi(\U)$ che a tutti gli insiemi $\D_i$.

La costruzione \`e pi\`u semplice assumendo \ssf{2}: \`e sufficiente costruire la sequenza di formule $\neg\psi_i(x)$ e appellarsi alla $\neg\Delta\jj$compattezza. I dettagli vengono lasciati al lettore.
\end{proof}   

Scriveremo \emph{$\Delta(b)$\/} per l'insieme delle formule della forma $\xi(x,b)$ per qualche $\xi(x,z)\in\Delta$ e indicheremo con \emph{$\Delta(A)$\/} l'unione di $\Delta(b)$ con $b$ che corre tra le tuple di elementi di $A$ (di lunghezza arbitraria). Scriveremo 

\def\ceqb#1#2#3{\parbox{8ex}{\ssf{m.}\hfill\emph{$#1$}}\parbox{5ex}{\hfil\emph{\ $#2$}}\emph{$#3$}}

\ceqb{M}{\preceq_{\Delta}}{\U}\hfill se $M^{|x|}\mathord\cap\neg\phi(\U)\;\neq\;\0$ per ogni $\phi(x)\in\Delta(M)$ non banale.

A parole, diremo che $M\subseteq\U$ \`e un \emph{$\Delta\jj$modello}. Osserviamo che $M\subseteq\U$ \`e un $\Delta\jj$modello se e solo se $M^{|x|}$ \`e \emph{$\Delta(M)\jj$denso}.

Quando $\Delta=L$ allora $M\preceq_{\Delta}\U$ si riduce a $M\preceq\U$. Infatti la condizione in \ssf{m} \`e quella richiesta dal test di Tarki-Vaught. Quando $\Delta=L(A)$ la condizione in \ssf{m} \`e equivalente a $A\subseteq M\preceq\U$. Infatti, se $\Delta$ contiene la formula $a\neq x'$, con $x'$ una variabile in $x$, allora gli insiemi $\Delta\jj$densi contengono $a$.   

Se $\Delta$ contiene tutte le formule della forma $fz\neq x'$, con $f$ simbolo di funzione, allora i $\Delta\jj$modelli sono sottostrutture di $\U$. Un caso notevole \`e quando $\Delta=L_{\rm qf}$, allora per $M\preceq_{\Delta}\U$ a volte si scrive $M\preceq_{1}\U$, a parole si dice che $M$ \`e una \emph{sottostruttura $1\jj$elementare\/} di $\U$, oppure che $M$ \`e \emph{esistenzialmente chiusa\/} in $\U$. Un termine alternativo per $\Delta\jj$modello potrebbe essere struttura  \emph{$\neg\Delta\jj$esistenzialmente chiusa\/} in $\U$ (ma qui optiamo per il termine pi\`u breve).

\begin{lemma}\label{lemmaomissionetipi}
Fissiamo $\Delta=\{\A\!\wedge\!\vee\}\Delta\subseteq L(\U)$ numerabile. Fissiamo anche due tuple di variabili $x$ e $z$. Sia $\D\subseteq\U^{|x|}$ un insieme con interno $\Delta\jj$denso. Sia $\B\subseteq\U^{|z|}$ l'insieme dei $b$ tali che $\D$ ha interno $\Delta(b)\jj$denso. Allora $\B$ \`e $\Delta\jj$comagro.
\end{lemma}

\begin{proof} Esplicitando la definizione, notiamo che $\B$ \`e formato dalle tuple $b$ tali tali che per ogni $\xi(x,z)\in\Delta$ vale una delle seguenti due alternative:\nobreak
\begin{itemize}
\item[b1.]  $\neg\xi(x,b)$ \`e inconsistente;
\item[b2.]  esiste $\phi(x,z)\in\Delta$ non banale tale che $\neg\phi(x,b)\imp x\in\D\wedge\neg\xi(x,b)$.
\end{itemize}
Sia $\B_\xi$ l'insieme delle tuple $b$ che soddisfano \ssf{b1} e \ssf{b2} per $\xi(x,z)$ fissata. \`E sufficiente mostrare che ogni $\B_\xi$ ha interno $\Delta\jj$denso (perch\'e allora $\B=\bigcap_{\xi \in \Delta} \B_\xi$, e $\Delta$ numerabile per ipotesi). Fissiamo quindi un arbitraria $\psi(z)\in\Delta$ non banale, dobbiamo esibire una $\sigma(z)\in\Delta$ non banale tale che
\begin{itemize}
 \item[c.] $\neg\sigma(z)\imp z\in\B_\xi\wedge\neg\psi(z)$.
\end{itemize}  
Consideriamo due casi. Se $\neg\psi(z)\wedge\neg\xi(x,z)$ \`e inconsistente, allora $\neg\psi(z)\imp z\in\B_\xi$ per \ssf{b1}, e quindi \ssf{c} \`e realizzata con $\psi(z)$ per $\sigma(z)$. Se invece $\neg\psi(z)\wedge\neg\xi(x,z)$ \`e consistente, allora usiamo che $\D$ ha interno $\Delta\jj$denso per ottenere $\phi(x)\in\Delta$ non banale tale che 
\begin{itemize}
\item[d.] $\neg\phi(x)\imp  x\in\D\wedge\E z\,[\neg\psi(z)\wedge\neg\xi(x,z)]$.
\end{itemize} 
Si noti che la formula $\E z\,[\neg\psi(z)\wedge\neg\xi(x,z)]$ sta in $\neg\Delta$ grazie all'ipotesi $\Delta=\{\A\!\wedge\!\vee\}\Delta$.
Ora scegliamo come $\neg\sigma(z)$ la formula $\E x\,[\neg\phi(x)\wedge\neg\psi(z)\wedge\neg\xi(x,z)]$ che \`e consistente per \ssf{d}. Verifichiamo \ssf{c}. La verit\`a di $\neg\sigma(z)\imp \neg\psi(z)$ \`e immediata da \ssf{d}. Per verificare $\neg\sigma(z)\imp z\in\B_\xi$, fissiamo un $b$ tale che $\neg\sigma(b)$, ovvero tale che $\neg\phi(x)\wedge\neg\psi(b)\wedge\neg\xi(x,b)$ \`e consistente. Quindi $b\in\B_\xi$ perch\'e, di nuovo per \ssf{d}, la formula $\neg\phi(x)\wedge\neg\psi(b)$ soddisfa quanto richiesto da \ssf{b2}.
\end{proof}

\begin{corollary}
Sotto le ipotesi del lemma~\ref{lemmaomissionetipi} e con la stessa notazione. Se $\C\subseteq\U^{|x|}$ \`e un insieme $\Delta\jj$comagro, allora l'insieme $\B\subseteq\U^{|z|}$ delle tuple $b$ tali che $\C$ \`e $\Delta(b)\jj$comagro \`e $\Delta\jj$comagro.
\end{corollary}

\begin{proof} 
Se $\C$ \`e l'intersezione di un'infinit\`a numerabile di aperti-densi $\D_i$. L'insieme $\B_i$ delle tuple $b$ tali che  $\D_i$ ha interno denso \`e comagro per il lemma~\ref{lemmaomissionetipi}. L'insieme $\B$ dell'enunciato \`e l'intersezione di tutti questi $\B_i$. \`E immediato che l'intersezione di un insieme numerabile di comagri \`e un comagro. Quindi $\B$ \`e comagro.
\end{proof}




% \begin{lemma}
% Siano $p_i(\U)$ per $i=1,2$ due $\Delta\jj$chiusi disgiunti. Allora esistono delle formule $\phi_i\in\Delta$ tali che $p_i(\U)\subseteq\neg\phi_i(\U)$ e $\neg\phi_i(\U)$ sono disgiunti.
% \end{lemma}



%Siamo interessati a costruire un $\Delta\jj$modello $M$ contenuto in un dato insieme $\D\subseteq\U$. Vogliamo isolare propriet\`a topologiche di $\D$ che assicurano l'esistenza ti tale modello. Per costruire $M$ imiteremo la dimostrazione del teorema di L\"owenheim-Skolem all'ingi\`u: costruiremo una catena di insiemi $A_i$ aggiungendo un elemento alla volta fino ad ottenere il modello desiderato. La propriet\`a da preservare al passo $i\jj$esimo non \`e semplicemente $A_i\subseteq\D$ ma \`e un qualcosa di pi\`u forte: $\D$ \`e $\Delta(A_i)\jj$aperto-denso.


\begin{theorem}[di omissione dei tipi (formulazione topologica)]\label{omissionetipitopologica}
Fissiamo $\Delta=\{\A\!\vee\!\wedge\}\Delta\subseteq L(\U)$ numerabile. Fissiamo anche una tupla di variabili $x$. Sia $\D\subseteq\U^{|x|}$ un insieme con interno $\Delta\jj$denso. Supponiamo che $\U^{|x|}$ sia $\Delta(A)\jj$Baire per ogni $A$ finito. Allora esiste un $\Delta\jj$modello $M$ tale che $\D$ ha interno $\Delta(M)\jj$denso. Lo stesso vale sostituendo ``ha interno denso'' con ``\`e comagro''.
\end{theorem}
\begin{proof} 
Il modello $M$ si costruisce come l'unione di una catena $\<A_i: i<\omega\>$ di insiemi. La strategia \`e la stessa usata per dimostrare il teorema di L\"owenheim-Skolem all'ingi\`u. Le condizioni da soddisfare in questo caso sono quelle date da \ssf{m} che gioca il ruolo del test di Tarski-Vaught. Durante la costruzione richiederemo che $\D$ abbia interno $\Delta(A_i)\jj$denso. Segue che $\D$ sar\`a anche $\Delta(M)\jj$denso come richiesto. Nel secondo caso, che sia comagro). Quindi

Posto $A_0=\0$, al passo $i+1$ fissiamo una enumerazione di lunghezza $\omega$ di tutte le formule non banali di $\Delta(A_i)$.  Sia $i=\<i_1,i_2\>$ e scegliamo $b$ soluzione $\neg\phi_{i_1}(x)$, la $i_1\jj$esima formula nell'enumerazione di $\Delta(A_{i_2})$. Inoltre richiediamo che $\D$ abbia interno $\Delta(A_i,b)\jj$denso. Un tale $b$ esiste perch\'e per il lemma~\ref{lemmaomissionetipi} l'insieme di questi $b$ \`e $\Delta(A_i)\jj$comagro quindi per ipotesi $\Delta(A_i)\jj$denso.

Si verifica facilmente (vedi dimostrazione~\ref{lowenheimskolemallingiu}) che $M$ \`e un $\Delta\jj$modello.
%
%La verifica del test di Tarski-Vaught per $M$ \`e identica a quella del teorema di L\"owenheim-Skolem all'ingi\`u.
\end{proof}

\begin{definition}
Diremo che il tipo $p(x)$ \`e \emph{$\Delta\jj$isolato\/} se vale per qualche $\psi(x)\in\Delta$ non banale $\neg\psi(x)\imp p(x)$. Se $\Delta=\{\vee\}\Delta$ questo significa $p(\U)$ ha interno non vuoto.  Nel caso $\Delta=\{\neg,\vee\}\Delta$ un $\Delta\jj$tipo completo \`e isolato se e solo se \`e principale (cfr.\@ paragrafo~\ref{frammenti}).\QED
\end{definition}

Ricordiamo che omettere un tipo vuol dire non realizzarlo, ovvero \emph{$M$ omette $p(x)$\/} se $M\subseteq \neg p(\U)$.

\begin{theorem}[di omissione dei tipi (formulazione classica)]\label{omissionetipiclassica}
Assumiamo $L$ numerabile e fissiamo un insieme di parametri $A$, anche numerabile. Sia $p(x)\subseteq L(A)$ un tipo non isolato (intendiamo $L(A)\jj$isolato). Allora esiste $M\preceq\U$ che contiene $A$ ed omette $p(x)$.
\end{theorem}

\begin{proof}
Sia $\Delta=L(A)$. Sia $\D=\neg p(\U)$ poich\'e $p(x)$ non \`e $\Delta\jj$isolato allora $\D$ ha interno $\Delta\jj$denso. Per il teorema~\ref{omissionetipitopologica} esiste $\Delta\jj$modello $M$ tale che $\D$ \`e $\Delta(M)\jj$denso. Poich\'e $\Delta$ contiene $L$ otteniamo $M\preceq\U$ e, poich\'e  $\Delta$ contiene le formule $a\neq x'$ per ogni $a\in A$ e $x'$ variabile in $x$, otteniamo $A\subseteq M$. Ora, $\Delta(M)$ contiene anche le formule $c\neq x$ per ogni $c\in M^{|x|}$ e quindi la densit\`a implica $M^{|x|}\subseteq\D$. Ovvero, $M$ omette $p(x)$.
\end{proof}

\begin{exercise}\label{ex_omissionetipiclassica}
Si dia una dimostrazione diretta del teorema~\ref{omissionetipiclassica}. Suggerimento: per poter usare una costruzione alla  L\"owenheim-Skolem all'ingi\`u serve prima mostrare che se $p(x)$ non \`e isolato da nessuna formula in $L(B)$ allora ogni formula consistente $\phi(x)\in L(B)$ ha una soluzione $a$ tale che $p(x)$ non \`e isolato da nessuna formula in $L(B,a)$. Il tipo $q(y)=\tp(a/B)$ va costruito con un processo di \textit{diagonalizzazione\/} imponendo, per ogni $\phi(y,x)\in L(B)$, che $\phi(a,x)$ non isoli $p(x)$. L'idea \`e la stessa del lemma~\ref{lemmaomissionetipi}, i dettagli ben pi\`u semplici.\QED
\end{exercise}

%%%%%%%%%%%%%%%%%%%%%%%%%%%%%%%%%%%%%%%%%
\section{Modelli atomici e modelli primi}


In questo paragrafo $T$ \`e una teoria completa senza modelli finiti ed $\U$ \`e un suo modello saturo $\U$ di cardinalit\`a $>|L|+\omega$. La notazione e le assunzioni implicite sono quelle presentate nel paragrafo~\ref{mostro}.

I questo paragrafo studieremo la seguente nozione:

\begin{definition}
Fissiamo $A$, un insieme arbitrario di parametri. Diremo che \emph{$M$ \`e un modello primo su $A$\/} se per ogni modello $N$ contenente $A$ esiste un'immersione elementare $h:M\imp N$ che fissa $A$. Diremo che \emph{$M$ \`e un modello primo\/} tout court se \`e primo sul vuoto.\QED
\end{definition}

Non esiste una perfetta controparte sintattica alla nozione di modello primo. La nozione di modello atomico \`e ci\`o che pi\`u si avvicina. Le due nozioni coincidono solo se ristrette ai modelli numerabili.

\begin{definition}
Fissiamo $A$, un insieme arbitrario di parametri. Diremo che \emph{$M$ \`e un modello atomico su $A$\/} se $A\subseteq M$ e per ogni $b$, tupla finita di elementi di $M$, il tipo $p(x)=\tp(b/\!A)$ \`e isolato. Diremo che \emph{$M$ \`e un modello atomico\/} tout court se \`e atomico sul vuoto.\QED
\end{definition}

%Fissato $A$ denotiamo con $\D_x$ l'unione di $p(\U)$ per $p(x)\in S_x(A)$ isolato. Diremo brevemente che i tipi (su $A$) isolati sono densi se per ogni tupla $x$ finita $\D_x$ \`e denso. 

Fissiamo $A$. Ricordiamo che una formula $\phi(x)$ \`e completa se \`e consistente e soddisfa alle seguenti condizioni (tra loro ovviamente equivalenti):
\begin{itemize}
 \item[1.] $\phi(x)\imp\psi(x)$ oppure $\phi(x)\imp\neg\psi(x)$
per ogni formula $\psi(x)\in L(A)$; 
\item[2.]  $\phi(x)\imp\psi(x)$ per ogni formula $\psi(x)\in L(A)$ consistente con $\phi(x)$;
\item[3.] esiste un unico tipo $p(x)\in S(A)$ che contiene $\phi(x)$.
\end{itemize}
La nozione dipende sensibilmente sia da $A$ che dalla tupla $x$ che per\`o spesso si devono intuire dal contesto. Quando \`e importante essere espliciti scriveremo \emph{completa in $L_x(A)$}. Quindi un modello $M$ \`e atomico su $A$ se e solo se ogni $a\in M^{|x|}$, con $x$ tupla finita, soddisfa una formula completa in $L_x(A)$. Il seguente lemma \`e immediato. 

\begin{lemma}\label{lem_isolazione1}
Se $\phi(x,z)$ \`e completa in $L_{x,z}(A)$ allora $\E z\,\phi(x,z)$ \`e completa in $L_{x}(A)$.\QED
\end{lemma}


\begin{comment}
\begin{lemma}
Sia $\phi(x)$ una formula completa in $L_x(A)$ e sia $a\models\phi(x)$. Sia $\psi(x,z)\in L(A)$ tale che $\psi(a,z)$ \`e completa in $L_z(A,a)$ allora $\phi(x)\wedge\psi(x,z)$ \`e completa in $L_{x,z}(A,a)$.
\end{lemma}
\begin{proof}
Sia  $\xi(x,z)\in L(A)$ e supponiamo che $\phi(x)\wedge\psi(x,z)\wedge\xi(x,z)$ sia consistente. Allora  $\phi(x)\imp\E z\,[\psi(x,z)\wedge\xi(x,z)]$. Quindi  $\psi(a,z)\wedge\xi(a,z)$ \`e consistente e quindi $\psi(a,z)\imp\xi(a,z)$. 
\end{proof}


\begin{void}[Dimostrazione della proposizione~\ref{prop_atomico_isolatidensi} seconda versione.] 
 Dimostriamo \ssf{2}$\IMP$\ssf{1}. La terminologia \`e quella introdotta nel paragrafo~\ref{paragrafoOTT} con $\Delta=L(A)$, che ometteremo di menzionare per alleggerire la notazione. Sia $x$ una tupla di variabili di lunghezza $\omega$. Se $p\in S_x(A)$ scriveremo $p\mathord\restriction n$ l'insieme delle formule in $p$ dove occorrono al pi\`u variabili in $x\mathord\restriction n$.  Definiamo il seguente insieme $\D\subseteq\U^{|x|}$

\ceq{\hfill \D}{=}{\bigcap_{n\in\omega\phantom{i}}\bigcup_{i\ge n}\;\D_{i}}\hfil
\ceq{\hfill \D_n}{=}{\bigcup_{p\restriction n\ \rm isolato}p(\U)} 

Mostriamo che da \ssf{2} segue che $\D$ ha interno denso. Sia $\psi(x_{\restriction n})$  consistente. Per ogni $i\in\omega$ la formula $\psi(x_{\restriction n+i})$, \`e anche ovviamente (abbiamo solo aggiunto delle variabili ridondanti). Da \ssf{2} segue che $\phi_i(x_{\restriction n+i})\imp\psi(x_{\restriction n+i})$ per delle formule complete $\psi_i(x_{\restriction n+i})$ che possiamo scegliere mutualmente consistenti. Quindi $p(\U)\subseteq\D_n$ per ogni $p\in S_x(A)$ che contiene tutte le formule $\phi(x_{\restriction n+i})$. 

Ora si $M^{|x|}\subseteq\D$. Per verificare che $M$ \`e atomico prendiamo una qualsiasi tupla $a\in M^{|x|}$ mostriamo che $q(x_{\restriction n})=\tp(a_{\restriction n}/A)$ \`e isolato. Ma questo \`e chiaro perch\'e $a\in \D_n$ quindi se $p(x)=\tp(a/A)$ allora $q(_{\restriction n})\subseteq p(x)\restriction n$.
\end{void}
\end{comment}

Fissati $A$ ed $x$, diremo che \emph{i tipi isolati sono densi\/} se  per ogni $\psi(x)\in L(A)$ consistente esiste una formula $\phi(x)\in L(A)$ completa tale che $\phi(x)\imp\psi(x)$. Con la terminologia introdotta nel paragrafo~\ref{paragrafoOTT} questo vuol dire che l'insieme $\D\subseteq\U^{|x|}$, unione di $\phi(\U)$ per $\phi(x)\in L(A)$ completa, ha interno $L(A)\jj$denso.

\begin{proposition}\label{prop_atomico_isolatidensi}
Assumiamo $L$ ed $A$ numerabili. Le seguenti affermazioni sono equivalenti:
\begin{itemize}
\item[1.] esiste un modello $M$ atomico su $A$;
\item[2.] per ogni tupla $x$ finita, i tipi isolati sono densi. 
\end{itemize}
\end{proposition}

\begin{proof}
Dimostriamo \ssf{1}$\IMP$\ssf{2}. Sia $\psi(x) \in L(A)$. Se $\psi(x)$ \`e consistente (in $\U$) allora \`e consistente in ogni modello, quindi anche in $M$. Sia allora $a \in M$ tale che $a \models \psi(x)$. Per l'osservazione di cui sopra, grazie all'ipotesi abbiamo che esiste $\phi(x)$ completa tale che $a \models \phi(x)$. Quindi $\phi(x) \imp \psi(x)$. Dimostriamo ora \ssf{2}$\IMP$\ssf{1}. Il modello atomico richiesto \`e unione di una catena di insiemi $\<A_i:i\in\omega\>$. Posto $A_0=A$, assumiamo come ipotesi induttiva che $A_i=A,a$ dove $a$ \`e un tupla finita che \`e soluzione di una formula $\phi(x)$ completa in $L_x(A)$.  Sia dunque $\psi(a,y)$ una formula consistente che vogliamo sia soddisfatta in $M$. (Tralasciamo i dettagli sull'enumerazione delle formule che sono come nella dimostrazione~\ref{dim_II_lowenheimskolemallingiu} del teorema di L\"owenheim-Skolem all'ingi\`u.) Allora $\phi(x)\wedge\psi(x,y) \in L(A)$ \`e consistente e per \ssf{2} esiste una formula completa in $L_{x,y}(A)$ tale che

\ceq{\hfill\phi'(x,y)}{\imp}{ \phi(x)\wedge\psi(x,y)}

dalla completezza di $\phi(x)$ segue $\phi(x)\imp\E y\ \phi'(x,y)$ e quindi, poich\`e vale $\phi(a)$, allora vale $\exists y \phi'(a,y)$, ovvero $\phi'(a,y)$ \`e consistente. Sia $b$ una qualsiasi sua soluzione, che sar\`a quindi anche soluzione di $\psi(a,y)$. Posto $A_{i+1}=A,a,b$ l'ipotesi induttiva \`e preservata. Infine, dal lemma~\ref{lem_isolazione1} segue  che $M$ \`e atomico.
\end{proof}



% \begin{proof}
% Il modello atomico richiesto \`e $M=A\cup\{a_i:i\in\omega\}$, dove $a=(a_i:i\in\omega)$ \`e la tupla che ora definiamo per induzione. Supponiamo di avere costruito la tupla  $a_{\restriction n}$ e che questa soddisfi una formula $\phi_n(x_{\restriction n})$ completa in $L_{x{\restriction n}}(A)$.  Sia dunque $\psi(a_{\restriction n},x_n)$ una formula consistente che vogliamo sia soddisfatta in $M$. (Tralasciamo i dettagli sull'enumerazione delle formule che sono come nella dimostrazione~\ref{dim_II_lowenheimskolemallingiu} del teorema di L\"owenheim-Skolem all'ingi\`u.) Allora $\phi_n(x_{\restriction n})\wedge\psi(x_{\restriction n+1})$ \`e consistente e per \ssf{2} esiste una formula completa in $L_{x{\restriction n+1}}(A)$ tale che
% 
% \ceq{\hfill\phi_{n+1}(x_{\restriction n+1})}{\imp}{ \phi_n(x_{\restriction n})\wedge\psi(x_{\restriction n+1})}
% 
% dalla completezza di $\phi(x_{\restriction n})$ segue che 
% 
% \ceq{\hfill\phi_n(x_{\restriction n})}{\imp}{\E x_n\ \phi_{n+1}(x_{\restriction n+1})}
% 
% e quindi $\phi_{n+1}(a_{\restriction n},x_n)$ \`e consistente. Sia $a_n$ una qualsiasi sua soluzione, che sar\`a quindi anche soluzione di  $\psi(a_{\restriction n},x_n)$. Dal lemma~\ref{lem_isolazione1} segue immediatamente che $M$ \`e atomico.
% \end{proof}



% If $\psi(x,y)$ isolates $p(x,y)=\tp(a,b/A)$ then
% \begin{itemize}  
% \item[1] $\psi(x,b)$ isolates  $p(x,b)=\tp(a/A,b)$;
% \item[2] $\E y\,\psi(x\,y)$ isolates $q(x)=\tp(a/A)$.\QED
% \end{itemize}

\begin{comment}
\begin{proposition} Sia  $M$ un modello atomico su $A$. Allora ogni $\D$ aperto-denso  nella topologia indotta da $A$ \`e anche aperto-denso nella topologia indotta da $M$. In particolare $M\subseteq \D$.
\end{proposition}

\begin{proof}
Per la proposizione~\ref{tgbnmklop} ogni $\D$ aperto-denso nella topologia indotta da $A$ \`e anche aperto-denso nella topologia indotta da $A,b$ per una qualsiasi tupla $b$ di elementi di $M$. Da questo segue che $\D$ \`e aperto-denso nella topologia indotta da $M$.
\end{proof}
\end{comment}

% 
% Sia $p(x)=\tp(a/A)$ e $q(x,y)=\tp((a,b/A)$. Se $\phi(x)$ isola $p(x)$ allora 
% \begin{itemize}   
% \item[1.]  $p(a,y)$ \`e isolato da una formula su $A$ e allora $\phi(x)$ isolata $a$ over $A,b$;
% \item[2.] se $\psi(a\,y)$ isola $q(y)=\tp(b/A,b)$ allora la formula $\phi(x)\wedge\psi(x,y)$ isola $p(x,y)=\tp(a,b/A_$.
% \end{itemize} 
% \Proof  To prove \casebox{1} assume the hypothesis and let $\xi(\aa\,\bb)$ be some true $A\jj$-formula. Let $\psi(\yy)$ be an $A\jj$-formula 
% which isolates $\bb$ over $A+\aa$.  So, $\psi(\yy)\imp\xi(\aa\,\yy)$ holds. Now use that $\phi(\xx)$ isolates $\aa$ over $A$ to obtain $\phi(\xx)\wedge\psi(\yy)\ \imp\ \xi(\xx\,\yy)$. Substituting $\bb$ for $\yy$ we obtain what is required.  
% The proof of \casebox{2} is very similar: assume the hypothesis and let $\xi(\xx\,\yy)$ be some $A\jj$-formula that is true at $\aa\,\bb$. Then 
% $\psi(\aa\,\yy)\imp\xi(\aa\,\yy)$ holds by the completeness of $\psi(\aa\,\yy)$. Now use that $\phi(\xx)$ isolates $\aa$ over $A$ to obtain 
% $\phi(\xx)\wedge\psi(\xx\,\yy)\ \imp\ \xi(\xx\,\yy)$ which proves the claim.\QED
% 
% \begin{proposition} Sia  $M$ un modello atomico su $A$. Allora ogni $\D$ aperto-denso  nella topologia indotta da $A$ \`e anche  aperto-denso nella topologia indotta da $M$. In particolare $M\subseteq \D$.
% \end{proposition}
% 
% 
% 
% We want to prove that $\aa\,\bb$ is isolated over $A$ if and only if $\aa$ is isolated over $A$ and $\bb$ is isolated over $A+\aa$. One direction is consequence of next proposition (the claim is somewhat more precise), the converse is proved in \ref{Lemma III} below.  
% \Proposition If $\phi(\xx)$ isolates $\aa$ over $A$ then:
% \begin{itemizeshort}   
% \item[\casebox{1}] if $A$ isolates $\bb$ over $A+\aa$ then $\phi(\xx)$ isolates $\aa$ over $A+\bb$; and
% \item[\casebox{2}] if $\psi(\aa\,\yy)$ isolates $\bb$ over $A+\aa$ then the formula $\phi(\xx)\wedge\psi(\xx\,\yy)$ isolates $\aa\,\bb$ over $A$.
% \end{itemizeshort} 
% \Proof  To prove \casebox{1} assume the hypothesis and let $\xi(\aa\,\bb)$ be some true $A\jj$-formula. Let $\psi(\yy)$ be an $A\jj$-formula 
% which isolates $\bb$ over $A+\aa$.  So, $\psi(\yy)\imp\xi(\aa\,\yy)$ holds. Now use that $\phi(\xx)$ isolates $\aa$ over $A$ to obtain $\phi(\xx)\wedge\psi(\yy)\ \imp\ \xi(\xx\,\yy)$. Substituting $\bb$ for $\yy$ we obtain what is required.  
% The proof of \casebox{2} is very similar: assume the hypothesis and let $\xi(\xx\,\yy)$ be some $A\jj$-formula that is true at $\aa\,\bb$. Then 
% $\psi(\aa\,\yy)\imp\xi(\aa\,\yy)$ holds by the completeness of $\psi(\aa\,\yy)$. Now use that $\phi(\xx)$ isolates $\aa$ over $A$ to obtain 
% $\phi(\xx)\wedge\psi(\xx\,\yy)\ \imp\ \xi(\xx\,\yy)$ which proves the claim.\QED




%\begin{proof} Sia $\xi(z,x)$ una formula a parametri in $A$ e prendiamo una $b$ una tupla di elementi di $M$ tale che $\xi(b,x)$ sia consistente. Mostriamo che esiste $\sigma(b,x)$ consistente con $\xi(b,x)$ e tale che $\A x\,[\xi(b,x)\wedge\sigma(b,x)\imp x\in\D]$. Sia $\phi(z)$ la formula che isola $p(x)=\tp(b/A)$. Poich\'e $\D$ \`e aperto-denso, esiste una formula consistente con $\E z\,[\phi(z)\wedge\xi(z,x)]$ e tale che   
% 
% \hfil$\A x\,\Big[\E z\,\big[\phi(z)\wedge\xi(z,x)\big]\,\wedge\,\sigma(x)\ \imp\ x\in\D\Big]$.
% 
% Quindi $\sigma(x)$ \`e la formula $\sigma(b,x)$ richiesta.\end{proof}

Mettiamo in evidenza una propriet\`a che useremo implicitamente nel seguito. La formula $\xi(a,x)$ isola il tipo $p(a,x)$ se vale $\xi(a,x)\imp p(a,x)$. Quest'espressione \`e equivalente alla congiunzione di $\xi(a,x)\imp\phi(a,x)$ al variare di $\phi(a,x)$ in $p(a,x)$. Quindi se $k$ \`e una mappa elementare definita in $a$ vale anche $\xi(ka,x)\imp p(ka,x)$ e possiamo concludere che $\xi(ka,x)$ isola il tipo $p(ka,x)$.

Il seguente lemma caratterizza i modelli atomici con una propriet\`a di estendibilit\`a delle mappe elementari. La propriet\`a \`e una sorta di duale di quella che caratterizzava la saturazione: qui fissiamo il dominio della mappa invece che il codominio.

Chiameremo $k$ una mappa elementare \emph{$A\jj$elementare\/} se $k\cup\id_A$ \`e anche elementare ovvero, se $k$ preserva la verit\`a di tutte le formule a parametri in $A$.

\begin{lemma}\label{atomicoestensione}
Assumiamo $L$ ed $A$ numerabili, ed $A\subseteq M$. Le seguenti affermazioni sono equivalenti:
\begin{itemize}
\item[1.] $M$ \`e atomico su $A$;
\item[2.] per ogni $A\subseteq N$, ogni mappa finita $A\jj$elementare $k:M\imp N$, ed ogni $b\in M$, esiste una mappa $A\jj$elementare $h:M\imp N$ che estende $k$ ed \`e definita in $b$;
\item[3.] come \ssf{2} ma con $k=\0$ e $b\in M^{<\omega}$.
\end{itemize}
Se $L$ o $A$ non sono numerabili rimangono comunque valide le implicazioni \ssf{1}$\,\IMP\,$\ssf{2}$\,\IFF\,$\ssf{3}.
\end{lemma}
\begin{proof}
Per dimostrare \ssf{1}$\IMP$\ssf{2} fissiamo una tupla $c$ che enumera $\dom k$ e sia $\xi(x,y)$ la formula che isola il tipo  $p(x,y)=\tp(c,b/A)$. Poich\'e $k:M\imp N$ \`e elementare $\xi(kc,x)$ ha una soluzione $d\in N$. Poich\'e $\xi(kc,x)\imp p(kc,x)$ otteniamo $p(kc,d)$. Quindi $k=h\cup\{\<b,d\>\}$ \`e l'estensione richiesta.

L'implicazione \ssf{2}$\IMP$\ssf{3} \`e chiara. Per dimostrare \ssf{3}$\,\IMP\,$\ssf{1} assumiamo $\neg$\ssf{1}. Esiste quindi una tupla finita di elementi di $M$ tale che il tipo $p(x)=\tp(b/A)$ non \`e isolato. Poich\'e sia $T$ che $A$ sono numerabili, possiamo usare il teorema di omissione dei tipi per ottenere un modello $N$ che contiene $A$ ed omette $p(x)$. Quindi nessuna mappa elementare $h:M\to N$ pu\`o essere definita in $b$.
\end{proof}

\begin{theorem}\label{atomicoprimo}
Assumiamo $L$ ed $A$ numerabili. Per ogni modello $M$, le seguenti affermazioni sono equivalenti:
\begin{itemize}
\item[1.] $M$ \`e numerabile ed atomico su $A$;
\item[2.] $M$ \`e primo su $A$.
\end{itemize}
\end{theorem}

\begin{proof}
Per dimostrare la direzione \ssf{1}$\,\IMP\,$\ssf{2} fissiamo un arbitrario modello $N$ che contiene $A$ e costruiamo un'immersione $A\jj$elementare $h:M\to N$ costruendo una catena $\<h_i:i<\omega\>$ tale che ogni $h_i:M\to N$ \`e una mappa $A\jj$elementare finita. La catena comincia con $h_0=\0$ e soddisfa alla condizione $b_i\in\dom h_{i+1}$, dove $b_i$ \`e $i\jj$esimo elemento di $M$ in una qualche fissata enumerazione. La condizione \ssf{2} del lemma~\ref{atomicoestensione} assicura l'esistenza di questa catena. 

Per dimostrare la direzione \ssf{2}$\,\IMP\,$\ssf{1} assumiamo che $M$ sia primo su $A$. La numerabilit\`a \`e ovvia: visto che esistono modelli numerabili ed $M$ deve potersi immergere in questi, $M$ non pu\`o che essere numerabile. Inoltre $M$ dev'essere atomico perch\'e verifica la condizione \ssf{3} del lemma~\ref{atomicoestensione}, dato che le immersioni elementari sono per definizione mappe elementari totali.
\end{proof}

\begin{theorem}\label{primiisomorfi}
Assumiamo $L$ ed $A$ numerabili. Allora due modelli primi su $A$ sono isomorfi.
\end{theorem}

\begin{proof} 
Siano $M$ ed $N$ due modelli primi su $A$. Per il teorema~\ref{atomicoprimo} questi sono modelli numerabili e atomici su $A$.  L'isomorfismo $h:M\to N$ viene costruito con il metodo dell'andirivieni. La catena di funzioni finite $\<h_i:i<\omega\>$ verr\`a costruita in modo tale che ogni $h_i:M\to N$ sia $A\jj$elementare e $b_i\in\dom h_{i+1}$ e $c_i\in\range h_{i+1}$, dove $b_i$ e $c_i$ sono gli $i\jj$esimi elementi di $M$ ed $N$ in una qualche fissata enumerazione. La catena comincia con $h_0=\0$. Al passo induttivo usiamo la caratterizzazione dei modelli atomici data da \ssf{2} del lemma~\ref{atomicoestensione} per trovare una mappa finita $A\jj$elementare $h_{i+^1\!/_2}:N\to M$ definita in $c_i$ che estende $h^{-1}_i$ e quindi prendiamo per $h_{i+1}:M\to N$ una qualsiasi estensione finita $A\jj$elementare di $h_{i+^1\!/_2}^{-1}:M\to N$ definita in $b_i$.
\end{proof}

\begin{exercise}
Sia $M$ un modello atomico su $A$ e sia $b$ una tupla finita di elementi di $M$. \`E $M$ \`e atomico su $A,b$? Suggerimento: pu\`o usare il lemma~\ref{atomicoestensione}, o anche una dimostrazione sintattica.\QED
\end{exercise}

\begin{exercise}
Si dimostri che ogni modello atomico \`e debolmente $\omega\jj$omogeneo. Si dimostri che se $L$ \`e numerabile ogni modello primo \`e omogeneo.\QED
\end{exercise}

\begin{exercise} 
Una teoria fortemente minimale ha sempre un modello primo?\QED
\end{exercise}

\begin{exercise}
Si consideri la teoria $T_{\rm rg}$. Per quali insiemi $A$ esiste un modello atomico su $A$? Si risponda alla stessa domanda per la teoria $T_{\rm oldse}$.\QED
\end{exercise}

%%%%%%%%%%%%%%%%%%%%%%%%%%%%%%%%%%%%%%
%%%%%%%%%%%%%%%%%%%%%%%%%%%%%%%%%%%%%%
%%%%%%%%%%%%%%%%%%%%%%%%%%%%%%%%%%%%%%
%%%%%%%%%%%%%%%%%%%%%%%%%%%%%%%%%%%%%%
%%%%%%%%%%%%%%%%%%%%%%%%%%%%%%%%%%%%%%
\section{Teorie sottili}

Anche in  questo paragrafo $T$ \`e una teoria completa senza modelli finiti ed $\U$ \`e un suo modello saturo $\U$ di cardinalit\`a $>|L|+\omega$. La notazione e le assunzioni implicite sono quelle presentate nel paragrafo~\ref{mostro}.

Una teoria $T$ si dice \emph{sottile\/} (in inglese, \emph{small\/}) se $S_x(T)$ \`e numerabile, per ogni tupla finita $x$. Una teoria sottile \`e in particolare numerabile.

\begin{proposition}\label{prop_small_countable_saturated}
Le seguenti affermazioni sono equivalenti:
\begin{itemize}
\item[1.] $T$ \`e sottile;
\item[2.] esiste un modello numerabile saturo.
\end{itemize}
\end{proposition}
\begin{proof}
L'implicazione \ssf{2}$\IMP$\ssf{1} \`e ovvia, dato che i tipi completi sono tipi di qualche tupla $b \in M$. Quindi sono al pi\`u quanti la cardinalit\`a del modello, ovvero una quantit\`a numerabile per ipotesi. Per dimostrare  \ssf{1}$\IMP$\ssf{2} osserviamo che se $T$ \`e sottile esiste un modello numerabile debolmente $\omega$-saturo. Ogni sua estensione \`e ovviamente anche  debolmente satura.  Ogni modello numerabile ha un'estensione ad un modello numerabile omogeneo (esercizio~\ref{exomogeneonumerabile} con $\U$ per $N$). Questo, \`e saturo per l'esercizio~\ref{saturazione debole}.
\end{proof}

Quindi le principali teorie introdotte nei capitoli precedenti sono sottili.

\begin{exercise}
Dalla proposizione~\ref{prop_small_countable_saturated} segue immediatamente che per $|x|<\omega$, le seguenti affermazioni sono equivalenti:
\begin{itemize}
\item[1.] $T$ \`e sottile;
\item[2.] per ogni $A$ finito $S_x(A)$ \`e numerabile.
\end{itemize}
Se ne dia una dimostrazione sintattica diretta.\QED
\end{exercise}

\begin{exercise}
Il linguaggio contiene $\omega$ predicati unari. Sia $T$ la teoria definita nell'esercizio~\ref{ex_linguaggi_unari}. Si dimostri che $T$ non \`e sottile.\QED
\end{exercise}

% \begin{proof}
% Per verificare \ssf{1}$\IMP$\ssf{2}, si osservi che se $S_x(a)$ \`e pi\`u che numerabile, per una qualche tupla finita $a$, allora l'insieme dei tipi $p(z,x)$ tali che $p(a,x)\in S_x(a)$, \`e un sottoinsieme pi\`u che numerabile di $S_{z,x}(T)$.
% 
% Per dimostrare \ssf{2}$\IMP$\ssf{1}, assumiamo \ssf{2} e supponiamo che $S_{z,x}(T)$ sia pi\`u che numerabile e scegliamo $z$ di lunghezza minima. Allora $S_z(T)$ \`e numerabile; sia quindi $(p_i(z):i<\omega)$ una sua enumerazione. Esiste almeno un $i<\omega$ tale che $S_i=\{p(z,x)\in S_{z,x}(T)\ :\ p_i(z)\subseteq p(z,x)\}$ \`e pi\`u che numerabile e sia $a\models p_i(z)$. Ma $S_x(a)$ ha la stessa cardinalit\`a di $S_i$.
% \end{proof}

\begin{definition}
Un albero binario di formule \`e una sequenza di formule $\<\phi_s(x) : s\in2^{<\omega}\>$ tali che:
\begin{itemize}
\item[1.] $\phi_{s0}(x)\vee\phi_{s1}(x)\imp\phi_s(x)$;
\item[2.] $\phi_{s0}(x)\wedge\phi_{s1}(x)$ \`e inconsistente.\QED
\end{itemize}
\end{definition}

In figura, la rappresentazione di un albero binario di formule:


% Set the overall layout of the tree
\tikzstyle{level 1}=[level distance=3.5cm, sibling distance=2cm]
\tikzstyle{level 2}=[level distance=3.5cm, sibling distance=1cm]
\tikzstyle{level 3}=[level distance=1.5cm, sibling distance=.5cm]

% Define styles for bags and leafs
\tikzstyle{bag} = [text width=6ex, text centered]
\tikzstyle{end} = [circle, minimum width=3pt,fill, inner sep=0pt]

\def\leaf{. . .}

\hfil
\begin{tikzpicture}[grow=right]
\node[bag] {$\phi_\0(x)$}
    child {
        node[bag] {$\phi_0(x)$}        
            child {
                node[bag] {$\phi_{00}(x)$}
                    child {
                       node[label=right: {\leaf}] {}
                       edge from parent
                    }    
                    child {
                       node[label=right: {\leaf}] {}
                       edge from parent
                    }  
                 edge from parent
            }
            child {
                node[bag] {$\phi_{01}(x)$}
                edge from parent
                    child {
                       node[label=right: {\leaf}] {}
                       edge from parent
                    }    
                    child {
                       node[label=right: {\leaf}] {}
                       edge from parent
                    }  
                 edge from parent
            }
       edge from parent 
    }
    child {
        node[bag] {$\phi_1(x)$}         
            child {
                node[bag] {$\phi_{10}(x)$}
                    child {
                       node[label=right: {\leaf}] {}
                       edge from parent
                    }    
                    child {
                       node[label=right: {\leaf}] {}
                       edge from parent
                    }  
                 edge from parent
            }
            child {
                node[bag] {$\phi_{11}(x)$}
                edge from parent
                    child {
                       node[label=right: {\leaf}] {}
                       edge from parent
                    }    
                    child {
                       node[label=right: {\leaf}] {}
                       edge from parent
                    }  
                 edge from parent
            } 
        edge from parent
    };
\end{tikzpicture}


\medskip
Ad ogni ramo $\alpha\in2^\omega$ di un albero binario associamo il tipo $p_\alpha(x)=\big\{\phi_{\alpha\restriction n}\;:\; n\in\omega\big\}$. Per saturazione, i tipi $p_\alpha(x)$ sono tutti consistenti.

\begin{proposition}\label{prop_small_tree}
Se $L$ \`e numerabile, le seguenti affermazioni sono equivalenti:
\begin{itemize}
\item[1.] $T$ \`e sottile;
\item[2.] non esiste un albero binario di formule pure.
\end{itemize}
L'implicazione \ssf{1}$\IMP$\ssf{2} vale anche se $L$ non \`e numerabile.
\end{proposition}

\begin{proof}
Per dimostrare \ssf{1}$\IMP$\ssf{2} \`e sufficiente osservare che i tipi $p_\alpha(x)$, per $\alpha\in 2^\omega$, sono mutualmente inconsistenti.

Per dimostrare \ssf{2}$\IMP$\ssf{1} assumiamo che $S_x(T)$ sia pi\`u che numerabile e costruiamo un albero binario di formule. Prendiamo come $\phi_\0(x)$ la formula $x=x$ e supponiamo di aver definito $\phi_s(x)$. Indichiamo con $A_s$, l'insieme dei tipi in $S_x(T)$ consistenti con $\phi_s(x)$. Assumiamo come ipotesi induttiva che $A_s$ sia pi\`u che numerabile. Mostriamo che esiste una formula $\psi(x)$ tale che, posto

\hfil$\phi_{s0}\ =\ \phi_s(x)\wedge\psi(x)$ \hfil e\hfil $\phi_{s1}\ =\ \phi_s(x)\wedge\neg\psi(x)$,

sia $A_{s0}$ che $A_{s1}$ sono pi\`u che numerabili. Supponiamo per assurdo che una tale formula $\psi(x)$ non esista. Fissiamo un enumerazione delle formule pure $\<\psi_i(x):i\in\omega\>$. % Definiamo $\psi_0(x)=\phi(x)$ e $\psi_{i+1}(x)=\psi_i(x)\wedge\phi_i(x)$ se l'insieme dei tipi in $A_s$ \`e pi\`u che numerabile, $\psi_{i+1}(x)=\psi_i(x)\wedge\neg\phi_i(x)$ altrimenti. Quindi i tipi in $A_s$ che non contengono qualcuna delle formule $\psi_i(x)$ sono in quantit\`a numerabile. Esiste un unico tipo che contiene tutte le formule $\psi_i(x)$. Questo contraddice la non numerabilit\`a di $A_s$.
Definiamo $B_0=A_s$ e sia $B_{i+1}$ l'insieme dei tipi in $B_i$ che contengono $\psi_i(x)$, se questo \`e pi\`u che numerabile, altrimenti $B_{i+1}$ \`e l'insieme dei tipi che contengono $\neg\psi_i(x)$. In entrambi i casi $B_i$ ha complemento numerabile. Avremo quindi 

\hfil$\displaystyle S_x(T)\sm\bigcap_{i\in\omega}B_i$ \ \ \`e numerabile\hfil e\hfil $\displaystyle\bigcap_{i\in\omega}B_i$ \ \ contiene un solo tipo.

Questo  contraddice la non numerabilit\`a di $S_x(T)$.
\end{proof}


\begin{proposition}
Se per ogni $|x|<\omega$, non esiste un albero binario di formule in $L_x(A)$ allora esiste un modello atomico su $A$.
\end{proposition}

\begin{proof}
Dimostriamo la contronominale. Omettiamo per chiarezza la referenza all'insieme di parametri $A$. Assumiamo che un modello atomico non esista. Equivalentemente, per la proposizione~\ref{prop_atomico_isolatidensi} che i tipi isolati non siano densi. Quindi esiste una formula consistente $\phi_\0(x)$ che non \`e conseguenza di nessuna formula completa. Supponiamo di aver definito, per $s\in 2^{<\omega}$, una formula $\phi_s(x)$ consistete. Cerchiamo una formula $\psi(x)$ tale che sia $\phi_{s0}(x):=\phi_s(x)\wedge\psi(x)$ che $\phi_{s1}(x):=\phi_s(x)\wedge\neg\psi(x)$ sono consistenti. Una tale $\psi(x)$ deve esistere altrimenti $\phi_s(x)$ sarebbe completa e per ipotesi $\phi_\0(x)$ non \`e conseguenza di una formula completa. Cos\`i otteniamo un albero binario di formule.
\end{proof}


%%%%%%%%%%%%%%%%%%%%%%%%%%%%%%%%%%%%%%
%%%%%%%%%%%%%%%%%%%%%%%%%%%%%%%%%%%%%%
%%%%%%%%%%%%%%%%%%%%%%%%%%%%%%%%%%%%%%
%%%%%%%%%%%%%%%%%%%%%%%%%%%%%%%%%%%%%%
%%%%%%%%%%%%%%%%%%%%%%%%%%%%%%%%%%%%%%
\section{Categoricit\`a numerabile}

Anche in questo paragrafo $T$ \`e una teoria completa con un modello infinito ed $\U$ \`e un suo modello saturo $\U$ di cardinalit\`a $>|L|+\omega$. La notazione e le assunzioni implicite sono quelle presentate nel paragrafo~\ref{mostro}.

Ricordiamo che una teoria $T$ si dice \emph{$\omega\jj$categorica\/} se ha, a meno di isomorfismi, un unico modello numerabile. Sappiamo, dall'esercizio~\ref{categorica->completa} che una teoria numerabile $\omega\jj$categorica \`e completa.

\begin{theorem}[(Engeler, Ryll-Nardzewski, Svenonius)] Sia $T$ una teoria numerabile. Le seguenti affermazioni sono equivalenti:
\begin{itemize}
\item[1] $T$ \`e $\omega\jj$categorica;
\item[2] ogni tipo puro in un numero finito di variabili \`e isolato.
\end{itemize}
\end{theorem}

\begin{proof} 
Dimostriamo \ssf{1}$\IMP$\ssf{2}. Se $T$ \`e $\omega\jj$categorica  tutti i suoi modelli numerabili sono primi (dato che tra due modelli numerabili c'\`e un isomorfismo, mentre per L\"owenheim-Skolem ogni modello pi\`u che numerabile ammette un modello numerabile come sottostruttura elementare) e quindi atomici. Osserviamo ora che per ogni tupla finita $a$, grazie a L\"owenheim-Skolem esiste un modello numerabile che la contiene. Questo modello \`e atomico, quindi $\tp(a)$ \`e isolato. Ogni tipo \`e sottoinsieme di un tipo completo, e in particolare ogni tipo soddisfacibile \`e sottoinsieme del tipo di qualche tupla. Quindi \`e anch'esso isolato, ovvero tutti i tipi puri in un numero finito di variabili sono isolati. Dimostriamo \ssf{2}$\IMP$\ssf{1}. Da \ssf{2} segue che tutti i modelli di $T$ sono atomici, quindi tutti quelli numerabili sono primi, e perci\`o $T$ \`e $\omega\jj$categorica per il teorema~\ref{primiisomorfi}.
%Dimostriamo \ssf{1}$\IMP$\ssf{2}. Se $T$ \`e $\omega\jj$categorica  tutti i suoi modelli numerabli sono (banalmente) primi e quindi atomici. Quindi tutti i tipi puri in un numero finito di variabili sono isolati. Dimostriamo \ssf{2}$\IMP$\ssf{3}. da \ssf{2} segue immediatamente che tutti i modelli numerabili sono atomici, quindi primi. Per dimostrare che sono anche saturi si osservi \ssf{1} implica anche i tipi con un numero finito di parametri sono isolati. Quindi tutti i modelli sono $\omega\jj$saturi. Dimostriamo \ssf{3}$\IMP$\ssf{1}. Fissiamo un modello $N$ primo e saturo (in quanto primo numerabile). Sia $M$ un arbitrario modello numerabile costruiamo un isomorfismo $h:M\imp N$ per andirivieni. Per estendere il dominio usiamo la saturazione di $N$ ed il lemma~\ref{saturo->ricco} per estendere l'immagine l'atomicit\`a di $N$ ed il lemma~\ref{atomicoestensione}.
\end{proof}

Dal seguente teorema otteniamo immediatamente altre caratterizzazioni dell'$\omega\jj$catego\-ri\-ci\-t\`a.

\begin{theorem}\label{omegacategoricaequivalenzevarie} Fissiamo anche una tupla di variabili $x$ di ariet\`a finita $n$. Sia $A$ un insieme di parametri. Le seguenti affermazioni sono equivalenti:
\begin{itemize}
\item[1.] $S_x(A)$ \`e finito;
\item[2.] il numero delle orbite $O(a/A)$ per $|a|=n$ \`e finito;
\item[3.] ogni tipo $p(x)\subseteq L(A)$ \`e isolato;
\item[4.] a meno di equivalenza, esiste un numero finito formule $\psi(x)\in L(A)$.
\end{itemize}
\end{theorem}

\begin{proof} L'equivalenza \ssf{1}$\,\IFF\,$\ssf{2} \`e immediata per l'omogeneit\`a di $\U$, poich\'e in ogni struttura omogenea $N$ vale $O(a/A)=p(N)$ con $p(x)=\tp(a/A)$. Dimostriamo \ssf{1}$\,\IMP\,$\ssf{3}. Se il numero di tipi completi \`e finito, $\neg p(x)$ \`e equivalente alla disgiunzione di tipi. Quindi $p(x)$ \`e equivalente ad una formula, ovvero \`e isolato.  Viceversa, per dimostrare \ssf{3}$\,\IMP\,$\ssf{1}, osserviamo che i tipi $S_x(A)$ ricoprono $\U$, quindi anche le formule che li isolano (che sono a loro equivalenti, dato che i tipi sono completi) formano un ricoprimento di $\U$. Per compattezza questo ricoprimento dev'essere finito. Mostriamo che \ssf{3}$\,\IMP\,$\ssf{4}. 
Sia $\psi(x) \in L(A)$. Se $\psi(x)$ \`e consistente, allora \`e soddisfatta da qualche $a \in \U$, ovvero $\displaystyle \psi(x) \imp \bigvee_{a \models \psi(x)} \tp(a/A)$. Per ogni $a$, sia $\psi_{\!a}(x)$ una formula che isola $\tp(a/A)$. Poich\'e i $\tp(a/A)$ sono completi, essi sono equivalenti alle $\psi_{\!a}(x)$, e quindi in particolare queste sono formule complete. In conclusione otteniamo $\displaystyle \psi(x) \iff \bigvee_{a \models \psi(x)} \psi_{\!a}(x)$. Ma le formule $\psi_{\!a}(x)$ sono in numero finito per \ssf{1}. Il viceversa, \ssf{4}$\,\IMP\,$\ssf{3}, \`e altrettanto semplice.
\end{proof}

\begin{corollary}  Le quattro propriet\`a considerate nel teorema~\ref{omegacategoricaequivalenzevarie}
\begin{itemize}
\item[1.] sono inconsistenti per $A$ infinito;
\item[2.] se valgono per $A=\0$ e per ogni $n$ allora valgono per tutti gli insiemi $A$ finiti; 
\item[3.] se valgono per un qualche $A$ finito, allora valgono anche per $A=\0$.
\end{itemize}
\end{corollary}

\begin{proof} Quando $A$ \`e infinito i tipi completi sono infiniti, questo dimostra \ssf{1}. Se la propriet\`a \ssf{4} del teorema~\ref{omegacategoricaequivalenzevarie} vale per le formule pure allora, sostituendo alcune variabili con parametri, otteniamo che questa vale anche su un arbitrario insieme finito $A$, questo dimostra \ssf{2}. Infine, se la propriet\`a \ssf{1} del teorema~\ref{omegacategoricaequivalenzevarie} vale per un qualche $A$ allora vale anche per ogni $B\subseteq A$.
\end{proof}


\begin{corollary} Le seguenti affermazioni sono equivalenti.
\begin{itemize}
\item[1.] $T$ \`e $\omega\jj$categorica;
\item[2.] per ogni $n$, il numero di tipi puri completi in $n$ variabili \`e finito;
\item[3.] per ogni $n$, il numero delle orbite di tuple di ariet\`a $n$ \`e finito;
\item[4.] per ogni $n$, il numero di formule pure in $n$ variabili \`e finito.\QED
\end{itemize}
\end{corollary}

La dimostrazione della seguente proposizione \`e lasciata per esercizio

\begin{proposition}\label{saturo-primo-su-A-omega-cat}
Sia $T$ numerabile. Le seguenti affermazioni sono equivalenti:
\begin{itemize}
\item[1.] $T$ \`e $\omega\jj$categorica;
\item[2.] per un qualche $A$ finito, esiste un modello che \`e sia saturo che primo su $A$.\QED
\end{itemize}
\end{proposition}


Concludiamo con un curioso risultato di Vaught. Modificano l'esempio presentato nell'esercizio~\ref{VaughtEsempio3modelli} non \`e difficile costruire teorie complete con esattamente $n$ modelli per ogni $3\le n<\omega$.

\begin{proposition}\label{VaughtThm3modelli}
Se $T$ \`e una teoria completa non $\omega\jj$categorica allora $T$ ha almeno $3$ modelli tra loro non isomorfi.
\end{proposition}

\begin{proof}
Se $T$ non \`e sottile ha un continuo di modelli. Assumiamo quindi che $T$ sia sottile. Esistono quindi un modello primo $M$ ed un modello saturo numerabile $N$. Questi non sono isomorfi perch\'e $T$ non \`e $\omega\jj$categorica. Per una qualche tupla $x$ di lunghezza finita esiste un tipo $p(x)$ non isolato. Fissiamo una tupla $a$ di elementi di $N$ che realizza $p(x)$. Sia $M_1\preceq N$ un modello primo su $a$. Quindi $M_1$ non \`e isomorfo a $M$ e non \`e nemmeno isomorfo ad $N$ per la proposizione~\ref{saturo-primo-su-A-omega-cat}. 
\end{proof}




\begin{exercise}
Assumiamo $L$ numerabile, sia $T$ una teoria completa $\omega\jj$categorica. Si  dimostri che per ogni $A$ finito $\<A\>_\U$ \`e una struttura finita.\QED
\end{exercise}

\begin{exercise}
Assumiamo $L$ numerabile. Supponiamo che $T$ sia $\omega\jj$categorica. Sia $A$ un insieme di parametri finito \`e $T$ anche $\omega\jj$categorica su $A$? (Ovvero ogni coppia di modelli $M$ ed $N$ che contengono $A$ esiste un isomorfismo $h:M\imp N$ che fissa $A$.) E se $A$ \`e un insieme infinito?\QED
\end{exercise}

\begin{exercise}
Si dimostri che se esiste un modello $M$ che realizza un numero finito di tipi in $S_x(T)$, per ogni tupla finita $x$, allora $T$ \`e $\omega\jj$categorica.\QED
\end{exercise}


%%%%%%%%%%%%%%%%%%
%%%%%%%%%%%%%%%%%%
%%%%%%%%%%%%%%%%%%
%%%%%%%%%%%%%%%%%%
\section{Versione giocattolo di un teorema di Zilber}

Come applicazione mostriamo se $T$ \`e teoria $\omega\jj$categorica e fortemente minimale allora $T$ non \`e finitamente assiomatizzabile.

Anche in questo paragrafo $T$ \`e una teoria completa senza modelli finiti ed $\U$ \`e un suo modello saturo di cardinalit\`a $>|L|+\omega$. La notazione e le assunzioni implicite sono quelle presentate nel paragrafo~\ref{mostro}.

Diremo che $T$ ha la \emph{propriet\`a del modello finito\/} se per ogni enunciato $\phi$ esiste $A$ struttura finita tale che 

\ceq{\ssf{pmf}\hfill\U\models\phi}{\IFF}{A\models\phi}

La propriet\`a \`e rilevante per la seguente:

\begin{proposition}\label{prop_pmf_fa}
Una teoria $T$ senza modelli finiti con la propriet\`a del modello finito non \`e finitamente assiomatizzabile.
\end{proposition}
\begin{proof}
Se $T$ fosse finitamente assiomatizzabile allora $T\proves\phi\proves T$ per un qualche enunciato $\phi$. E quindi $A\models T$ per una qualche struttura finita, ma $T\proves \E^{>k}\!x\,(x=x)$ per ogni $k$. Contraddizione.
\end{proof}
Abbiamo bisogno della seguente definizione. Diremo che $C$ \`e un \emph{insieme omogeneo\/} per ogni coppia di tuple $a,c\in C$ tale che $a\equiv c$ e per ogni elemento  $b\in C$ esiste un elemento $d\in C$ tale che $a\,b\equiv c\,d$.

\begin{lemma}\label{lem_zilbergiocattolo} Se $T$ \`e numerabile e $\omega\jj$categorica ed ogni insieme finito \`e contenuto in un insieme finito omogeneo allora $T$ ha la propriet\`a del modello finito.
\end{lemma}
\begin{proof}
Per convenienza conviene dimostrare \ssf{pmf} anche per formule con parametri. Dimostreremo che per ogni $n$ esiste $A\subseteq\U$, sottostruttura finita tale che \ssf{pmf} vale per tutti gli $A\jj$enunciati $\phi$ tali che

$\#$\hfil numero dei parametri $\phi$\ \  $+$\ \  numero di quantificatori di $\phi$\ \ $\le n$.

Fissato $n$ scegliamo un qualsiasi $A$ omogeneo in cui tutti i tipi $p(z)$ con $z$ di lunghezza $n$ vengono realizzati. Lasciamo al lettore verificare che $A$ \`e una sottostruttura (l'idea \`e simile a quella esposta nel seguito). Le ipotesi su $T$ garantiscono l'esistenza di un insieme finito con queste propriet\`a. Ora dimostriamo \ssf{pmf} per induzione.

Per le formule atomiche non c'\`e nulla da dimostrare e cos\`i pure per il passo induttivo dei connettivi booleani. Vediamo quindi il passo induttivo del quantificatore esistenziale. Consideriamo quindi la formula $\E y\,\phi(c,y)$ dove $c\in A^{<n}$ e $|y|=1$. La direzione $\PMI$ \`e ovvia dall'ipotesi induttiva notando che se la formula $\E y\,\phi(c,y)$ \`e soddisfa la condizione $\#$, anche le formule $\phi(c,d)$ soddisfano questa condizione. Per dimostrare l'implicazione $\IMP$, assumiamo $\U\models\E y\, \phi(c,y)$. Siano $a\in A^{<n}$ e $b\in A$ una soluzione di $\phi(x,y)$ con $a\equiv c$. Una soluzione esiste in $A$ perch\'e per ipotesi $A$ realizza tutti i tipi con $n$ variabili. Quindi per l'omogeneit\`a di $A$ esiste $d$ tale che $a,b\equiv c,d$.
\end{proof}

\begin{proposition}\label{prop_zilbergiocattolo} Se $T$ \`e fortemente minimale, allora ogni insieme algebricamente chiuso \`e omogeneo.
\end{proposition}
\begin{proof}
Sia quindi $C$ un insieme algebraicamente chiuso e siano $a,c\in C$ tuple tali che $a\equiv c$. Sia $b$ un elemento di $C$. Consideriamo il caso in cui $b\in\acl a$. Sia $f$ un automorfismo di $\U$ che manda $a$ in $c$ e sia $d:=fb\in\acl c\subseteq  C$.  Allora $a,b\equiv c,d$ come richiesto. Nel caso in cui $b\notin\acl a$, osserviamo che ogni $d\notin\acl c$ \`e tale che $a,b\equiv c,d$ e che un tale $d$ esiste in $C$ altrimenti $C=\acl c\neq\acl a$ che contraddirebbe l'ipotesi $a\equiv c$.
\end{proof}

Raccogliendo quanto sopra dimostrato otteniamo il seguente:

\begin{theorem}\label{thm_zilbergiocattolo}
Se $T$ \`e $\omega$-categorica e fortemente minimale allora non \`e finitamente assiomatizzabile.
\end{theorem}

\begin{proof}
Quando $T$ \`e $\omega\jj$categorica la chiusura algebrica di un insieme finito \`e un insieme finito. Quindi dalla proposizione~\ref{prop_zilbergiocattolo} segue che $T$ soddisfa le ipotesi del lemma~\ref{lem_zilbergiocattolo}. Quindi ha la propriet\`a del modello finito e per la proposizione~\ref{prop_pmf_fa} non \`e finitamente assiomatizzabile.
\end{proof}



Se nel teorema~\ref{thm_zilbergiocattolo} indeboliamo le ipotesi richiedendo solamente che $T$ sia totalmente categorica, otteniamo un famoso teorema di Boris Zilber. (Una teoria si dice \textit{totalmente categorica} se in ogni cardinalit\`a ha esattamente un modello a meno di isomorfismi.) Lo stesso teorema \`e stato anche dimostrato indipendentemente da Cherlin, Harrington e Lachlan con una prova che sfrutta la classificazione dei gruppi finiti semplici. Questo teorema ha segnato la nascita di quella che ora si chiama \textit{teoria geometrica della stabilit\`a}, in inglese \textit{geometric stability theory}, che studia quelle propriet\`a geometriche della teoria dei modelli cui abbiamo brevemente accennato nel capitolo~\ref{geometria}.

Il teorema di Zilber richiede un percorso molto lungo. Il teorema~\ref{thm_zilbergiocattolo} \`e comunque un  passaggio obbligato e porter\`a il lettore un passo pi\`u avanti.




\begin{exercise} 
Assumiamo $L$ numerabile. Sia $T$ una teoria completa fortemente minimale. Dimostrare che le seguenti affermazioni sono equivalenti:
\begin{itemize}
\item[1.]$T$ \`e $\omega\jj$categorica;
\item[2.] tutti gli insiemi finiti hanno chiusura algebrica finita.\QED
\end{itemize}
\end{exercise}


