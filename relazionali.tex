\chapter{Alcune strutture relazionali}
\label{relazionali}
%\setcounter{page}{1}

\def\ceq#1#2#3{\parbox{20ex}{$\displaystyle #1$}\parbox{4ex}{\hfil$#2$}$\displaystyle #3$}



%In questo capitolo $x, y,z,\dots$ denotano singole variabili, $a,b,c,\dots$ singoli parametri~--~non tuple 

Due ordini densi numerabili e senza estremi sono isomorfi. Questo \`e un risultato elementare attribuito a Cantor. In questo capitolo esamineremo al microscopio la costruzione di Cantor. Vedremo anche i grafi aleatori, una struttura per certi versi molto simile a quella degli ordini densi.  Dimostreremo che le corrispondenti teorie hanno eliminazione dei quantificatori e sono $\omega\jj$categoriche. Nel capitolo~\ref{eliminazione} torneremo sull'eliminazione dei quantificatori in modo pi\`u concettuale ma \`e utile poter usare questo risultato fin da subito. 

%%%%%%%%%%%%%%%%%%%%%%%%%%%%%%%%%%
%%%%%%%%%%%%%%%%%%%%%%%%%%%%%%%%%%
%%%%%%%%%%%%%%%%%%%%%%%%%%%%%%%%%%
%%%%%%%%%%%%%%%%%%%%%%%%%%%%%%%%%%
%%%%%%%%%%%%%%%%%%%%%%%%%%%%%%%%%%
%%%%%%%%%%%%%%%%%%%%%%%%%%%%%%%%%%
\section{Gli ordini densi}
\label{ordinilinearidensi}
Il linguaggio $L$ degli ordini stretti contenente un solo simbolo di relazione binaria $<$ che useremo con notazione infissa. Una struttura $M$ di segnatura $L$ \`e un \emph{ordine parziale (stretto)} se rende validi i seguenti assiomi (usiamo le usuali abbreviazioni con l'ovvio significato, inoltre, scriviamo delle formule intendendo la loro chiusura universale)

\begin{itemize}
\item[ir.] $x \not< x$
\item[tr.] $x< z< y\ \imp\ x< y$
\end{itemize}

Questi assiomi affermano che  $<$ viene interpretato in una relazione riflessiva e transitiva
\begin{itemize}
\item[as.] $x<y\ \imp\ y\not<x$.
\end{itemize}

Diremo che l'ordine \`e un \emph{lineare\/} o \emph{totale\/} se vale

\begin{itemize}
\item[ln.] $x< y \vee y< x \vee x=y$
\end{itemize}

Un ordine lineare si dice \emph{denso\/} se se vale

\begin{itemize}
\item[nb]  $\E x,y\ (x\neq y)$;
\item[d] $x<y\ \imp\ \E z\;(x<z<y)$.
\end{itemize}

Per evitare ordini banali \ssf{nb} richiedere l'esistenza almeno due elementi, poi da \ssf{d} segue che ogni ordine denso ha infiniti elementi.

Se $a<b$ diremo che $a$ \`e un minorante di $b$ e che $b$ \`e un maggiorante di $a$. Un \emph{elemento massimale\/} \`e un elemento senza maggioranti. Un \emph{elemento minimale\/} \`e un elemento senza minoranti. Diremo che un ordine \`e  \emph{senza estremi\/} se non ha n\'e elementi massinali n\'e elementi minimali, ovvero se vale

\begin{itemize}
\item[se.] $\E y\ (x<y)\ \wedge\ \E y\ (y<x)$
\end{itemize}

I numeri razionali $\QQ$ con l'ordine usuale sono l'esempio canonico di ordine lineare denso senza estremi. Indiche remo con \emph{$T_{\rm ol}$} la teoria degli ordini lineari, con \emph{$T_{\rm old}$} la teoria degli ordini lineari densi e con \emph{$T_{\rm oldse}$} la teoria degli ordini lineari densi e senza estremi.

% 
% \begin{lemma}%\label{}
% Sia $M\models T_{\rm ol}$. Sia $A\subseteq M$ e sia $p(x)=\tp(a/A)$, dove $a$ \`e una tupla finita di elementi di $M$. Allora, se $A$ \`e finito, esiste una formula $\phi(x)\in p(x)$ tale che $T_{\rm ol}\models \phi(x)\iff p(x)$.
% \end{lemma}


%\def\ceq#1#2#3{\hspace*{20ex}\llap{#1}{\parbox{6ex}{\hfil#2}}{\rlap{#3}\hspace{12ex}}}
%\ceq{}{}{}
% 
% \begin{lemma}
% Per ogni mappa $h:M\imp N$ tra modelli di $T_{\rm ol}$ le seguenti affermazioni sono equivalenti
% \begin{itemize}
% \item[a.] $h:M\imp N$ \`e un immersione parziale;
% \item[b.] $M\ \models\  a<b\ \ \ \IFF\ \ \ N\ \models\  h(a)<h(b)$
% \end{itemize}
% \end{lemma}
% 
% \begin{proof}
% \`E sufficiente mostrare che viene preservata la verit\`a di $x\neq y$, poi l'equivalenza segue lemma~\ref{immparfatto}. Questa viene preservata per \ssf{b}, infatti negli ordini lineari \`e equivalente a $x<y\ \vee\ y<x$.
% \end{proof}

Introduciamo un po' di notazione per aumentare la leggibilit\`a della dimostrazione del prossimo teorema. Se $A$ e $B$ sono sottoinsiemi di un insieme ordinato scriveremo $A<B$ se $a<b$ per ogni $a\in A$ e per ogni $b\in B$. Scriveremo $a<B$ e $A<b$ al posto di $\{a\}<B$ e $A<\{b\}$. Un ordine lineare \`e un modello di $M\models T_{\rm oldse}$ esattamente quando:

\begin{itemize}
\item[dse.] per ogni coppia di insiemi finiti $A, B\subseteq M$ tali che $A<B$ esiste $c$ tale  $A<c<B$.
\end{itemize} 

Si osservi che $A<B$ \`e sempre vera quando uno dei due insiemi \`e vuoto, quindi la condizione \textit{senza estremi\/} corrisponde al caso degenere in cui $A$ o $B$ sono vuoti.

Finalmente enunciamo il teorema principale sugli ordini densi senza estremi. Da questo discendono gli altri teoremi di questo paragrafo. 


\begin{lemma}\label{lem_ordinericco}
Siano $M\models T_{\rm ol}$ numerabile, $N\models T_{\rm oldse}$, e $k:M\imp N$ un'immersione parziale finita. Allora esiste un'immersione $h:M\imp N$ che estende $k$.
\end{lemma}

Vedremo due dimostrazioni di questo semplice teorema. La prima \`e diretta:

\begin{proof}
Sia $\<a_i:i<\omega\>$ un'enumerazione di $M$. Definiremo per induzione una catena di isomorfismi parziali finiti $h_i:M\imp N$. Al termine della costruzione porremo

\hfil$\displaystyle h\ \ :=\ \ \bigcup_{i\in\omega} h_i$.

Dalla proposizione~\ref{cateneisomorfismiparziali} sappiamo che $h:M\imp N$, in quanto unione di una catena di immersioni parziali, \`e un'immersione parziale. Qui sotto definiremo $h_{i+1}$ in modo che $a_i\in\dom h_{i+1}$, quindi $h$ sar\`a totale. Al passo base dell'induzione definiamo $h_0=k$ in modo da soddisfare la richiesta $k\subseteq h$. Supponiamo di aver definito $h_i$, al passo $i+1$ estendiamo il dominio di definizione di $h_i$ all'elemento $a_i$ come segue.

Come prima cosa controlliamo se $a_i$ appartiene gi\`a al dominio di definizione $h_i$. (Pu\`o succedere che $a_i$ stia nel dominio di definizione di $k$, o che l'enumerazione scelta abbia delle ripetizioni.) In questo caso non serve fare nulla, \`e sufficiente porre $h_{i+1}=h_i$. Nel caso in cui $a_i\notin\dom h_i$ definiamo $h_{i+1}=h_i\cup\{\<a_i,c\>\}$ dove l'elemento $c$ \`e un elemento di $N$ scelto nel modo che ora mostriamo. Denotiamo con $A$ e $B$ i seguenti insiemi

\hfil$A\displaystyle \ =\ \big\{a\in\dom(h_i)\;:\;a<a_i\big\}$\hfil e\hfil $B\ =\ \big\{b\in\dom(h_i)\;:\;a_i<b\big\}$

e con $h_i[A]$ ed $h_i[B]$ le loro immagini secondo $h_i$. Per la linearit\`a di $M$, e poich\'e $a_i\notin\dom h_i$ questi insiemi formano una partizione di $\dom h_i$. Osserviamo che, $A<B$ e, poich\'e $h_i$ \`e un immersione parziale, anche $h_i[A]<h_i[B]$. Quindi scegliamo come $c$ un qualsiasi elemento $h_i[A]<c<h_i[B]$. Un tale $c$ esiste perch\'e $N$ \`e denso e senza estremi, e $h_i[A],h_i[B]$ sono finiti.

Per completare la dimostrazione \`e sufficiente verificare che $h_{i+1}$ \`e un immersione parziale, ovvero preserva l'ordine. Come ipotesi induttiva abbiamo assunto che $h_i$ preservava l'ordine. Sappiamo anche che $h_{i+1}$ coincide con $h_i$ in tutto il suo dominio di definizione tranne che per $a_i$ in cui vale $c$, quindi rimane solo da controllare che per ogni $d\in\dom h_i$ se $d<a_i$ allora $h_i(d)<c$ e se $a_i<d$ allora $c<h_i(d)$. Ma questo \`e chiaro per la scelta di $c$ e perch\'e $A$ e $B$ ricoprono $\dom h_i$.
\end{proof}

Ora deriviamo dal lemma~\ref{lem_ordinericco} un classico teorema di Cantor: due qualsiasi ordini lineari densi di cardinalit\`a numerabile sono isomorfi. Ne dimostreremo una versione pi\`u generale, il teorema di Cantor si ottiene prendendo per $k:M\imp N$ la mappa vuota che tra strutture relazionali \`e sempre un'immersione parziale (cfr. Corollario~\ref{corollariocaratteristica}). Questa costruzione \`e molto generale e verr\`a riutilizzata in altri contesti \`e un esempio di una tecnica molto versatile che viene chiamata \emph{costruzione andirivieni}, in inglese: \emph{back-and-forth}.

\begin{theorem}\label{zigzagcantor}
Per ogni $M$ ed $N$ modelli numerabili di $T_{\rm oldse}$ ed ogni mappa finita $k:M\imp N$ le seguenti affermazioni sono equivalenti:
\begin{itemize} 
\item[1.] $k:M\imp N$ \`e una immersione parziale; 
\item[2.] esiste un isomorfismo $g:M\imp N$ che estende $k$.
\end{itemize}
\end{theorem}

\begin{proof}
L'implicazione \ssf{2}$\,\IMP\,$\ssf{1} \`e ovvia. Per dimostrare \ssf{1}$\,\IMP\,$\ssf{2} fissiamo delle enumerazioni di $M$ ed $N$. Diciamo $M=\{a_i\;\;i\in\omega\}$ ed $N=\{b_i\;\;i\in\omega\}$. Definiamo per induzione su $i$ una catena di isomorfismi parziali finiti. Ci cureremo di ottenere che $a_i\in\dom g_{i+1}$ e $b_i\in\range g_{i+1}$ per ogni $i\in\omega$. In questo modo otterremo che

\hfil$\displaystyle g\ \ =\ \ \bigcup_{i\in\omega}g_i$

\`e sia totale che suriettiva. Al passo $0$ poniamo $g_0=k$.

Il passo induttivo $i+1$ si divide in due mezzi passi. Col primo mezzo passo estendiamo a $g_{i+^1\!/_2}$ in modo da ottenere $a_i\in\dom g_{i+^1\!/_2}$ col secondo estendiamo ulteriormente per ottenere $b_i\in\range g_{i+1}$. 

Per il lemma~\ref{lem_ordinericco} esiste un'immersione $h:M\imp N$ che estende $g_i:M\imp N$. Definiamo quindi $g_{i+^1\!/_2}=g_i\cup\{\<a_i,ha_i\>\}$. Ora applichiamo lo stesso lemma alla mappa inversa $\big(g_{i+^1\!/_2})^{-1}:N\to M$ per ottenere un'estensione totale $f:N\to M$. Quindi definiamo $g_{i+1}=g_{i+^1\!/_2}\cup\{\<fb_i,b_i\>\}$.
\end{proof}



\begin{comment}
% \begin{proof}
% L'implicazione \ssf{2}$\,\IMP\,$\ssf{1}. \`e ovvia. Per dimostrare \ssf{1}$\,\IMP\,$\ssf{2} fissiamo delle enumerazioni di $M$ ed $N$. Diciamo $M=\{a_i\;\;i\in\omega\}$ ed $N=\{b_i\;\;i\in\omega\}$. Definiamo per induzione su $i$ una catena di isomorfismi parziali finiti distinguendo passi pari dai passi dispari. Ci cureremo di ottenere che $a_i\in\dom g_{2i+1}$ e  $b_i\in\range g_{2i+2}$ per ogni $i\in\omega$. In questo modo otterremo che
% 
% \hfil$\displaystyle g\ \ =\ \ \bigcup_{i\in\omega}g_i$
% 
% \`e sia totale che suriettiva. Al passo $0$ poniamo $g_0=k$. 
% 
% Passo $2i+1$. Per il lemma~\ref{lem_ordinericco} esiste un'immersione $h:M\imp N$ che estende $g_{2i}:M\imp N$. Definiamo quindi 
% 
% \hfil $g_{2i+1}\ =\ h\restriction (\dom g_{2i}\cup\{a_i\})$.
% 
% Passo $2i+2$. Per il lemma~\ref{lem_ordinericco} esiste un'immersione $h:N\imp M$ che estende $g^{-1}_{2i+1}:N\imp M$. Definiamo 
% 
% \hfill $g^{-1}_{2i+2}\ =\ h\restriction (\dom g^{-1}_{2i+1}\cup\{b_i\})$.
% \end{proof}
\end{comment}

Una teoria si dice \emph{$\omega\jj$categorica\/} se ha un unico modello numerabile a meno di isomorfismi. Il teorema di Cantor (cio\`e il teorema~\ref{zigzagcantor} con $k$ la funzione vuota) dice che la teoria degli ordini lineari densi senza estremi \`e $\omega\jj$categorica. 

\begin{exercise}
L'assioma di densit\`a \ssf{od} fa uso di due quantificatori. Si dice che ha la forma \emph{$\A\E$}, ovvero \emph{universale-esistenziale}. Esiste un enunciato universale equivalente modulo $T_{\rm ol}$, a questo assioma? Esiste un enunciato esistenziale equivalente a questo assioma? Esiste un enunciato delle forma \emph{$\E\A$}, ovvero \emph{esistenziale-universale}, equivalente a questo assioma?\QED
\end{exercise}

\begin{exercise}
Consideriamo $\QQ$ ed $\RR$ come strutture nel linguaggio degli ordini stretti. Mostrare che $\QQ\preceq \RR$. (L'esercizio sar\`a ovvio una volta dimostrato il teorema~\ref{imparzialemappeelementari}. Al momento si pu\`o procedere usando L\"owenheim-Skolem all'ingi\`u, e poi ragionare come nell'esempio~\ref{expl_ordini_elem}.)\QED
\end{exercise}

\begin{exercise}
Le seguenti teorie assiomatizzano entrambe la teoria degli ordini lineari \textit{con\/} estremi, ma in un linguaggio diverso. Sia $L$ il linguaggio degli ordini stretti.
\begin{itemize}
\item[] $T_1\ \ =\ \ T_{\rm old}\ \cup\  \Big\{\E x\,z\,\A y\,\ \ x\le y\le z\Big\}$, il linguaggio \`e $L$.
\item[] $T_2\ \ =\ \ T_{\rm old}\ \cup\  \Big\{\A x\ \ 0\le x\le 1\Big\}$, il linguaggio estende $L$ con due costanti $0$ ed $1$.
\end{itemize} 
Si dimostri che il lemma~\ref{lem_ordinericco} vale con $T_2$ al posto di $T_{\rm ol}$ e con $T_2\cup\{\ssf{d}\}$ al posto di $T_{\rm oldse}$. Si porti un controesempio per dimostrare che questo non vale per $T_1$.\QED 
\end{exercise}

\begin{exercise}\label{VaughtEsempio3modelli}
Sia $L$ il linguaggio degli ordini stretti a cui aggiungiamo le costanti $\{c_i: i\in\omega\}$. Sia $T$ la teoria che, oltre agli assiomi degli ordini lineari densi e senza estremi, contiene per ogni $i\in\omega$ l'enunciato $c_i<c_{i+1}$. Mostrare che esistono tre modelli numerabili di questa teoria tra loro non isomorfi. Questo \`e un esercizio pi\`u difficile degli altri. Si tratta di un famoso esempio di teoria completa con esattamente tre modelli (a meno di isomorfismi). L'esempio~--~dovuto a Robert Vaught~--~compare in un famoso articolo in cui dimostra che nessuna teoria completa pu\`o avere esattamente due modelli numerabili. Vedi~\ref{VaughtThm3modelli}\QED
\end{exercise}


%%%%%%%%%%%%%%%%%%%%%%%%%%%%%%%%%%
%%%%%%%%%%%%%%%%%%%%%%%%%%%%%%%%%%
%%%%%%%%%%%%%%%%%%%%%%%%%%%%%%%%%%
%%%%%%%%%%%%%%%%%%%%%%%%%%%%%%%%%%
\section{Grafi aleatori}
\label{grafoaleatorio}

Sia $L$ un linguaggio che contiene un'unica relazione binaria $r$. In teoria dei modelli, per \emph{grafo\/} si intende una struttura che interpreta $r$ in una relazione irriflessiva e simmetrica, ovvero una struttura che rende veri i seguenti assiomi

\begin{itemize}
\item[g1.] $\neg r(x,x)$\hfill irriflessivit\`a
\item[g2.] $r(x,y)\imp r(y,x)$\hfill simmetria
\end{itemize}

Gli elementi di un grafo vengono chiamati \emph{vertici\/} o \emph{nodi}. Un \emph{arco\/} di un grafo $M$ \`e una coppia non ordinata $\{a,b\}$ tale che $M\models r(a,b)$. A parole diremo che $a$ \`e \emph{legato\/} a $b$.

Un \emph{grafo aleatorio\/} (in inglese: \emph{random graph}) \`e una struttura che soddisfa i seguenti assiomi (il secondo \`e uno schema di assiomi, cio\`e per ogni intero positivo $n$, un enunciato diverso)
\begin{itemize}
\item[nb.] $\E x,y,z\ \ (x\neq y\neq z\neq x)$
\item[ga.] $\displaystyle\bigwedge^n_{i,j=1} x_i\neq y_j\ \ \imp\ \ \E z \bigwedge^n_{i=1} \big[r(x_i,z) \wedge \neg r(z,y_i)\big]$
\end{itemize}

La teoria dei grafi aleatori verr\`a denotata con \emph{$T_{\rm rg}$}. La seguente proposizione mostra che si tratta di una teoria consistente.

\begin{proposition}
Esiste un grafo aleatorio.
\end{proposition}
\begin{proof}
Prendiamo come dominio del grafo aleatorio l'insieme dei numeri naturali e stabiliamo che $r(n,m)$ vale se l'$n\jj$esimo numero primo divide $m$ oppure se, viceversa, (per renderla simmetrica) l'$m\jj$esimo numero primo divide $n$. La verifica dell'assioma \ssf{ga} \`e immediata.
\end{proof}

Lo schema di assiomi \ssf{ga} gioca il ruolo che negli ordini avevano l'assioma di densit\`a. Dice che dati due insiemi finiti $X$ e $Y$ disgiunti, esiste un nodo $z$ che \`e legato a tutti gli elementi di $X$ e a nessun elemento di $Y$. Osserviamo che $X$ ed $Y$ non debbono necessariamente avere la stessa cardinalit\`a perch\'e \ssf{ga} non richiede $x_i\neq x_j$ n\'e  $y_i\neq y_j$. L'assioma \ssf{nb} dice che il grafo non \`e vuoto, insieme a \ssf{ga} basta per garantire che sia infinito.

\begin{proposition}
Tutti i grafi aleatori sono infiniti.
\end{proposition}
\begin{proof}
I grafi aleatori hanno almeno tre elementi. Quindi possiamo scegliere due elementi distinti $a,b\in M$ e porre $Y=\{a,b\}$ e $X=M\sm Y\neq \0$.  Per \ssf{ga}, esiste $z$ legato a tutti gli elementi di $X$ e a nessun elemento di $Y$. Per l'irriflessivit\`a, $z\in\{a,b\}$, diciamo $z=a$. Sempre per \ssf{ga}, esiste $w$ legato a $b$ e a nessun elemento di $X\cup\{a\}$. Per l'irriflessivit\`a $w\in X\cup\{a\}$. Ma tutti gli elementi di $X\cup\{a\}$ sono legati a qualche elemento in $X\cup\{a\}$: infatti per costruzione $a$ \`e legato a tutti gli elementi di $X$. Contraddizione.
\end{proof}

Gli assiomi in \ssf{ga} non richiedono $z\notin X\cup Y$. Comunque per l'irriflessivit\`a $z\notin X$ e aggiungendo la richiesta $z\notin Y$ non otteniamo una teoria pi\`u forte. 

\begin{exercise}\label{assioma_ga_esteso}
In ogni grafo aleatorio valgono per ogni $n$ i seguenti enunciati

\hfill$\displaystyle\bigwedge^n_{i,j=1} x_i\neq y_j\ \ \imp\ \ \E z \bigwedge^n_{i=1} \big[r(x_i,z) \wedge \neg r(z,y_i)\wedge y_i\neq z\big]$.\QED
\end{exercise}

\`E immediato che le immersioni parziali $h:M\imp N$ tra due grafi sono mappe iniettive tali che per ogni coppia di elementi $a,b\in\dom h$:

\hfil $M\ \models\  r(a,b)\ \ \ \IFF\ \ \ N\ \models\  r(h(a),h(b))$

Il seguente teorema asserisce per i grafi aleatori quello che il lemma~\ref{lem_ordinericco} asseriva per gli ordini lineari densi cio\`e che questi sono ricchi relativamente alla classe dei grafi numerabili.

\begin{lemma}\label{lem_graforicco}
Sia $M$ un grafo di cardinalit\`a numerabile, $N$ un grafo aleatorio, e $k:M\imp N$ un'immersione parziale finita. Allora esiste un'immersione $h:M\imp N$ che estende $k$.
\end{lemma}
\begin{proof} Riprendiamo lo schema della dimostrazione per gli ordini lineari, con la stessa notazione. \`E sufficiente mostrare come trovare in $N$ un elemento $c$ che rende $h_{i+1}:=h_i\cup\{(a_i,c)\}$ un'immersione parziale. Poniamo\medskip

\hfil$A=\big\{a\in\dom h_i\ :\ M\models r(a,a_i)\big\}\ \ \ \textrm{e}\ \ \ B=\big\{b\in\dom h_i\ :\  M\models \neg r(b,a_i)\big\}$.\medskip

Questi insiemi sono finiti e disgiunti quindi, per l'assioma {\sf ga}, possiamo scegliere $c$ tale che\medskip

\hfil$\displaystyle\bigwedge_{a\in A}r(h_i(a),c)\ \ \wedge\ \bigwedge_{b\in B}\neg r(h_i(b),c)$.

\`E immediato verificare che $h_{i+1}:M\imp N$ \`e un'immersione parziale.
\end{proof}

Quindi otteniamo anche il seguente corollario, corrispondente al corollario~\ref{zigzagcantor}. Nel caso $k=\0$ il corollario dice che la teoria dei grafi aleatori \`e $\omega\jj$categorica.

\begin{corollary}\label{gaomegacat}
Per ogni $M$ ed $N$ modelli numerabili di $T_{\rm rg}$ ed ogni mappa finita $k:M\imp N$ le seguenti affermazioni sono equivalenti:
\begin{itemize} 
\item[1.] $k:M\imp N$ \`e una immersione parziale; 
\item[2.] esiste un isomorfismo $g:M\imp N$ che estende $k$.
\end{itemize}
\end{corollary}
\begin{proof}
Segue dal lemma~\ref{lem_graforicco} con letteralmente la stessa dimostrazione del teorema~\ref{zigzagcantor}.
\end{proof}

\begin{exercise}
Sia $N$ un grafo aleatorio e sia $M\subseteq N$ con $N\sm M$ finito. \`E vero che anche $M$ \`e un grafo aleatorio?\QED% Suggerimento: si dimostri che in un grafo aleatorio, dati $\{x_1,\dots,x_n\}$ e $\{y_1,\dots,y_n\}$, esistono infiniti $z$ che testimoniano la validit\`a di \ssf{ga}.
\end{exercise}

\begin{exercise}
Sia $N$ un grafo aleatorio si dimostri che per ogni elemento $b\in N$ l'insieme $r(b,N)$ \`e un grafo aleatorio.\QED
\end{exercise}

\begin{exercise}\label{unionedisgiunta1}
Sia $N$ un grafo che \`e unione libera di due grafi aleatori numerabili. Cio\`e $N$ ha come dominio l'unione disgiunta dei domini di due grafi aleatori $N_1$ ed $N_2$ e come relazione l'unione di $r^{N_1}$ ed $r^{N_2}$. Mostrare che questo \`e un grafo universale non aleatorio. Si esibisca una mappa finita $k:N\to N$ che non si estende ad un automorfismo.\QED
\end{exercise}

\begin{exercise}\label{unionedisgiunta2}
Sia $\K$ la classe delle strutture che sono unione libera di due grafi aleatori (vedi esercizio~\ref{unionedisgiunta1}). Si dimostri che $\K$ \`e una classe elementare, ovvero che esiste una teoria $T$ tale che $\K=\Mod(T)$. Suggerimento: si usi che per ogni due elementi di grafo aleatorio esiste un terzo a cui entrambi sono collegati (si usa dire: la distanza tra due vertici di un grafo aleatorio \`e $\le 2$). Si dimostri che $T$ \`e $\omega\jj$categorica.\QED
\end{exercise}


\begin{exercise}\label{ex_grafo_aleatorio_no_finiti_definibili} Dati $A\subseteq N\models T_{\rm rg}$ si dimostri che ogni formula $\phi(x)\in L(A)$ se ha una soluzione in $N\sm A$ allora ha infinite soluzioni.\QED
\end{exercise}


\begin{exercise}\label{N1N2}
Siano $N_1$ ed $N_2$ due grafi aleatori numerabili. Sia $N$ un grafo che ha per dominio l'unione disgiunta di $N_1$ ed $N_2$ e come archi quelli di $N_1$ pi\`u quelli di $N_2$ pi\`u quelli che congiungono tutti i vertici di $N_1$ con tutti i vertici di $N_2$. Si dimostri concisamente che $N$ non \`e un grafo aleatorio. Esiste un immersione di $N$ in un grafo aleatorio? Mostrare che l'insieme $N_1$ \`e definibile con parametri in $N$ ma non \`e definibile senza parametri.\QED %Suggerimento: \`e simile all'esercizio~\ref{unionedisgiunta} per quanto visto all'esercizio~\ref{grafocomplementare}.
\end{exercise}

%$\E x,y\; \Big[ r(x,b)\wedge r(y,b)\wedge \neg\E z\,\big[z\neq b\wedge r(x,z)\wedge r(x,z)\big]\Big]$

\begin{exercise}
Siano $N_1$ ed $N_2$ due grafi aleatori numerabili e sia $c\in N_2$. Sia $N$ un grafo che ha per dominio l'unione disgiunta di $N_1$ ed $N_2$ e come archi quelli di $N_1$ pi\`u quelli di $N_2$ pi\`u quelli che congiungono tutti i vertici di $N_1$ con $c$. Si dimostri concisamente che $N$ non \`e un grafo aleatorio. \`E l'insieme $N_1$ definibile con una formula pura? Si dimostri che due qualsiasi di questi grafi sono tra loro isomorfi.\QED
\end{exercise}


\begin{exercise}
Sia $T$ la teoria definita nell'esercizio~\ref{unionedisgiunta2} e sia $N\models T$. Si dimostri che per ogni grafo numerabile $M$, ogni mappa finita $k:M\to N$ che preserva la verit\`a delle formule esistenziali ha un'estensione ad una mappa totale $h:M\to N$ che preserva la verit\`a le formule esistenziali.\QED
\end{exercise}

% %%%%%%%%%%%%%%%%%%%%%%%%%%%%%%%%%immersione
% %%%%%%%%%%%%%%%%%%%%%%%%%%%%%%%%
% %%%%%%%%%%%%%%%%%%%%%%%%%%%%%%%%
% %%%%%%%%%%%%%%%%%%%%%%%%%%%%%%%%%%%%%%%%
% \section{Equivalenze mutualmente indipendenti}
% 
% La segnatura $L_{\rm re}$ contiene solo relazioni binarie. Non fisseremo la cardinalit\`a di $L_{\rm re}$, che pu\`o essere arbitaria. La teoria che denomineremo con $T_{\rm re}$ contienme per ogni $r\in L$ gli assiomi che affermano
% 
% \begin{itemize}
% \item[re.] $r(x,y)$ definisce una relazione di equivalenza.
% \end{itemize}
% 
% La teoria delle equivalenze mutualmente indipendenti $T_{\rm rei}$ contiene anche, per ogni $n, m$ ed ogni scelta delle relazioni $r_1,\dots,r_n, s_1,\dots,s_m\in L$ gli assiomi
% % 
% % \begin{itemize}
% % \item[i.] $\displaystyle\neg \A z\ \bigg[\bigwedge^n_{i=1}s_i(x_0,z)\; \iff\; \bigvee^n_{i=1} r_i(x_i,z)\bigg]$
% % \end{itemize}
% %Questo schema dice che la nessuna classe della relazione di equivalenza e $s_1\wedge\dots\wedge s_m$ \`e unione di classi delle relazioni $r_1,\dots,r_n$. 
% 
% 
% \begin{itemize}
% \item[i.] $\displaystyle  \bigwedge^n_{i=1}\neg r_i(x_0,x_i)\ \imp\ \E z\ \bigg[\bigwedge^n_{i=1}r_i(x_0,z)\; \wedge\; \bigwedge^n_{i=1} \neg s_i(x_i,z)\bigg]$
% \end{itemize}




\section{Strutture omogenee-universali}\label{omogenee-universali}

La seguente definizione generalizza gli esempi esposti nei paragrafi precedenti.

\begin{definition}
Sia $\M$ una classe di strutture e $\lambda$ un cardinale infinito. Diremo che una struttura $N\in\M$ \`e \emph{$\lambda\jj$ricca\/} per $\M$ se per ogni $M\in\M$ di cardinalit\`a $\le\lambda$ ed ogni immersione parziale finita $k:M\imp N$ di cardinalit\`a $<\lambda$ esiste un'immersione $h:M\imp N$ che estende $k$. Quando $\M$ \`e la classe dei modelli di una qualche teoria $T$, diremo che $N$ \`e un modello ricco \emph{di\/} $T$.

Diremo che $N$ \`e una struttura \emph{ricca\/} tout court  se \`e $\lambda\jj$ricca con $\lambda=|N|$. Le strutture ricche sono anche chiamate strutture \emph{omogenee-universali\/} per la ragione che vedremo tra breve.\QED
\end{definition}

Il lemma~\ref{lem_ordinericco} mostra che ogni $N\models T_{\rm oldse}$ \`e un modello $\omega\jj$ricco della teoria degli ordini lineari. Il lemma~\ref{lem_graforicco} mostra l'analogo per i grafi aleatori e la teoria dei grafi. (E vale anche una sorta di viceversa, cfr.\@ esercizio~\ref{ex_poijnmkiuyg}.) Pi\`u avanti descriveremo i modelli ricchi della teoria dei domini di integrit\`a, e di varie altre teorie.

\begin{exercise}\label{ex_poijnmkiuyg}
Si dimostri che le seguenti affermazioni sono equivalenti:
\begin{itemize}
\item[1.] $N\models T_{\rm oldse}$;
\item[2.] $N$ \`e un modello ricco di $T_{\rm ol}$.
\end{itemize}
Si dimostri una proposizione analoga per $T_{\rm rg}$ e $T_{\rm grf}$.\QED
\end{exercise}



Possiamo riformulare il teorema~\ref{zigzagcantor} nel seguente modo.

\begin{theorem}\label{riccozigzag}
Sia $\M$ una classe di strutture e siano $M, N\in\M$ due modelli ricchi della stessa cardinalit\`a $\lambda$. Allora ogni immersione parziale $k:M\imp N$ di cardinalit\`a $<\lambda$ si estende ad un isomorfismo. In particolare se $M$ ed $N$ hanno la stessa caratteristica sono isomorfi.
\end{theorem}

\begin{proof}
Come osservato, la dimostrazione teorema~\ref{zigzagcantor} dipende solo dal lemma~\ref{lem_ordinericco} e quindi pu\`o essere riportata pari pari al caso generale. Solo, ora $k$ potrebbe non essere finita. Ma \`e sufficiente estendere la costruzione del teorema~\ref{zigzagcantor} ad ordinali infiniti prendendo l'unione ai passi limite. 
\end{proof}

Il teorema~\ref{riccozigzag} non menziona teorie, infatti la ricchezza non \`e in generale una propriet\`a elementare (cio\`e, invariante per equivalenza elementare). Quando succede, otteniamo un risultato di categoricit\`a. 

\begin{corollary}
Sia $\M$ una classe di strutture e sia $T$ una teoria. Supponiamo che tutti i modelli di $T$ di cardinalit\`a $\lambda$ siano ricchi ed abbiano la stessa caratteristica. Allora allora $T$ \`e $\lambda\jj$categorica.\QED
\end{corollary}

La seguente \`e un'altra importante conseguenza dell'assiomatizzabilit\`a dei modelli ricchi.

\begin{lemma}\label{imparzialemappeelementari}
Sia $\lambda$ un cardinale infinito $\ge|L|$. Sia $\M$ una classe di strutture e sia $T$ una teoria.  Supponiamo che tutti i modelli di $T$ di cardinalit\`a $\lambda$ siano ricchi. Allora le seguenti affermazioni sono equivalenti per ogni $M,N\models T$ ed ogni mappa $k:M\to N$:\nobreak
\begin{itemize}
\item[1.] $k:M\to N$ \`e un'immersione parziale;
\item[2.] $k:M\to N$ \`e una mappa elementare.
\end{itemize}
In particolare, se tutti i modelli di $T$ hanno la stessa caratteristica, $T$ \`e completa.
\end{lemma}

Il risultato vale sotto ipotesi pi\`u deboli, vedi esercizio~\ref{ex_imparzialemappeelementari}.

%\def\tqf{\hspace{-.3ex}\raisebox{-.25ex}{\mbox{\rm\tiny qf}}}


\begin{proof}
La direzione \ssf{2}$\IMP$\ssf{1} \`e ovvia, dimostriamo quindi \ssf{1}$\IMP$\ssf{2}. Come osservato nel lemma~\ref{naturafinitamapel}, possiamo assumere senza perdere di generalit\`a che $k$ sia finita. Consideriamo prima il caso in cui $M$ ed $N$ siano modelli ricchi in $\M$. Esiste quindi un isomorfismo, in particolare una mappa elementare, $g:M\to N$ che estende $k$. Gli isomorfismi sono mappe elementari, quindi anche $k$ \`e elementare.

Per il caso generale ragioniamo come nel lemma~\ref{categorica->completa}. Applichiamo prima il teorema di L\"owenheim-Skolem all'ins\`u per ottenere $M'\succeq M$ ed $N'\succeq N$ entrambi di cardinalit\`a maggiore di $\lambda$. Ora vogliamo $M''$ ed $N''$ di cardinalit\`a $\lambda$ tali che $\dom k\subseteq M''\preceq M'$ ed $\range k\subseteq N''\preceq N'$. L'esistenza di questi modelli segue dal teorema di L\"owenheim-Skolem all'ingi\`u poich\'e abbiamo assunto $k$ finita. Ora, per quanto dimostrato sopra $k:M''\to N''$ \`e elementare. Segue che  $k:M\to N$ \`e elementare.

L'osservazione finale segue immediatamente dal corollario~\ref{corollariocaratteristica}.
\end{proof}

\begin{exercise}\label{ex_imparzialemappeelementari}
Si mostri che nel teorema~\ref{imparzialemappeelementari} \`e sufficiente assumere che tutti i modelli di $T$ di cardinalit\`a $\le|L|+\omega$ hanno un estensione elementare ad una struttura ricca di cardinalit\`a $\lambda$. (Cfr.\@ con esercizio~\ref{ricchezza_saturazione_EQ}.)\QED
\end{exercise}

Quando le immersioni parziali tra modelli di una teoria $T$ coincidono con le mappe elementari, \`e sempre per una ragione molto fondamentale: tutte le formule sono equivalenti a formule senza quantificatori. 
%
% Per cominciare verifichiamo che ogni tipo completo \`e equivalente ad un tipo senza quantificatori: se $p(x)$ \`e un tipo completo e $p_{\tqf}(x)$ \`e l'insieme delle formule libere in $p(x)$ allora  
% 
% \hfil$\displaystyle T\proves\ p(x)\iff p_{\tqf}(x)$.
% 
% Infatti se $N\models p_{\tqf}(b)$, per mostrare che $N\models p(b)$, \`e sufficiente prendere una qualsiasi realizzazione $M\models p(a)$ e considerare l'immersione parziale $h:M\to N$, con $h=\{(a,b)\}$. Se questa \`e elementare, $N\models p(b)$.
%
%Trasportare questa osservazione a livello delle formule \`e pi\`u laborioso e richiede l'uso del teorema di compattezza. 
Fissiamo un po' di terminologia. Diremo che la teoria $T$ \emph{ammette\/} (o \emph{ha\/}) \emph{eliminazione dei quantificatori} se se ogni formula $\psi(x)$ esiste una formula senza quantificatori $\phi(x)$ tale che 

\hfil$\displaystyle T\proves\ \psi(x)\iff\phi(x)$.

L'eliminazione dei quantificatori verr\`a ripresa e trattata in dettaglio nel capitolo~\ref{eliminazione}. Qui citiamo solo senza dimostrazione il seguente risultato (una versione semplificata del corollario~\ref{criterioeq1}). Nel frattempo dell'eliminazione dei quantificatori useremo solo la forma debole stabilita dal lemma~\ref{imparzialemappeelementari}.


\begin{theorem}\label{tutteleimmersionisonoelementari=eliminazioneqantificatori}
Le seguenti affermazioni sono equivalenti:
\begin{itemize}
\item[1.] $T$ ha eliminazione dei quantificatori;
\item[2.] ogni immersione parziale tra modelli di $T$ \`e una mappa elementare.\QED
\end{itemize}
\end{theorem}

In particolare, l'eliminazione dei quantificatori \`e utile per stabilire la completezza delle teorie.

\begin{corollary}\label{elimquanToldseTrg}
Le teorie  $T_{\rm oldse}$  e $T_{\rm rg}$ sono teorie complete che ammettono eliminazione dei quantificatori.\QED
\end{corollary}


\begin{exercise}\label{ex_R=Mod(Th(R))}
Sia $\lambda$ un cardinale infinito $\ge|L|$ e sia $T$ una teoria arbitraria. Supponiamo che la seguente classe non sia vuota 

\ceq{\hfill\R}{=}{\big\{M\ :\ M \textrm{ modello }\lambda\textrm{-ricco di }T\big\}}

Si dimostri che se $\R$ \`e una classe elementare, cio\`e $\R=\Mod\big(\Th(R)\big)$, allora $\Th(\R)$ \`e una teoria $\lambda\jj$categorica con eliminazione dei quantificatori.\QED 
\end{exercise}

\begin{definition}\label{def_omogenea_universale}
Sia $\M$ una classe di strutture e $\lambda$ un cardinale infinito. Diremo che una struttura $N\in\M$ \`e \emph{$\lambda\jj$universale\/} se per ogni $M\in\M$ di cardinalit\`a $\le\lambda$ esiste un'immersione $h:M\imp N$ che estende $k$. 

Diremo che $N$ \`e \emph{$\lambda\jj$(ultra)omogenea\/} se per ogni immersione parziale finita $k:N\imp N$ di cardinalit\`a $<\lambda$ esiste un automorfismo $h:N\imp N$ che estende $k$. Come per la ricchezza, se $\lambda=|N|$, diremo semplicemente \emph{universale\/} e \emph{omogenea}.
\end{definition}

Per esempio, il grafo vuoto, o quello completo sono omogenei. A aggiungendo ad un grafo aleatorio un nodo sconnesso da tutti gli altri, otteniamo un grafo universale che per\`o non \`e omogeneo.


\begin{theorem}\label{ricco<->universaleomogeneo}
Sia $\M$ una classe di strutture tutte con la stessa caratteristica. Allora le seguenti affermazioni sono equivalenti:
\begin{itemize}
\item[1.] $N$ \`e ricco;
\item[2.] $N$ \`e universale ed omogeneo.
\end{itemize}
\end{theorem}
\begin{proof} 
Dimostriamo \ssf{1}$\IMP$\ssf{2}. Per ipotesi la mappa vuota \`e un immersione parziale, quindi l'universalit\`a \`e una caso particolare della ricchezza. L'omogeneit\`a segue dal teorema~\ref{riccozigzag}. Dimostriamo \ssf{2}$\IMP$\ssf{1}. Dobbiamo verificare che ogni immersione parziale finita $k:M\imp N$, con $M$ un grafo numerabile, pu\`o essere estesa ad un immersione. Per l'universalit\`a esiste un'immersione $f:M\imp N$. La mappa $k\circ f^{-1}:N\imp N$ \`e un'immersione parziale finita che, per ultraomogeneit\`a, ha un estensione ad un automorfismo $h:N\imp N$. \`E immediato verificare che $h\circ f:M\imp N$ estende $k$. Questa \`e dunque l'immersione richiesta.
\end{proof}

Una generalizzazione immediata del concetto di modello ricco si ottiene sostituendo le immersioni parziali con $\Delta\jj$morfismi dove $\Delta$ \`e una insieme di formule. Tipicamente, la classe dei morfismi dovr\`a essere intuita dal contesto. Il teorema~\ref{riccozigzag} si generalizza a $\Delta\jj$morfismi a condizione che $\Delta$ sia chiuso per negazione e contenga la formula $x=y$. Questo assicura che l'inverso di un $\Delta\jj$morfismo sia ancora un $\Delta\jj$morfismo. Chiaramente in questo caso la costruzione ad andirivieni produce sono un $\Delta\jj$morfismo biettivo che, se $\Delta$ non contiene $\pmaL$, potrebbe non essere un isomorfismo.  

Nel capitolo~\ref{saturazione} introdurremo i modelli $\lambda\jj$saturi e mostreremo che, se come morfismi prendiamo le mappe elementari, $N$ \`e un modello $\lambda\jj$saturo se e solo se \`e $\lambda\jj$ricco nella classe dei modelli di $\Th(N)$. In quel capitolo studieremo concetti di universalit\`a e omogeneit\`a relativi alle mappe elementari.

Un'ulteriore generalizzazione si ottiene prendendo come morfismi un insieme di mappe scelte con criteri combinatori. Una tecnica introdotta da Hrushovski negli anni '90 per ottenere strutture con propriet\`a molto sorprendenti (tra queste, il controesempio menzionato nel paragrafo~\ref{tfm}). 


\begin{exercise}\label{categorica->completa}
Sia $\lambda$ un cardinale infinito $\ge|L|$. Ogni teoria $\lambda\jj$categorica senza modelli finiti \`e completa. Mostrare che l'ipotesi $|L|\le\lambda$ \`e necessaria. (Suggerimento: sia $\lambda$ un cardinale infinito. Il linguaggio contiene solo costanti, una per ogni ordinale $<\lambda$. La teoria $T$ dice che esistono infiniti elementi e, o $i=0$ per ogni $i<\lambda$, o $i\neq j$ per ogni $i<j<\lambda$.)\QED
% \begin{proof}
% Dati $M$ ed $N$ modelli di $T$, vogliamo dimostrare che $M\equiv N$. Per il teorema di L\"owenheim-Skolem all'ins\`u  esistono due modelli $M'\succeq M$ ed $N'\succeq N$ di cardinalit\`a maggiore di $\lambda$. Per L\"owenheim-Skolem all'ingi\`u esistono due modelli $M''\preceq M'$ ed $N''\preceq N'$ di cardinalit\`a $\lambda$. Per $\lambda\jj$categoricit\`a $M''$ ed $N''$ sono isomorfi, in particolare $M''\equiv N''$. Quindi $M\equiv N$. 
% \end{proof}
\end{exercise}

\begin{exercise}
Si dimostri che le teorie $T_{\rm oldse}$  e $T_{\rm rg}$ non sono $\lambda\jj$categoriche per nessun cardinale $\lambda$ non numerabile. Suggerimento: la prima affermazione \`e semplice, per la seconda si dimostri l'esistenza di un un grafo aleatorio $N$ di cardinalit\`a $\lambda$ con solo nodi da cui escono $\lambda$ archi e lo si confronti con il grafo complementare (quello che ha un arco esattamente dove $N$ non ce l'ha).\QED
\end{exercise}

\section{Un esercizio risolto}\label{eserciziorisolto}

Nel prossimo capitolo incontreremo modelli $\omega\jj$ricchi di teorie non $\omega\jj$categoriche. \`E utile vedere prima un esempio in un contesto pi\`u semplice.

Per $\alpha\le\omega$ il linguaggio $L_\alpha$ estende quello degli ordini stretti con i predicati unari $r_i$ per $i<\alpha$. La teoria  $T_\alpha$ afferma che $<$ definisce un ordine lineare e che $\neg \big[r_i(x)\wedge r_j(x)\big]$ per ogni $i<j<\alpha$. Denotiamo con 

\ceq{\hfill\R_\alpha}{=}{\big\{M\ :\ M \textrm{ modello }\omega\textrm{-ricco di }T_\alpha\big\}}

Vogliamo assiomatizzare la teoria $\Th(\R_\alpha)$. Il caso $\alpha<\omega$ \`e semplice perch\'e $\R_\alpha$ \`e elementare e non differisce da quanto visto nei paragrafi~\ref{ordinilinearidensi} e~\ref{grafoaleatorio}.

\begin{proposition}\label{prop_ex_R_n}
Dato $\alpha<\omega$, sia $R_\alpha$ la teoria che  estende $T_\alpha$ con l'assioma che dice che l'ordine non ha estremi e le seguenti varianti dell'assioma di densit\`a:
\begin{itemize}
\item[d$_i$] $x<y\ \imp\ \E z\,[\,x<z<y\ \wedge\ r_iz]$ \ \ \ \ per ogni $i<\alpha$;
\item[d$_\alpha$] $\displaystyle x<y\ \imp\ \E z\,\Big[ x<z<y\ \wedge\ \bigwedge_{i<\alpha} \neg r_iz\Big]$.
\end{itemize}
Allora $\R_\alpha=\Mod(R_\alpha)$ e quindi $R_\alpha$ \`e una teoria $\omega\jj$categorica che ammette eliminazione dei quantificatori.
\end{proposition}


\begin{proof}
Per verificare la coerenza di $R_\alpha$ si consideri $\QQ$, con l'ordine naturale e come $r_i$ si prenda l'insieme dei razionali della forma: 

\hfil $n+\dfrac{m}{p_i^k}$\hfil dove $n\in\ZZ$ ed $m,k\in\NN$ tali che $0<m<p^k_i$,

dove con $p_i$ denotiamo l'$i\jj$esimo numero primo. 

La dimostrazione del lemma~\ref{lem_ordinericco} pu\`o essere facilmente adattata per dimostrare che tutti i modelli di $R_\alpha$ sono $\omega\jj$ricchi e quindi che $\R_\alpha=\Mod(R_\alpha)$. Segue quindi anche che $R_\alpha$ \`e $\omega\jj$categorica e per il lemma~\ref{imparzialemappeelementari} ha eliminazione dei quantificatori.
\end{proof}

Invece $\R_\omega$ non  \`e una classe elementare, mostriamo come assiomatizzare $\Th(\R_\omega)$. 

\begin{proposition}\label{prop_ex_R_omega}
Sia $R_\omega$ la teoria che  estende $T_\omega$ con l'assioma che dice che l'ordine non ha estremi e gli assiomi \ssf{d$_i$} della proposizione~\ref{prop_ex_R_n} per ogni $i<\omega$. Allora $\ccl(R_\omega)=\Th(\R_\omega)$. Inoltre $R_\omega$ ammette eliminazione dei quantificatori.
\end{proposition}

\begin{proof}
La consistenza di $R_\omega$ segue per compattezza dalla consistenza di $R_\alpha$ per ogni $\alpha<\omega$. Ora, posto $p(x) = \{\neg r_ix\ :\ i<\omega\}$, il lettore pu\`o facilmente verificare che un modello di $R_\omega$ \`e ricco se e solo se realizza tutti i tipi  (per tutte le possibili coppie di parametri $a,b$) della forma

\begin{itemize}
\item[d$_\omega$] $\displaystyle a<b\ \imp\ \big[ a<x<b\ \wedge\ p(x)\big]$.
\end{itemize}

Per concludere che $\ccl(R_\omega)=\Th(\R_\omega)$ \`e sufficiente verificare che ogni modello numerabile di $R_\omega$ ha un'estensione elementare ad un modello che realizza tutti i tipi della forma \ssf{d$_\omega$}. Lasciamo al lettore anche questa verifica. Il modello richiesto pu\`o essere costruito con una catena elementare di lunghezza $\omega$ in cui $M_{i+1}$ realizza tutti i tipi della forma \ssf{d$_\omega$} con $a,b\in M_i$. L'unione della catena \`e il modello richiesto. 

Per dimostrare l'eliminazione dei quantificatori serve applicare il lemma~\ref{imparzialemappeelementari} ma indebolendo le ipotesi come osservato nell'esercizio~\ref{ex_imparzialemappeelementari}. Infatti, come notato sopra ogni modello numerabile di $R_\omega$ ha un estensione ad un modello numerabile ricco.
\end{proof}






\begin{exercise}\label{ex_infiniti_colori_1}
Per $T$ una delle teorie elencate qui sotto, denotiamo con 

\ceq{\hfill\R}{=}{\big\{M\ :\ M \textrm{ modello }\omega\textrm{-ricco di }T\big\}}

si assiomatizzi $\Th(\R)$ e si dica se \`e $\omega\jj$categorica e se ammette eliminazione dei quantificatori. Nel caso $\R$ non sia una classe elementare, si indichi cosa contraddistingue i modelli ricchi di $\Th(\R)$ dai modelli non ricchi.


\begin{itemize}
\item[1.] Il linguaggio consiste di due simboli di relazione binaria $<$ ed $r$. La teoria $T$ afferma che $<$ definisce un ordine lineare ed $r$ una relazione di equivalenza.
\item[3.] Il linguaggio consiste dei predicati unari $r_i$ per $i<\alpha$.  La teoria $T$ \`e vuota.  Si distinguano due casi: $\alpha<\omega$ e $\alpha=\omega$.\QED
\end{itemize}
\end{exercise}
